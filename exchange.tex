\chapter{The Logic of Information Exchange}
\label{ch:exchange}

\subsection{Summary}

The previous chapters treated the computational interpretations of
disjunctive formulae like $(\phi\imp\psi)\lor(\psi\imp\phi)$ or
$(\phi\limp\psi)\oplus(\psi\limp\phi)$.  In this chapter, we try
replacing these disjunctions with conjunctions.
In the former case, the change renders the logic inconsistent.
If we add the axiom $(\phi\imp\psi)\land(\psi\imp\phi)$ to the
intuitionistic propositional logic,
we can prove any formula.  However in the lattar case, the change does
not make the system meaningless.
In this chapter, we treat
the axioms of the form $(\phi\limp\psi)\otimes(\psi\limp\phi)$
on top of IMLL2, the second order formulation of intuitionistic
multiplicative linear
logic.  In essence, the axiom allows two processes to wait for one
another and then exchange information.

