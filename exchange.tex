\chapter{The Logic of Information Exchange}
\label{ch:exchange}

\subsection{Summary}

The previous chapters treated the computational interpretations of
disjunctive formulas like $(\phi\imp\psi)\lor(\psi\imp\phi)$ or
$(\phi\limp\psi)\oplus(\psi\limp\phi)$.  In this chapter, we try
replacing these disjunctions into conjunctions.
In the former case, the change renders the logic inconsistent.
If we add the axiom $(\phi\imp\psi)\land(\psi\imp\phi)$ to the
intuitionistic propositional logic,
we can prove any formula.  However in the lattar case, the change does
not make the system meaningless.  We treat
the axioms of the form $(\phi\limp\psi)\otimes(\psi\limp\phi)$
on top of IMLL2, the second order formulation of intuitionistic
multiplicative linear
logic.  In essence, the axiom allows two processes to wait for one
another and then exchange information.

\subsection{Dynamics}

We define terms\index{term}~$t$ by BNF:
\[
 t ::= x\mid \lambda x.t\mid (t)t\enspace.
\]
We use $\Lambda$ for the set of terms.
A stack\index{stack}~$\pi$ and a process\index{process}~$p$
are defined by BNF:
\begin{align*}
 \pi &::= \epsilon\mid p\cdot\pi\\
 p   &::= t,\pi
\end{align*}
For two stacks $\sigma$ and $\pi$, the concatenation $\sigma\cdot\pi$ is
defined
inductively on $\sigma$: i.e. $\epsilon\cdot\pi = \pi$ and $(e
\cdot \sigma') \cdot \pi = e\cdot (\sigma'\cdot \pi)$.  By the latter
equation, $\cdot$ can be considered associative.
We use $\Pi$ for the set of stacks.

The reduction relations are the smallest preorders on processes
 and on stacks (both denoted by $\red$) that satisfies:
\begin{description}
 \item[(cong)] $p\red p'$ implies $\sigma \cdot p \cdot \pi\red
      \sigma\cdot p'\cdot \pi$ \enspace;
 \item[(cong')] $\pi\red\pi'$ implies $t,\pi\red t,\pi'$\enspace;
 \item[(push)]
	    $(t)u, \pi   \red t,(u,\epsilon)\cdot\pi$\enspace;
 \item[(pull)]
	    $\lambda x.t,(u,\epsilon)\cdot\pi
	     \red
	     t[u/x], \pi$\enspace;
 \item[(ex)]
           $\pi\cdot (c, (t,\pi_0)\cdot \pi_1)\cdot\pi' \cdot (\co
      c,(u,\sigma_0)\cdot \sigma_1) \cdot \pi''\red
      \pi\cdot (u, \sigma_0\cdot \pi_1)\cdot\pi' \cdot (
      t,\pi_0\cdot \sigma_1) \cdot \pi''
      $
      \enspace.
\end{description}


\subsection{Statics}

We define an IMLL2 formula~$\phi$ by BNF:
\[
 \phi :: =
 X \mid
 \phi\limp\phi\mid
 \forall X\phi\enspace.
\]
As abbreviations, we can introduce $\otimes$ and $\bot$:
\begin{align*}
 \phi\otimes\psi &\equiv \forall X ((\phi\limp\psi\limp X)\limp X)
 \text{ where }\phi\otimes\psi\text{ does not contain } X \text{ freely.}\\
 \bot &\equiv \forall X X\enspace.
\end{align*}

The type system employs conjunctive hypersequents.  Usually,
different components in a hypersequent are interpreted disjunctively.
However, components in a conjunctive hypersequent are interpreted
conjunctively by $\otimes$.

 \begin{figure}
\centering
\AxiomC{}
\LL  {Ax}
\useq{\xphi}{\xphi}
\DisplayProof
%
\hfill
\AxiomC{$\mathcal H$}
\AxiomC{$\mathcal H$}
\LL{mix}
\BinaryInfC{$\mathcal H\hmid \mathcal{H'}$}
\DisplayProof
%
\ruleskip
\aseq{\hyp{}\xphi,\G}{\tj t\psi}
\LL{$\limp$I}
\useq{\hyp{}\G}{\tj{\lambda x.t}{\phi\limp\psi}}
\DisplayProof
\ruleskip
%
\aseq{\hyp{}\G}{\tj t{\phi_0\limp\cdots\limp \phi_n \limp\psi}}
\aseq{\D_0}{\tj{u_0}{\phi_0} \hmid \cdots \hmid \D_n\tr\tj{u_n}{\phi_n}}
\LL   {$\limp$E}
\bseq{\hyp{}\G,\D_0,\ldots,\D_n}{\tj{(((t)u_0)\cdots) u_n}\psi}
\DisplayProof
\ruleskip
\aseq{\hyp{}\G}{\tj t\phi}
\LL   {$\forall$I}
\useq{\hyp{}\G}{\tj t{\forall X\phi}}
\DisplayProof (no free $X$ in $\mathcal H$ or $\G$)
\hfill
%
\aseq{\hyp{}\G}{\tj{t}{\forall X\phi}}
\LL   {$\forall$E}
\useq{\hyp{}\G}{\tj{t}{\phi[\psi/X]}}
\DisplayProof
\ruleskip
%
\aseq{\hyp{}\G}{\tj{t}{\phi}\hmid \D\tr\tj{u}{\psi}}
\LL  {com}
\useq{\hyp{}\G}{\tj{(c)t}{\psi}\hmid \D\tr\tj{(\co c)u}{\phi}}
\DisplayProof ($c$ is fresh)
\ruleskip
%
% \aseq{\hyp{}\G}{\tj{t}{\phi}\hmid \D\tr\tj{u}{\psi}}
%   \LL{$\otimes$I}
% \useq{\hyp{}\G,\D}{\tj{(t\conc u)}{\phi\otimes \psi}}
% \DisplayProof
%   %
%   \ruleskip
%   \aseq{\hyp{}\G}{\tj{t}{\phi\otimes\psi}\hmid
%   \tj{x}{\phi},\tj{y}{\psi},\D\tr \tj{u}{\theta}}
%   \LL{$\otimes$E}
%   \useq{\hyp{}\hyp'\G,\D}{\tj{\letpar t x y u}{\theta}}
%   \DisplayProof
  \caption{The typing rules of $\NMLL$.  Exchange rules are omitted.}
 \end{figure}

\subsection{Specification Using Poles}

\begin{definition}
 \label{ex:def:pole}
A pole\index{pole}~$\bbot$ is a set of stacks
which satisfies
\begin{enumerate}
 \item \label{ex:red-closed} $\pi$ is in $\bbot$ if $\pi\red \sigma$ and
       $\sigma\in\bbot$; and
 \item \label{ex:conc-closed} $\pi\cdot \sigma$ is in $\bbot$ if $\pi$
       and $\sigma$ are in $\bbot$.
\end{enumerate}
\end{definition}

For a set~$\mathcal Z$ of
stacks, $\mathcal Z\rightarrow\bbot$ denotes
the set of
terms
$t$ such that
for any
stack $\pi\in\mathcal Z$,
the single-process stack $(t,\pi)\cdot\epsilon$ is in $\bbot$.

\fix{define tuple nsem}

\fix{define Form}

For $\phi\in\form(2^E)$ and $|\cdot|^-_0\colon\pvar\rightarrow 2^E$\kern
-2pt,
we define $\nsem{\phi}\in
2^E$ inductively:
\begin{align*}
 \nsem{\mathcal Z} =& \mathcal Z \text{ for } \mathcal Z\in 2^E\\
 \nsem{X}=& |X|_0^- \\
 %%%
 \nsem{\phi_0\limp\cdots\limp\phi_n\limp\psi}=&
 \{e_0\cdot e_1\cdot\cdots\cdot e_n\cdot\pi\mid \\ &\,\,\,  e_0\cdot
 e_1\cdot\cdots\cdot e_n
 \in(\nsem{\phi_0}, \nsem{\phi_1}, \ldots, \nsem{\phi_n})\rightarrow\bbot
 \text{ and  }
 \\&\,\,\,\pi\in\nsem\psi\}\text{ where }\psi\text{ is not of the form }\psi_0\limp\psi_1\\
 %%%
 \nsem{\forall X\phi}=&
 \bigcup_{\mathcal Z\in 2^\sPi} \nsem{\phi[\mathcal Z/X]}\enspace.
\end{align*}
Using this, we define $\sem \phi=\nsem{\phi}\rightarrow\bbot$.
\fix{define tuple positive sem}
 \begin{proposition}
  \label{nsem-tuple}
  When $\pi$ is in
  $\left(\mathcal{Z}_0,\ldots,\mathcal{Z}_m\right)\rightarrow \bbot$
  and $\sigma$ is in
  $\left(\mathcal{Z}_{m+1},\ldots,\mathcal{Z}_n\right)\rightarrow \bbot$,
  then,
  the concatenation $\pi\cdot\sigma$ is in
  $\left(\mathcal{Z}_0,\ldots,\mathcal{Z}_m,
  \mathcal{Z}_{m+1},\ldots,\mathcal{Z}_n\right)$.
 \end{proposition}
  \begin{proof}
   By condition~\ref{ex:conc-closed} of Definition~\ref{ex:def:pole}.
  \end{proof}

 \begin{proposition}
  \label{nsem-imp}
  If $\sigma$ is in
  $(\nsem{\phi_0},\nsem{\phi_1},\ldots,\nsem{\phi_m})\rightarrow\bbot$
  and $\pi\in\nsem{\psi}$.
  $\nsem{\phi_0\limp\phi_1\limp\cdots\limp\phi_m\limp\psi}$.
 \end{proposition}
  \begin{proof}
   Let $\psi$ be $\phi_{m+1}\limp\cdots\phi_n\limp\psi'$ where $\psi'$
   is not of the form $\psi'_0\limp\psi'_1$.
   Take such $e_0,\ldots,e_m$ and $\pi$.
   Since $\pi$ is in $\nsem{\psi}$,
   it is of the form
   \[
    e_{m+1}\cdot e_{m+2}\cdot \cdots \cdot e_n\cdot \pi'
   \]
   where $e_{m+1}\cdot e_{m+2}\cdot \cdots e_n$ is in
   $(\nsem{\phi_{m+1}}, \ldots, \nsem{\phi_n})\rightarrow\bbot$
   and $\pi'$ is in $\nsem{\psi'}$.
   By Proposition~\ref{nsem-tuple},
   $ e_0\cdot\cdots\cdot e_m\cdot e_{m+1}\cdot e_{m+2}\cdot \cdots \cdot e_n $ is in
   $\left(\nsem{\phi_0},\ldots,\nsem{\phi_m},\nsem{\phi_{m+1}}, \ldots,
   \nsem{\phi_n}\right)\rightarrow \bbot$.  The statement follows.
  \end{proof}

For $\G = \tj{x_1}{\phi_1},\ldots,\tj{x_n}{\phi_n}$,
we denote by $\sem{\G}$ the set of sequences $(t_1,\dots,t_n)$
 where each term $t_i$ is in $\sem{\phi_i}$.
For that sequence~$\vec t$, $[\vec{t}/\G]$ denotes the simultaneous substitution
$[t_i/x_i]_{0\le i \le n}$.
For a hypersequent, we
define a set of s-executables
$
\semo{\G_0\tr\tj{t_0}{\phi_0}\hmid\cdots\hmid\G_n\tr\tj{t_n}{\phi_n}}
$
to be the set of executables of the form
$
\left(t_0[\vec{g_0}/\G_0],\pi_0\right)\cdot\cdots\cdot
\left(t_n[\vec{g_n}/\G_n],\pi_n\right)
$
 where
$\vec{g_i} \in \sem{\G_i}$ and $\pi_i\in \nsem{\phi_i}$.

 \begin{theorem}[Adequacy]
  If a hypersequent~$\mathcal H$ is derivable, the stacks in $\semo{\mathcal
  H}$ are in $\bbot$.
 \end{theorem}
  \begin{proof}
   By induction on the derivation of $\mathcal{H}$.
   \begin{description}
    \item[($\limp$E)]
	 Take any $\pi\in\nsem{\psi}$, $\vec g\in\sem{\G}$,
	 $\vec{d_i}\in\sem{\D_i}$ for each $0\le i\le n$ and
	 $h\in\semo{\mathcal H}$.
	 We have to show that this stack is in $\bbot$:
	 \[
	 \sigma = h\cdot \left(t[\vec g/\G],
	 (u'_0,\epsilon)\cdot\cdots\cdot (u'_n,\epsilon)\cdot \pi\right)\enspace.
	 \]
	 By induction hypothesis on the derivation of $u_i$'s,
	 $ (u'_0,\epsilon)\cdot\cdots\cdot (u'_n,\epsilon) $ is in
	 $\sem{\phi_0,\ldots,\phi_n}$.
	 So, by Proposition~\ref{nsem-imp},
	 $(u'_0,\epsilon)\cdot\cdots\cdot (u'_n,\epsilon)\cdot \pi$ is
	 in $\nsem{\phi_0\limp\cdots\limp\phi_n\limp\psi}$.
	 By induction hypothesis on the derivation of $t$,
	 the reduct is in $\bbot$.
	 Since $\bbot$ is closed for $\rev\red$, $\sigma$ is also in $\bbot$.
   \end{description}
  \end{proof}

  \begin{proposition}[Specification of the axiom]
   Let $g$ have type ${\forall X\forall
   Y((X\limp Y)\times (Y\limp X))}$.
   Then, the s-executable
   \[
   [g, \mats{z}{z}{x\cdot\pi_Y}{w}{w}{y\cdot\pi_X}]
   \]
   reduces to a multiset on
   $\{(x,\pi_X), (y,\pi_Y)\}$.
  \end{proposition}

\subsection{Consistency}

\begin{proposition}
  No term has the type $\forall X X$.
\end{proposition}
\begin{corollary}
 The system $\NMLL$ is consistent.
\end{corollary}
