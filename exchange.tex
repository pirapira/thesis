\chapter{The Logic of Information Exchange}
\label{ch:exchange}

\subsection{Summary}

After studying $(\phi\limp\psi)\oplus (\psi\limp\phi)$,
it is natural to change the additive disjunction $\oplus$ into
multiplicative conjunction $\otimes$ and see what happens%
\footnote{Takeuti Izumi asked what about conjunctions.}.
A natural way to add $(\phi\limp\psi)\otimes(\psi\limp\phi)$ as an axiom
is adding a pair of primitives $\lpair{c,\co c}$ so that
$\cdots ct \cdots \co c u \cdots$ reduces to
$\cdots u  \cdots t \cdots$: in words,
$c$ outputs $\co c$'s input and vice versa.
We can obtain the send-receive communication when we specialize the
axiom as $(\phi\limp \one)\otimes(\one\limp\phi)$.  The left hand side
$\tj{c}{\phi\limp\one}$ is the sending primitive and
the right hand side $\tj{\co c}{\one \limp\phi}$ is the receiving
primitive.

When we want to use these primitives anywhere in lambda terms,
there is one problem: what happens to $\co c(c t)$?
In this case, we do not know the output of $c$ because we only have an
equality between $\co c$'s input and $c$'s output.
Fortunately, we just want to know the output of $\co c$, which is the
input of $c$, that is, $t$.
In a more complicated case $\co c(\co d(c (d t)))$,
we can reason the output of $\co c$ as the input of $c$ as the output of
$d$ as the input of $\co d$ as the output of $c$ as the input of $\co c$
as the output of $\co d$ as the input of $d$, which is $t$.
\fix{how does it generalize?}

\section{Definitions}

\subsection{Types}
For a set $S$, we define $\phi\in\form(S)$ by BNF:
\[
 \phi::=s\mid \one \mid X \mid \phi\otimes\phi\mid \phi\limp\phi\mid
 \phi\oplus\phi\mid \phi\with\phi\mid !\phi\mid \forall X.\phi
\]
where $s$ is in $S$ and $X$ is a propositional variable.
The propositional variable in the last clause is binding.
A type is an element of $\form(\emptyset)$.

\subsection{Terms and Free Variables}

Following Abramsky~\citep{abramsky1993computational}, we defne patterns
binding sets of
variables:
\begin{itemize}
 \item $\ast$ and $\_$ are patterns binding $\emptyset$,
 \item $\lpair{x,\_}$, $\lpair{\_,x}$ and $!x$ are patterns binding
       $\{x\}$,
 \item $x\otimes y$ and $x@y$ are patterns binding $\{x,y\}$.
\end{itemize}
Using patterns, we define a term $t$ with free variables~$S$.
We assume countably infinitely many \textit{channels}\index{channel}
with involution
satisfying $\co c\neq c$ and $\co{\co c} = c$.
\begin{itemize}
 \item a variable $x$ is a term with free variables $\{x\}$,
 \item if $t$ is a term with free variables~$S$, $u$ is a term with
       free variables~$S'$ and $S$ and $S'$ are disjoint, $t\otimes u$ and
       $tu$ are terms with free variables $S\cup S'$,
 \item if $t$ and $u$ are terms with free variables $S$, then
       $\lpair{t,u}$ is a term with free variables~$S$,
 \item if $t$ is a term with free variables~$S$, then
       $\inl x, \inr t$ and $!t$ are terms with free variables~$S$
 \item if $t$ is a term with free variables $S\cup \{x\}$ and $x$ is not
       in $S$, then $\lambda x.t$ is a term with free variables~$S$,
 \item if $t$ is a term with free variables~$S$, $p$ is a pattern
       binding $S'$, $u$ is a term with free variables $S'\cup S''$ and
       $S\cap S'' = S'\cap S'' = \emptyset$, then,
       $\letin t p u$ is a term with free variables $S\cup S''$,
 \item if $t$ is a term with free variables $S$,
       $u$ is a term with free variables $S''\cup \{x\}$,
       $v$ is a term with free variables $S''\cup \{y\}$,
       $x,y\notin S''$ and $S\cap S'' = \emptyset$,
       $\mat t x u y v$ is a term with free variables $S\cup S''$, and
 \item channels are terms with free variables~$\emptyset$.
\end{itemize}
Note that a term with free variables $S$ is not a term with free
variables $S'$ when $S\neq S'$.  Only the last clause is original,
introducing channels, which are our communication primitives.

\subsection{Typing Derivations}

On top of Abramsky's~\citep{abramsky1993computational} we add a rule to
make $(\phi\limp\psi)\otimes(\psi\limp\phi)$ derivable.

A hypersequent is inductively defined as
\begin{align*}
 \hypert ::=\, &\G\tr\tj{t}{\phi}
 \mid\, (\hypert\hmid \hypert)
 \mid\, \lpair{\hypert,\hypert}^n
\end{align*}
where $n$ is a natural number and $\G$ is a context. \fix{define context}
The last clause is used to implement the so-called additive box.
\fix{add citation for additive box}
\fix{definition of terms contains natural number?}
Differently from the previous chapters,
here, the hypersequent $\G\tr\phi\hmid \D\tr\psi$ is interpreted as the
conjunction of components:
$(\bigotimes\G\limp\phi)\otimes (\bigotimes\D\limp\psi)$.

The typing rules are in Figure~\ref{fig:exchange:rules}.
 \begin{figure}
  \centering
  % axiom
  \AxiomC{}
  \LL{Ax}
  \UnaryInfC{$\tj{x}{\phi}\tr\tj{x}{\phi}$}
  \DisplayProof
  %
  \hfill
  \AxiomC{$\hypert$}
  \AxiomC{$\hypert'$}
  \LL{merge}
  \BinaryInfC{$\hypert\hmid\hypert'$}
  \DisplayProof
  % cut XXX is this necessary?; yes
  \hfill
  \AxiomC{$\hypert\hmid\G\tr\tj{t}{\phi}\hmid\tj{x}{\phi},\D\tr\tj{u}{\psi}$}
  \LL{Cut}
  \UnaryInfC{$\hypert\hmid\G,\D\tr\tj{u[t/x]}{\psi}$}
  \DisplayProof
  \ruleskip
  % exchange
  \AxiomC{$\hypert\hmid\G,\tj{x}{\phi},\tj{y}{\psi},\D\tr\tj{t}{\theta}$}
  \LL{IE}
  \UnaryInfC{$\hypert\hmid\G,\tj{y}{\psi},\tj{x}{\phi},\D\tr\tj{t}{\theta}$}
  \DisplayProof
  %
  \hfill
  \AxiomC{$\hypert\hmid \G \tr\tj t\phi \hmid \D \tr\tj u\psi\hmid \hypert'$}
  \LL{EE}
  \UnaryInfC{$\hypert\hmid \D \tr\tj u\psi \hmid \G \tr\tj t\phi \hmid \hypert'$}
  \DisplayProof
  \ruleskip
  %
  \ruleskip
  % 1R
  \AxiomC{}
  \LL{$\one$R}
  \UnaryInfC{$\tr\tj{\ast}{\one}$}
  \DisplayProof
  %
  \hfill
  % 1L
  \AxiomC{$\hypert\hmid\G\tr\tj{t}{\phi}$}
  \LL{$\one$L}
  \UnaryInfC{$\hypert\hmid\G,\tj{z}{\one}\tr\tj{t}{\phi}$}
  \DisplayProof
  %
  \ruleskip
  % otimes R
  \AxiomC{$\hypert\hmid\G\tr\tj{t}{\phi}\hmid\D\tr\tj{u}{\psi}$}
  \LL{$\otimes$R}
  \UnaryInfC{$\hypert\hmid\G,\D\tr\tj{t\otimes u}{\phi\otimes \psi}$}
  \DisplayProof
  %
  \hfill
  % sync
  \AxiomC{$\hypert\hmid\G\tr\tj{t}{\phi}\hmid \D\tr\tj{u}{\psi}$}
  \LL{sync}
  \UnaryInfC{$\hypert\hmid
  \G\tr\tj{ct}{\psi}\hmid \D\tr\tj{\co cu}{\phi}$}
  \DisplayProof
  %
  \ruleskip
  % otimes L
  \AxiomC{$\hypert\hmid\G,\tj{x}{\phi},\tj{y}{\psi}\tr\tj{t}{\theta}$}
  \LL{$\otimes$L}
  \UnaryInfC{$\hypert\hmid\G,\tj{z}{\phi\otimes\psi}\tr\tj{\letin{z}{x\otimes
  y}{t}}{\theta}$}
  \DisplayProof
  %
  \ruleskip
  % limp R
  \AxiomC{$\hypert\hmid\G,\tj{x}{\phi}\tr\tj{t}{\psi}$}
  \LL{$\limp$R}
  \UnaryInfC{$\hypert\hmid\G\tr\tj{\lambda x.t}{\phi\limp \psi}$}
  \DisplayProof
  %
  \hfill
  % limp L
  \AxiomC{$\hypert\hmid\G\tr\tj{t}{\phi}\hmid\tj{x}{\psi},\D\tr\tj{u}{\theta}$}
  \LL{$\limp$L}
  \UnaryInfC{$\hypert\hmid\G,\tj{f}{\phi\limp\psi},\D\tr \tj{u[(ft)/x]}{\theta}$}
  \DisplayProof
  %
  \ruleskip
  % andR
  \AxiomC{$\hypert_0\hmid\G\tr\tj{t}{\phi}$}
  \AxiomC{$\hypert_1\hmid\G\tr\tj{u}{\psi}$}
  \LL{$\with$R}
  \BinaryInfC{$\lpair{\hypert_0,\hypert_1}_n
  \hmid\G\tr\tj{\lpair{t,u}_n}{\phi\with\psi}$}
  \DisplayProof \\
  where $\hypert_0$ and $\hypert_1$ are of the same type \fix{define}
  and $n$ is a natural number such that no $\lpair{\_,\_}_n$ appears
  in the branches.
  %
  \ruleskip
  % andL0
  \AxiomC{$\hypert\hmid\G,\tj{x}{\phi}\tr\tj{t}{\theta}$}
  \LL{$\with$L$_0$}
  \UnaryInfC{$\hypert\hmid\G,\tj{z}{\phi\with\psi}\tr\tj{\letin{z}{\lpair{x,\_}}{t}}{\theta}$}
  \DisplayProof
  %
  \ruleskip
  % andL1
  \AxiomC{$\hypert\hmid\G,\tj{y}{\psi}\tr\tj{t}{\theta}$}
  \LL{$\with$L$_1$}
  \UnaryInfC{$\hypert\hmid\G,\tj{z}{\phi\with\psi}\tr\tj{\letin{z}{\lpair{\_,y}}{t}}{\theta}$}
  \DisplayProof
  %
  \ruleskip
  % oplus R0
  \AxiomC{$\hypert\hmid\G\tr\tj{t}{\phi}$}
  \LL{$\oplus$R$_0$}
  \UnaryInfC{$\hypert\hmid\G\tr\tj{\inl{t}}{\phi\oplus\psi}$}
  \DisplayProof
  %
  \hfill
  % oplus R1
  \AxiomC{$\hypert\hmid\G\tr\tj{u}{\psi}$}
  \LL{$\oplus$R$_1$}
  \UnaryInfC{$\hypert\hmid\G\tr\tj{\inr{u}}{\phi\oplus\psi}$}
  \DisplayProof
  %
  \ruleskip
  % oplus L
  \AxiomC{$\hypert \hmid \G,\tj{x}{\phi}\tr\tj{u}{\theta}\hmid
  \G,\tj{y}{\psi}\tr\tj{v}{\theta}$}
  \LL{$\oplus$L}
  \UnaryInfC{$\hypert\hmid\G,\tj{z}{\phi\oplus\psi}\tr\tj{\mat{z}{x}{u}{y}{v}}{\theta}$}
  \DisplayProof
  %
  \ruleskip
  % !R
  \AxiomC{$!\hypert\hmid!\G\tr\tj{t}{\phi}$}
  \LL{$!$R}
  \UnaryInfC{$!\hypert\hmid!\G\tr\tj{!t}{!\phi}$}
  \DisplayProof
  where $!\hypert$ contains only components typed $!\D\tr !\psi$.
  %
  \ruleskip
  % dereliction
  \AxiomC{$\hypert\hmid\G,\tj{x}{\phi}\tr\tj{t}{\psi}$}
  \LL{Dereliction}
  \UnaryInfC{$\hypert\hmid\G,\tj{z}{!\phi}\tr\tj{\letin{z}{!x}{t}}{\psi}$}
  \DisplayProof
  %
  \ruleskip
  % contraction
  \AxiomC{$\hypert\hmid\G,\tj{x}{!\phi},\tj{y}{!\phi}\tr\tj{t}{\psi}$}
  \LL{Contraction}
  \UnaryInfC{$\hypert\hmid\G,\tj{z}{!\phi}\tr\tj{\letin{z}{x@y}{t}}{\psi}$}
  \DisplayProof
  %
  \ruleskip
  % weakening
  \AxiomC{$\hypert\hmid\G\tr\tj{t}{\psi}$}
  \LL{Weakening}
  \UnaryInfC{$\hypert\hmid\G,\tj{z}{!\phi}\tr\tj{\letin{z}{\_}{t}}{\psi}$}
  \DisplayProof
  %
  \ruleskip
  % forall R
  \AxiomC{$\hypert\hmid\G\tr\tj{t}{\phi}$}
  \LL{$\forall$R}
  \UnaryInfC{$\hypert\hmid\G\tr\tj{t}{\forall X.\phi}$}
  \DisplayProof (No free $X$ appears in $\G$ or $\hypert$)
  %
  \hfill
  % forall L
  \AxiomC{$\hypert\hmid\G,\tj{x}{\phi[\psi/X]}\tr\tj{t}{\phi}$}
  \LL{$\forall$L}
  \UnaryInfC{$\hypert\hmid\G,\tj{x}{\forall X.\phi}\tr\tj{t}{\phi}$}
  \DisplayProof
  %
  \ruleskip
  %
  \caption{Most rules are taken from Abramsky~\citep{abramsky1993computational}.
  The sync rule is original.}
  \label{fig:exchange:rules}
 \end{figure}
For example,
the formula $(\phi\limp\psi)\otimes(\psi\limp\phi)$ is realized by
the following derivation.
 \begin{center}
  \AxiomC{}
  \UnaryInfC{$\tj{x}{\phi}\tr\tj{x}{\phi}$}
  \AxiomC{}
  \UnaryInfC{$\tj{y}{\psi}\tr\tj{y}{\psi}$}
  \BinaryInfC{$\tj{x}{\phi}\tr\tj{x}{\phi} \hmid
  \tj{y}{\psi}\tr\tj{y}{\psi}$}
  \UnaryInfC{$ \tj{x}{\phi}\tr\tj{cx}{\psi} \hmid \tj{y}{\psi}
  \tr\tj{\co c y}{\phi}$}
  \UnaryInfC{$ \tr\tj{\lambda x.cx}{\phi\limp\psi} \hmid \tj{y}{\psi}
  \tr\tj{\co c y}{\phi}$}
  \UnaryInfC{$ \tr\tj{\lambda x.cx}{\phi\limp\psi} \hmid
  \tr\tj{\lambda y.\co c y}{\phi\limp\phi}$}
  \UnaryInfC{$ \tr\tj{{(\lambda x.cx) \otimes (\lambda y.\co c
  y)}}{(\phi\limp\psi)\otimes(\phi\limp\phi)}$}
  \DisplayProof
 \end{center}
Another example shows how we can type the term $\co c(c x)$.
 \begin{center}
\AxiomC{$\tj{x}{\phi}\tr\tj{x}{\phi}$}
\AxiomC{$\tj{y}{\psi}\tr\tj{y}{\psi}$}
\BinaryInfC{$\tj{x}{\phi}\tr\tj{x}{\phi} \hmid
\tj{y}{\psi}\tr\tj{y}{\psi} $}
\UnaryInfC{
$\tj{x}{\phi}\tr\tj{cx}{\psi} \hmid
\tj{y}{\psi}\tr\tj{\co cy}{\phi} $
}
\LL{(Cut)}
\UnaryInfC{
$\tj{x}{\phi}\tr\tj{\co c(cx)}{\phi}$
}
\DisplayProof
 \end{center}

\subsection{Evaluation Relation}

The set of canonical forms remains the same as Abramsky's
system~\citep{abramsky1993computational}:
\[
 \lpair{t,u}\qquad !t\qquad \ast\qquad v\otimes w\qquad \lambda
 x.t\qquad \inl{v}\qquad\inr{w}
\]
where $v$ and $w$ are canonical forms.

An evaluation hypersequent $\hypere$ is defined by the following
grammar:
\[
 \hypere ::= t\eval t\mid (\hypere\hmid \hypere)\mid
 \tuple{\hypert_0,\hypert_1}_n\eval \vec t \mid
 \tuple{\hypert_0,\hypert_1}_n\eval \tuple{\hypert_0,\hypert_1}_n
\]
where the size of $\hypert_0,\hypert_1$ and $\vec t$ have to be the same.

$\mathcal E$ stands for a sequence of evaluation relations.
Most rules are the same as Abramsky's~\citep{abramsky1993computational}.
We add the semantics for channels.
The definition is inductive in Figure~\ref{fig:eval}.

 \begin{figure}
  \centering
  \AxiomC{}
  \UnaryInfC{$\ast\eval\ast$}
  \DisplayProof
  \hfill
  \AxiomC{$\hypere\hmid t\eval \ast \hmid u\eval v$}
  \UnaryInfC{$\hypere\hmid \letin t \ast u\eval v$}
  \DisplayProof
  \hfill
  \AxiomC{$\hypere\hmid t\eval v\hmid u\eval w$}
  \UnaryInfC{$\hypere\hmid t\otimes u\eval v\otimes w$}
  \DisplayProof
  \ruleskip
  \AxiomC{$\hypere\hmid t\eval v\otimes d\hmid u[v/x,w/y]\eval v'$}
  \UnaryInfC{$\hypere\hmid \letin t {x\otimes y} u \eval v'$}
  \DisplayProof
  \hfill
  \AxiomC{$\hypere$}
  \AxiomC{$\hypere'$}
  \BinaryInfC{$\hypere\hmid \hypere'$}
  \DisplayProof
  \hfill
  \AxiomC{}
  \UnaryInfC{$\lambda x.t\eval \lambda x.t$}
  \DisplayProof
  \ruleskip
  \AxiomC{$\hypere\hmid t\eval \lambda x.t'\hmid u\eval v\hmid
  t'[v/x]\eval w$}
  \UnaryInfC{$\hypere\hmid tu\eval w$}
  \DisplayProof
  \hfill
  \AxiomC{$\hypere\hmid t\eval v\hmid u\eval w$}
  \UnaryInfC{$\hypere\hmid ct\eval w\hmid \co cu\eval v$}
  \DisplayProof
  \ruleskip
  \AxiomC{}
  \UnaryInfC{$\lpair{\hypert_0,\hypert_1}_n\eval
  \lpair{\hypert_0,\hypert_1}_n
  \hmid
  \lpair{t, u}_n\eval
  \lpair{t, u}_n$}
  \DisplayProof
  \ruleskip
  \AxiomC{$\hypere\hmid \lpair{\hypert_0,\hypert_1}_n \eval
  \lpair{\hypert_0,\hypert_1}_n
  \hmid t\eval\lpair{t_0,t_1}_n\hmid \hypert_0\eval \vec{o_0}\hmid t_0\eval v_0
  \hmid  u[v_0/x]\eval w$}
  \UnaryInfC{$\hypere \hmid
  \lpair{\hypert_0,\hypert_1}_n \eval \vec{o_0} \hmid
  \letin t {\lpair{x,\_}} u\eval w$}
  \DisplayProof
  \fix{define the vect abbreviation}
  \ruleskip
  \AxiomC{$\hypere\hmid \lpair{\hypert_0,\hypert_1}_n \eval
  \lpair{\hypert_0,\hypert_1}_n
  \hmid t\eval\lpair{t_0,t_1}_n\hmid \hypert_1\eval \vec{o_1}\hmid t_1\eval v_1
  \hmid  u[v_1/y]\eval w$}
  \UnaryInfC{$\hypere \hmid
  \lpair{\hypert_0,\hypert_1}_n \eval \vec{o_1} \hmid
  \letin t {\lpair{\_,y}} u\eval w$}
  \DisplayProof
  \ruleskip
  \AxiomC{$\hypere\hmid t\eval v$}
  \UnaryInfC{$\hypere\hmid \inl{t}\eval \inl{v}$}
  \DisplayProof
  \hfill
  \AxiomC{$\hypere\hmid u\eval w$}
  \UnaryInfC{$\hypere \hmid \inr{u}\eval \inr{w}$}
  \DisplayProof
  \ruleskip
  \AxiomC{$\hypere\hmid t\eval \inl{v}\hmid u[v/x]\eval w$}
  \UnaryInfC{$\hypere\hmid \mat t x u y {u'}\eval w$}
  \DisplayProof
  \ruleskip
  \AxiomC{$\hypere\hmid t\eval \inr{v}\hmid u'[v/y]\eval w$}
  \UnaryInfC{$\hypere\hmid \mat t x u y {u'}\eval w$}
  \DisplayProof
  \ruleskip
  \AxiomC{}
  \UnaryInfC{$\hypere\hmid !t\eval !t$}
  \DisplayProof
  \hfill
  \AxiomC{$\hypere\hmid t\eval !s\hmid s\eval v\hmid u[v/x]\eval w$}
  \UnaryInfC{$\hypere \hmid \letin t {!x} u\eval w$}
  \DisplayProof
  \ruleskip
  \AxiomC{$\hypere\hmid t\eval !s\hmid u\eval v$}
  \UnaryInfC{$\hypere \hmid \letin t {\_} u\eval v$}
  \DisplayProof
  \hfill
  \AxiomC{$\hypere\hmid t\eval !s\hmid u[!s/x,!s/y]\eval v$}
  \UnaryInfC{$\letin t {x@y} u\eval v$}
  \DisplayProof
  \ruleskip
  \AxiomC{$\hypere\hmid t\eval v\hmid u[v/x]\eval w$}
  \LL{eval-subst}
  \UnaryInfC{$ \hypere\hmid u[t/x] \eval w $}
  \DisplayProof
  \caption{The definition of evaluation relation.}
  \label{fig:eval}
 \end{figure}

\section{Convergence}

Convergence states that if
$\tr\tj t \phi$ is derivable,
there exists a canonical form~$v$ so that $t\eval v$ is derivable.
Since the proof of convergence is inductive over
the typing derivations, we have to generalize the statement.

For that we first assign a set of closed terms to each type.
\newcommand{\terms}{\mathcal{T}}
\fix{use of $v$ is ambiguious...?}
We denote by $\terms$ the set of closed terms.
Given $\sem{X}_0\in 2^\terms$ for each propositional variable~$X$,
we define $\sem{\cdot}\colon \form{2^\terms}\rightarrow \form{2^\terms}$
as follows:
\begin{align*}
 \sem{\mathcal X} &= \mathcal X\\
 \sem{X} &= \sem{X}_0\\
 \sem{\one} &= \{t\in \terms \mid t\eval \ast\} \\
 \sem{\phi\otimes \psi}&= \{t\in\terms \mid t\eval v\otimes w,\quad
 v\in\sem{\phi},\quad w\in \sem{\psi}\}\\
 \sem{\phi\limp\psi}&= \{t\in\terms \mid t\eval \lambda x.v,\quad
 \forall u\in\sem{\phi}.(tu\in\sem{\psi})\}\\
 \sem{\phi\with\psi}&= \{t\in\terms \mid t\eval\lpair{u,v},\quad
 u\in\sem{\phi}, v\in\sem{\psi}\}\\
 \sem{\phi\oplus\psi}&= \{t\in\terms\mid (t\eval \inl{v}, v\in
 \sem{\phi})\text{ or }(t\eval\inr{w}, w \in \sem{\psi})\}\\
 \sem{!\phi}&= \{t\in\terms \mid t\eval !u, \quad u\in\sem{\phi}\}\\
 \sem{\forall X.\phi}&= \bigcap\{\sem{\phi[\mathcal X/X]}\mid \psi\in
 2^\terms\}\enspace.
\end{align*}

For a context~$\G = \tj{x_0}{\phi_0},\ldots,\tj{x_n}{\phi_n}$,
$\semo{\G}$ denotes the set of sequences $u_0,\ldots,u_n$ of closed terms
such that each $u_i$ is in $\sem{\phi_i}$.

 \begin{proposition}[General Convergence]
  \label{thm:generalconvergence}
  If
  $\G_0\tr\tj{t_0}{\phi_0}\hmid\cdots\hmid \G_n\tr\tj{t_n}{\phi_0}$
  is derivable,
  for any $(\vec{u^i}) \in \semo{\G_i}$ for ${0\le i \le n}$,
  there exist canonical forms $(v_i\in\semo{\phi_i})_{0\le i\le n}$ so
  that $t_0[\vec{u^0}/ \G_0]\eval v_0\hmid\cdots\hmid
  t_n[\vec{u^n}/\G_n]\eval v_n$ is derivable.
 \end{proposition}
  \begin{proof}
   By induction on the type derivation.
   \begin{description}
    \item[($\limp$L)] The derivation ends as
	  \begin{center}
	   \AxiomC{$\hypert\hmid \G\tr\tj{t}{\phi}\hmid \tj{x}{\psi},
	   \D\tr\tj{u}{\theta}$}
	   \UnaryInfC{$\hypert\hmid
	   \G,\tj{f}{\phi\limp\psi},\D\tr\tj{u[(ft)/x]}{\theta}$}
	   \DisplayProof\enspace.
	  \end{center}
	 Take any $O\in\semo{\hypert}$, $\vec g\in\semo{\G}$, $\vec
	 d\in\semo{\D}$ and $v\in\semo{\phi\limp\psi}$.
	 By induction hypothesis,
	 \[
	  \hypert\eval \vec{w'}\hmid t[\vec g/\G]\eval
	   w_0\hmid u[w/x][\vec d/\D]\eval w
	 \]
	 is derivable for any $w \in \semo{\psi}$.
	 Since $w_0$ is in $\semo{\phi}$ and $v$ is in
	 $\semo{\phi\limp\psi}$,
	 $fw_0$ is in $\psi$.
	 Thus,
	 \[
	 \hypert\eval\vec{w'}\hmid t[\vec g/\G]\eval w_0\hmid
	 u[(fw_0)/x][\vec d/\D]\eval w_1
	 \]
	 is derivable.
	 The last term $u[(fw_0)/x][\vec d/\D]$ is identical to
	 $u[fy/x][w_0/y][\vec d/\D]$.
	 Thus, by rule eval-subst in Figure~\ref{fig:eval},
	 \[
	  \hypert\eval\vec{w'}\hmid u[fy/x][t/y][\vec d/\D]\eval w_1
	 \]
	 is derivable.
    \item[(Cut)] Essentically the same as and simpler than ($\limp$L).
    \item[(Sync)]
	 The derivation ends as
	 \AxiomC{$\hypert\hmid \G\tr\tj{t}{\phi}\hmid \D\tr\tj{u}{\psi}$}
	 \UnaryInfC{$\hypert\hmid \G\tr\tj{ct}{\psi}\hmid \D\tr\tj{\co
	 cu}{\phi}$}
	 \DisplayProof\enspace.
	 Take any $O\in\semo{\hypert}$, $\vec g\in \semo{\G}$ and $\vec
	 d\in\semo{\D}$.
	 By induction hypothesis, the hyper evaluation
	 \[
	  O\eval \vec{v'}\hmid t[\vec g/\G] \eval v\hmid u[\vec
	 d/\D]\eval w
	 \]
	 is derivable.
	 Thus,
	 \[
	  O\eval \vec{v'}\hmid (ct)[\vec g/\G] \eval w\hmid (\co cu)[\vec
	 d/\D]\eval v
	 \]
	 is also derivable.
    \item[(Contraction)]
	 The derivation ends as
	  \begin{center}
	   \AxiomC{$\hypert\hmid\G,\tj{x}{!\phi},\tj{y}{!\phi}\tr\tj{t}{\psi}$}
	   \UnaryInfC{$\hypert\hmid \G,\tj{z}{!\phi}\tr\tj{\letin z
	   {x@y} t}{\psi}$}
	   \DisplayProof
	  \end{center}
	 Take any $O\in\semo{\hypert}$, $\vec{g}\in\semo{\G}$ and
	 $u\in\semo{!\phi}$.
	 Since $u$ is in $\semo{!\phi}$, there exists a term $u'$ in
	 $\semo{\phi}$
	 where $u\eval !u'$.
	 Moreover, by induction hypothesis,
	 \[
	  \hypert\eval\vec{o}\hmid t[\vec g/\G][!u'/x][!u'/y]\eval t'
	 \]
	 is derivable for some $\vec{o}$ and $t'\in\semo{\psi}$.
	 By merge,
	 \[
	  \hypert\eval\vec{o}\hmid t[\vec g/\G][!u'/x][!u'/y]\eval
	 t'\hmid u\eval !u'
	 \]
	 is derivable and then
	 \[
	  \hypert\eval\vec{o}\hmid
	 (\letin{z}{x@y}{t})[\vec g/\G][!u'/z]\eval t'
	 \]
	 is also derivable.
    \item[($\forall$R)]
	 Since the induction hypothesis holds regardless of $|X|_0$.
    \item[Other rules]
	 Other cases are easier.
   \end{description}
   \fix{complete}
  \end{proof}

   \begin{corollary}[Consistency of \fix{name of logic}]
    $\tr\tj{t}{\forall X.X}$ is not derivable for any term~$t$.
   \end{corollary}
    \begin{proof}
     Assume $\tr\tj{t}{\forall X.X}$ is derivable.
     Then, by \thref{thm:generalconvergence},
     $t\eval t'$ is derivable for some $t'$ in $\semo{\forall X.X}$.
     However, $\semo{\forall X.X}$ is empty.
    \end{proof}

    \section{Sessions and Processes}
    \subsection{Session Types as Abbreviations}

    As abbreviation, we introduce session types.
    Instead of the standard notation $!\phi.\psi$, we use
    $\sendtype{\phi}{\psi}$ in order to avoid the confusion with
    $!\phi$.
    For symmetry, we use $\recvtype{\phi}\psi$ instead of $?\phi.\psi$.
    The definitions and the descriptions are modification from
    \citet{wadler2012propositions}.
    \begin{align*}
     \sendtype\phi\psi&\equiv \phi\limp\psi &\text{output value of type $\phi$ then behave as $\psi$} \\
     \recvtype\phi\psi&\equiv \phi\otimes\psi &\text{input value of type $\phi$ then behave as $\psi$}\\
     \oplus\{l_i\colon \phi_i\}_{i\in I} &\equiv {\phi_0}\with
     \cdots \with {\phi_n}, \quad I = \{0,\ldots,n\} & \text{select from behaviours
     $\phi_i$ with label $l_i$}\\
     \with\{l_i\colon \phi_i\}_{i\in I} &\equiv {\phi_0}\oplus
     \cdots \oplus {\phi_n}, \quad I = \{0,\ldots,n\}& \text{offer choice of
     behaviours $\phi_i$ with label $l_i$}
     \\
     \terminate &\equiv \one &\text{terminator}
    \end{align*}
    where $I$ is a finite downward-closed set of natural numbers like
    $\{0,1,2,3\}$.


    The grammar
    \[
     \phi,\psi ::= \terminate\mid X \mid \sendtype\phi\psi \mid
     \recvtype\phi\psi
     \mid \oplus\{l_i\colon\phi_i\}_{i\in I}
     \mid \with\{l_i\colon\phi_i\}_{i\in I}
     \mid !\phi
    \]
    covers all types.

    A linear type ($\phi^\ell$ possibly with subscript) is generated by
    this grammar:
    \[
     \phi^\ell ::= \terminate\mid
     \sendtype{\psi}{\phi^\ell} \mid
     \recvtype{\psi}{\phi^\ell}
     \mid \oplus\{l_i\colon\phi^\ell_i\}_{i\in I}
     \mid \with\{l_i\colon\phi^\ell_i\}_{i\in I}
    \]

    We define duals of linear types.
    Again the definition is almost the
    same as \citet{wadler2012propositions}'s except that $\terminate$ is
    self-dual.
    \begin{align*}
     \overline{\sendtype\psi{\phi^\ell}}&= \,\recvtype\psi{\overline{\phi^\ell}}\\
     \overline{\recvtype\psi{\phi^\ell}}&= \,\sendtype\psi{\overline{\phi^\ell}}\\
     \overline{\oplus\{l_i\colon \phi^\ell_i\}_{i\in I}} &=
     \with\{l_i\colon \overline{\phi^\ell_i}\}_{i\in I} \\
     \overline{\with\{l_i\colon \phi^\ell_i\}_{i\in I}} &=
     \oplus\{l_i\colon \overline{\phi^\ell_i}\}_{i\in I} \\
     \overline{\terminate} &= \terminate\enspace.
    \end{align*}
    \fix{see other literature as well for the definition of session
    types}

    \subsection{Processes as Abbreviations}

    We define the sending and receiving as abbreviations:
    \begin{align*}
     \sendterm x u t &\equiv t[(xu) /x] &\text{send $u$ through channel
     $x$ and then use $x$ in $t$} \\
     \recvterm x y t &\equiv \letin x {\tuple{y,x}} t & \text{receive
     $y$ through channel $x$ and use $x$ and $y$ in $t$.}
    \end{align*}

    \subsubsection{Typing Process Abbreviations}
    The session type abbreviation and the processes abbreviation are
    consistent as the next proposition shows.
     \begin{proposition}
      These rules are admissible.\\
      \AxiomC{$\hypert\hmid\tj{y}{\psi}, \tj{x}{\chi}\tr\tj{t}{\phi}$}
      \UnaryInfC{$\hypert\hmid\tj{x}{\recvtype\psi\chi}\tr\tj{\recvterm x y
      t}{\phi}$}
      \DisplayProof
      \hfill
      \AxiomC{$\hypert\hmid \G,\tj{x}{\chi}\tr\tj{t}{\phi}\hmid \D\tr\tj{u}\psi$}
      \UnaryInfC{$\hypert\hmid \G,\D,\tj{x}{\sendtype
      \psi\chi}\tr\tj{\sendterm x u t}{\phi}$}
      \DisplayProof
     \end{proposition}
     Before entering the proof, we note that the types of $x$ change in
     the rules.  This reflects the intuition of session types: the
     session type of a channel changes after some communication occurs
     through the channel.
      \begin{proof}
       Without abbreviations, the first rule is actually one of the
       original rules:
	\begin{center}
	 \AxiomC{$\hypert\hmid\tj{y}{\psi}, \tj{x}{\chi}\tr\tj{t}{\phi}$}
	 \UnaryInfC{$\hypert\hmid\tj{x}{\psi\otimes\chi}\tr\tj{\letin x
	 {y\otimes x} t}{\phi}$}
	 \DisplayProof\enspace.
	\end{center}
       Without abbreviations, the second rule is also one of the
       original rules:
	\begin{center}
	 \AxiomC{$\hypert\hmid \G,\tj{x}{\chi}\tr\tj{t}{\phi}\hmid \D\tr\tj{u}\psi$}
	 \UnaryInfC{$\hypert\hmid \G,\D,\tj{x}{
      \psi\limp\chi}\tr\tj{t[(xu)/x]}{\phi}$}
	 \DisplayProof\enspace.
	\end{center}
      \end{proof}


    \subsection{Implementing Complex Channels}
    We introduced primitives $\lpair{c,\co c}$ implementing
    $(\one\limp\phi)\otimes(\phi\limp\one)$.
    These can be seen as channels of session types
    $\recvtype\phi\terminate$ and $\sendtype\phi\terminate$.
    Indeed, $\recvtype\phi\terminate$ is $\phi\otimes\one$ (which is
    coerceable to $\one\limp\phi$) and $\sendtype\phi\terminate$ is
    $\phi\limp \one$.
    We can generalize this phenomenon to the more complicated session
    types.
     \begin{theorem}[Session Realizers]
      For any linear type~$\phi^\ell$\kern -2pt, the hypersequent
      $\tr\tj{t}{\phi^\ell}\hmid \tr\tj{u}{\overline{\phi^\ell}}$
      is derivable for some terms $t$ and $u$.
     \end{theorem}
      \begin{proof}
       Induction on $S$.
       \begin{description}
	\item[(end)] \AxiomC{} \UnaryInfC{$\tr\tj\ast\one$}
	     \AxiomC{} \UnaryInfC{$\tr\tj\ast\one$}
	     \BinaryInfC{$\tr\tj\ast\one\hmid\tr\tj\ast\one$}
	     \DisplayProof is what we seek.
	\item[($\sendtype{\psi}{\phi^\ell}$)]
	     By induction hypothesis,
	     $\tr\tj{t'}{\phi^\ell}\hmid \tr \tj{u'}{\overline{\phi^\ell}}$ is
	     derivable.  Using this, we can make the following
	     derivation:
	      \begin{center}
	      \AxiomC{}
	       \UnaryInfC{$\tj{x}{\psi}\tr\tj{x}{\psi}$}
	       \AxiomC{$\tr\tj{t'}{\phi^\ell}\hmid \tr
	       \tj{u'}{\overline{\phi^\ell}}$}
	       \BinaryInfC{$\tj{x}{\psi}\tr\tj{x}{\psi}\hmid
	       \tr\tj{t'}{\phi^\ell}\hmid \tr
	       \tj{u'}{\overline{\phi^\ell}}$}
	       \UnaryInfC{$\tj{x}{\psi}\tr\tj{cx}{\phi^\ell}\hmid
	       \tr\tj{(\co ct')\otimes u'}{\psi\otimes \overline{\phi^\ell}}$}
	       \UnaryInfC{$\tr\tj{\lambda x.cx}{\psi\limp\phi^\ell}\hmid
	       \tr\tj{(\co ct')\otimes u'}{\psi\otimes \overline{\phi^\ell}}$}
	       \DisplayProof\enspace.
	      \end{center}
	\item[($\recvtype\psi\phi$)]
	     Symmetric to above.
	\item[($\oplus\{l_i\colon \phi_i\}$)]
	     By induction hypothesis,
	     for each $i\in I$, we have
	     \[
	      \tr\tj{t_i}{{\phi_i}}\hmid \tr\tj{u_i}{{\overline{\phi_i}}}
	     \]
	     derived.  Hence derivable is
	     \[
	      \tr\tj{t_i}{{\phi_i}}\hmid \tr\tj{i(u_i)}{\oplus_{j\in
	     I}
	     {\overline{\phi_j}}}\enspace.
	     \]
	     Combining $|I|$ such derivations, we can derive
	     \[
	     \tr\tj{\tuple{t_i}_{i\in I}^n}{\with_{i\in I}{\phi_i}}
	     \hmid
	     \tuple{\tr\tj{i(u_i)}{\oplus_{j\in
	     I}{\overline{\phi_j}}}}_{i\in I}^n
	     \]
	     for a fresh natural number~$n$.
	     \fix{define and explain about this form of hypersequents}
	\item[($\with\{l_i\colon \phi_i\}$)]
	     Symmetric to above.
       \end{description}
      \end{proof}
      We call the pair $t,u$ the session realizers of $\phi^\ell$ and
      denote them by $\leftside{\phi^\ell}, \rightside{\phi^\ell}$.
      If have two terms that uses free variables of type $\phi^\ell$ and
      $\overline{\phi^\ell}$,
      we can replace those free variables by session realizers.
       \begin{corollary}
	If
	$\hypert_0\hmid \G,\tj{x}{\phi^\ell}\tr\tj{t}{\psi}$ and
	$\hypert_1\hmid \D,\tj{y}{\overline{\phi^\ell}}\tr\tj{u}{\theta}$
	are derivable,
	\[
	\hypert_0\hmid \G\tr\tj{t[\leftside{\phi^\ell}/ x]}{\psi}
	\hmid \hypert_1\hmid \D\tr\tj{u[\rightside{\phi^\ell}/ y]}{\theta}
	\]
	is also derivable.
       \end{corollary}
	\begin{example}
	 Using the session realizers, we can implement the following
	 pair of terms performing higher order communication.
	 \[
	 \tr
	 \tj{
	 \sendterm x {\leftside{\sendtype{\one \oplus \one}\terminate}}
	 {\recvterm {\rightside{\sendtype{\one\oplus\one}\terminate}} z
	 z}}
	 {\one\oplus\one}
	 \hmid
	 \tj{
	 \recvterm x w {\sendterm w {\inl \ast} \ast}}
	 {\one}
	 \]
	 The left term sends a channel and the right term receives the
	 channel.  Then the right term uses the channel to send one bit
	 information to the left term.

	 Without abbreviations, the above derivation is actually
	 \fix{fill}.
	 We can derive their canonical forms as follows.
	 \fix{fill}.
	\end{example}


      \subsubsection{Evaluation of Processes}

      The intention of defining $\sendterm x u {t_0}$ and $\recvterm y z {t_1}$
      is mimicking communication in process calculi.
      When we substitute $x$ and $y$ with session type realizers,
      these terms actually make communication.
       \begin{theorem}
	Let $t_0$ be a term containing $x$ and $t_1$ a term containing
	$y$ and $z$ as free variables.
	If this evaluation hypersequent
	\[
	\letin {\leftside{\phi^\ell}\otimes \rightside{\phi^\ell}}
	{x\otimes y}{(t_0\otimes t_1[u'/z])}\eval v\otimes w\hmid u\eval u'
	\]
	is derivable, so is
	\[
	 \letin{\leftside{\sendtype{\psi}{\phi^\ell}}\otimes
	\rightside{\sendtype{\psi}{\phi^\ell}}}{x\otimes y}{
	\left((\sendterm x u {t_0})\otimes (\recvterm y z {t_1})\right)}
	\eval v\otimes w
	\enspace .
	\]
       \end{theorem}
	\begin{proof}
	 The assumption is only achievable through a derivation ending
	 in:
	  \begin{center}
	   \AxiomC{$\vdots$}
	   \UnaryInfC{$\leftside{\phi^\ell}\eval L\hmid
	   \rightside{\phi^\ell}\eval R\hmid t_0[L/x]\eval v\hmid
	   t_1[u'/z][R/y]\eval w\hmid u\eval u'$}
	   \doubleLine
	   \UnaryInfC{$\leftside{\phi^\ell}\otimes\rightside{\phi^\ell}\eval
	   L\otimes R\hmid (t_0\otimes t_1[u'/z])[L/x][R/y]\eval
	   v\otimes w\hmid u\eval u'$}
	   \UnaryInfC{
	   $\letin{\leftside{\phi^\ell\otimes
	   \rightside{\phi^\ell}}}{x\otimes y}{t_0\otimes
	   t_1[u'/z]}\eval v\otimes w\hmid u\eval u'$
	   }
	   \DisplayProof\enspace .
	  \end{center}
	 We can restart from the topmost shown step.
	  \begin{center}
	   \AxiomC{}
	   \UnaryInfC{
	   \footnotesize
	   $\lambda x.cx\eval \lambda x.cx$}
	   \AxiomC{\footnotesize
	   $\leftside{\phi^\ell}\eval L\hmid
	   \rightside{\phi^\ell}\eval R\hmid t_0[L/x]\eval v\hmid
	   t_1[u'/z][R/y]\eval w\hmid u\eval u'$}
	   \UnaryInfC{\footnotesize
	   $cu\eval L\hmid\co c(\leftside{\phi^\ell})\eval u'\hmid
	   \rightside{\phi^\ell}\eval R\hmid t_0[L/x]\eval v\hmid
	   t_1[u'/z][R/y]\eval w$}
	   \BinaryInfC{$(\lambda x.cx)u\eval L\hmid\co
	   c(\leftside{\phi^\ell})\eval u'\hmid
	   \rightside{\phi^\ell}\eval R\hmid t_0[L/x]\eval v\hmid
	   t_1[u'/z][R/y]\eval w$}
	   \UnaryInfC{$
	   t_0[(\lambda x.cx)u/x]\eval v
	   \hmid\co
	   c(\leftside{\phi^\ell})\eval u'\hmid
	   \rightside{\phi^\ell}\eval R\hmid
	   t_1[u'/z][R/y]\eval w$
	   }
	   \UnaryInfC{$
	   t_0[(\lambda x.cx)u/x]\eval v
	   \hmid
	   \co
	   c(\leftside{\phi^\ell}) \otimes \rightside{\phi^\ell}
	   \eval u' \otimes R \hmid
	   t_1[u'/z][R/y]\eval w$
	   }
	   \UnaryInfC{$
	   t_0[(\lambda x.cx)u/x]\eval v
	   \hmid \letin{\co
	   c(\leftside{\phi^\ell})\otimes
	   \rightside{\phi^\ell}}{z\otimes y}{t_1}\eval w$
	   }
	   \UnaryInfC{$
	   t_0[(\lambda x.cx)u/x] \otimes \left(\letin{\co
	   c(\leftside{\phi^\ell})\otimes
	   \rightside{\phi^\ell}}{z\otimes y}{t_1}\right)
	   \eval
	   v \otimes w
	   $}
	   \UnaryInfC{$
	   \letin{(\lambda x.cx)\otimes \co
	   c(\leftside{\phi^\ell})\otimes
	   \rightside{\phi^\ell}}{x\otimes y}{
	   t_0[xu/x] \otimes \left(\letin{y}{z\otimes y}{t_1}\right)}
	   \eval
	   v \otimes w
	   $}
	   \DisplayProof
	  \end{center}
	 The conclusion is identical to our goal up to abbreviations.
	\end{proof}

  \subsection{Copycatting}
  \begin{theorem}
   For any linear type~$\phi^\ell$,
   we can derive
   $\tj{x}{\phi^\ell},\tj{y}{\overline{\phi^\ell}}\tr
   \tj{t}{\one}$
   for some term~$t$.
  \end{theorem}
  \begin{proof}
   By induction on $\phi^\ell$.
   \begin{description}
    \item[($\terminate$)]
	 Since $\overline{\terminate} =\terminate$, the derivation
	  \begin{center}
	   \AxiomC{}
	   \UnaryInfC{$\tr\tj{\ast}\one$}
	   \UnaryInfC{$\tj{y}{\terminate}\tr\tj{\ast}\one$}
	   \UnaryInfC{$\tj{x}{\terminate},\tj{y}{\terminate}\tr\tj{\ast}\one$}
	   \DisplayProof
	  \end{center}
	 suffices.
    \item[($\sendtype\psi{\phi^\ell}$)]
	 Using the induction hypothesis (IH.), we obtain a derivation:
	  \begin{center}
	   \AxiomC{$\vdots$ (IH.)}
	   \UnaryInfC{$\tj{x}{\phi^\ell},\tj{y}{\overline{\phi^\ell}}\tr\tj{t}{\one}$}
	   \AxiomC{}
	   \UnaryInfC{$\tj{z}{\psi}\tr\tj{z}{\psi}$}
	   \BinaryInfC{$\tj{x}{\sendtype{\psi}{\phi^\ell}},\tj{y}{\overline{\phi^\ell}},\tj{z}{\psi}\tr\tj{\sendterm
	   x z t}{\one}$}
	   \UnaryInfC{$\tj{x}{\sendtype{\psi}{\phi^\ell}}\tr\tj{y}{\recvtype
	  \psi {\overline{\phi^\ell}}}\tr\tj{\recvterm y z {\sendterm x
	   z t}}{\one}$}
	   \DisplayProof
	  \end{center}
    \item[($\recvtype\psi{\phi^\ell}$)]
	 Symmetric to above.
    \item[($\oplus\{l_i\colon\phi^\ell_i\}$)]
	 \fix{fill}
    \item[($\with\{l_i\colon\phi^\ell_i\}$)]
	 Symmetric to above.
   \end{description}
  \end{proof}


    \fix{translate the reductions}

% \section{Toward Determinacy}

% Determinancy states that if $t\eval v$ and $t\eval w$ are derivable,
% then, $v$ and $w$ are identical.
% Since the proof of dterminacy goes inductively over evaluation
% derivations\fix{define
% evaluation derivations},
% we have to generalize the statement.
%  \begin{proposition}[General Determinacy]
%   If
%   $t_0\eval v_0\hmid \cdots \hmid t_n\eval v_n$ and
%   $t_0\eval w_0\hmid \cdots \hmid t_n\eval w_n$ are
%   derivable,
%   then each $v_i$ is identical to $w_i$.
%  \end{proposition}
%   \begin{proof}
%    By induction on the length of the first derivation.
%    \fix{complete}
%   \end{proof}

\fix{sending and receiving as macros,  what you can do with $x:!\phi.S?$ }

\section{Proof Nets}

\fix{talk about proof nets}

\section{Related Work}

\subsection{Wadler}

\subsection{Pfenning}

\subsection{Beffara}

\section{Conclusion}


\section{Discussion}

It is tempting to add modalities to types so that the
modalities show agents and then study the relationship with the
multiparty session types \fix{cite}.

\fix{give rationale for giving up determinacy.. channel inside lambda}
