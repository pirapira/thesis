\chapter{The Logic of Information Exchange}
\label{ch:exchange}

\subsection{Summary}

The previous chapters treated the computational interpretations of
disjunctive formulas like $(\phi\imp\psi)\lor(\psi\imp\phi)$ or
$(\phi\limp\psi)\oplus(\psi\limp\phi)$.  In this chapter, we try
replacing these disjunctions into conjunctions.
In the former case, the change renders the logic inconsistent.
With the axiom $(\phi\imp\psi)\land(\psi\imp\phi)$,
we can prove any formula.  However in the lattar case, the change does
not make the system meaningless.  We treat
the axioms of the form $(\phi\limp\psi)\otimes(\psi\limp\phi)$
on top of IMLL2, the second order formulation of intuitionistic
multiplicative linear
logic.  In essence, the axiom allows two processes to wait for one
another and then exchange some pieces of information.

\subsection{Dynamics}

We define terms\index{term}~$t$ by BNF:
\[
 t ::= x\mid \lambda x.t\mid (t)t\mid c \mid (t\conc t)\enspace.
\]
We use $\Lambda$ for the set of terms.
A stack\index{stack}~$\pi$ is defined by BNF:
\[
 \pi ::= \epsilon\mid t\cdot\pi \mid \tuple{\pi,\pi}
\]
We use $\Pi$ for the set of stacks.
An executable is a multiset on $\Lambda\times \Pi$.

The reduction relation $\red$ is the smallest preorder on executables
that satisfies:
\begin{description}
 \item[(cong)] if
      $e_0         \red e_1$
      then
      $e_0 \conc e \red e_1\conc e$\enspace;
 \item[(abort)] $[\abort,\pi]\red \emptyset$ \enspace;
 \item[(push)]
	    $[(t)u,\pi]      \red [t,u\cdot\pi]$      \enspace;
 \item[(store)]
	    $[\lambda x.t,u\cdot\pi]
	     \red
	     [t[\ast_u/x],      \pi]$\enspace;
 \item[(load)]
	    $[\ast_u,\pi]\red [u,\pi]$\enspace;
 \item[(ex)]
           $[c, t\cdot\pi\conc \co c,
           u\cdot\sigma] \red
           [u,\pi\conc t,\sigma]$\enspace;
 \item[(dist)]
           $[(t\conc u),\tuple{\pi,\sigma}]  \red [(t,\pi)\conc (u,\sigma)]$ \enspace.
\end{description}


\subsection{Statics}

We define an IMLL2 formula~$\phi$ by BNF:
\[
 \phi :: =
 X \mid
 \phi\limp\phi\mid
% \phi\otimes\phi\mid
 \forall X\phi\enspace.
\]
As abbreviations, we can introduce $\otimes$ and $\bot$:
\begin{align*}
 \phi\otimes\psi &\equiv \forall X ((\phi\limp\psi\limp X)\limp X)
 \text{ do not contain }\\
 \bot &\equiv \forall X X\enspace.
\end{align*}

The type system here employs conjunctive hypersequents.  Usually,
different components in a hypersequent are interpreted disjunctively.
However, components in a conjunctive hypersequent are interpreted
conjunctively by $\otimes$.

 \begin{figure}
\centering
\AxiomC{}
\LL  {Ax}
\useq{\xphi}{\xphi}
\DisplayProof
%
\hfill
\AxiomC{$\mathcal H$}
\AxiomC{$\mathcal H$}
\LL{mix}
\BinaryInfC{$\mathcal H\hmid \mathcal{H'}$}
\DisplayProof
%
\ruleskip
\aseq{\hyp{}\xphi,\G}{\tj t\psi}
\LL{$\limp$I}
\useq{\hyp{}\G}{\tj{\lambda x.t}{\phi\limp\psi}}
\DisplayProof
\hfill
%
\aseq{\hyp{}\G}{\tj t{\phi\limp\psi}}
\aseq{\hyp'\D}{\tj u\phi}
\LL   {$\limp$E}
\bseq{\hyp{}\hyp'\G,\D}{\tj{(t)u}\psi}
\DisplayProof
\ruleskip
\aseq{\hyp{}\G}{\tj t\phi}
\LL   {$\forall$I}
\useq{\hyp{}\G}{\tj t{\forall X\phi}}
\DisplayProof (no free $X$ in $\mathcal H$ or $\G$)
\hfill
%
\aseq{\hyp{}\G}{\tj{t}{\forall X\phi}}
\LL   {$\forall$E}
\useq{\hyp{}\G}{\tj{t}{\phi[\psi/X]}}
\DisplayProof
\ruleskip
%
\aseq{\hyp{}\G}{\tj{t}{\phi}\hmid \D\tr\tj{u}{\psi}}
\LL  {com}
\useq{\hyp{}\G}{\tj{(c)t}{\psi}\hmid \D\tr\tj{(\co c)u}{\phi}}
\DisplayProof ($c$ is fresh)
\ruleskip
%
% \aseq{\hyp{}\G}{\tj{t}{\phi}\hmid \D\tr\tj{u}{\psi}}
%   \LL{$\otimes$I}
% \useq{\hyp{}\G,\D}{\tj{(t\conc u)}{\phi\otimes \psi}}
% \DisplayProof
%   %
%   \ruleskip
%   \aseq{\hyp{}\G}{\tj{t}{\phi\otimes\psi}\hmid
%   \tj{x}{\phi},\tj{y}{\psi},\D\tr \tj{u}{\theta}}
%   \LL{$\otimes$E}
%   \useq{\hyp{}\hyp'\G,\D}{\tj{\letpar t x y u}{\theta}}
%   \DisplayProof
  \caption{The typing rule of $\NMLL$.}
 \end{figure}

\subsection{Specification Using Poles}

We define programs and environments in the same way as
Subsection~\ref{pole:spec}.

\begin{definition}
 \label{def:pole}
A pole\index{pole}~$\bbot$ is a set of executables
which satisfies
\begin{enumerate}
 \item \label{red-closed} $e$ is in $\bbot$ if $e\red e'$ and
       $e'\in\bbot$; and
 \item \label{conc-closed} $e\conc f$ is in $\bbot$ if $e$ and $f$ are in $\bbot$.
\end{enumerate}
\end{definition}

\fix{define $\sempair{\phi^+,\psi^-}$}

 \begin{theorem}[Adequacy]
  \fix{state}
 \end{theorem}
 \fix{prove}

  \begin{proposition}[Specification of the axiom]
   Let $g$ have type ${\forall X\forall
   Y((X\limp Y)\times (Y\limp X))}$.
   Then, the s-executable
   \[
   [g, \mats{z}{z}{x\cdot\pi_Y}{w}{w}{y\cdot\pi_X}]
   \]
   reduces to a multiset on
   $\{(x,\pi_X), (y,\pi_Y)\}$.
  \end{proposition}

\subsection{Consistency}

\begin{proposition}
  No term has the type $\forall X X$.
\end{proposition}
\begin{corollary}
 The system $\NMLL$ is consistent.
\end{corollary}
