\chapter{The Logic of Information Exchange}
\label{ch:exchange}

\subsection{Summary}

The previous chapters treated the computational interpretations of
disjunctive formulae like $(\phi\imp\psi)\lor(\psi\imp\phi)$ or
$(\phi\limp\psi)\oplus(\psi\limp\phi)$.  In this chapter, we try
replacing these disjunctions with conjunctions.
In the former case, the change renders the logic inconsistent.
If we add the axiom $(\phi\imp\psi)\land(\psi\imp\phi)$ to the
intuitionistic propositional logic,
we can prove any formula.  However in the lattar case, the change does
not make the system meaningless.
In this chapter, we treat
the axioms of the form $(\phi\limp\psi)\otimes(\psi\limp\phi)$
on top of IMLL2, the second order formulation of intuitionistic
multiplicative linear
logic.  In essence, the axiom allows two processes to wait for one
another and then exchange information.

\subsection{Dynamics}

We also assume countably infinitely many channels with involution
satisfying $\co c\neq c$ and $\co{\co c} = c$.
We define terms\index{term}~$t$ by BNF:
\[
 t ::= x\mid \lambda x.t\mid (t)t\mid c
\]
where $c$ is a channel.
As an abbreviation, $\lambda x_0x_1\cdots x_n.t$
denotes $\lambda x_0. \lambda x_1. \ldots . \lambda x_n.t$.
We use $\Lambda$ for the set of terms.
A stack\index{stack}~$\pi$ and a process\index{process}~$p$
are defined by BNF:
\begin{align*}
 \pi &::= \epsilon\mid p\cdot\pi\\
 p   &::= (t,\pi)\enspace.
\end{align*}
For a process $p$, we define a term~$\bar p$ inductively on
the construction of $p$:
\begin{align*}
 \overline{(t,p_0\cdot\cdots \cdot p_n)}&= (((t)\overline{p_0})\cdots)\overline{p_n}\enspace.
\end{align*}
For two stacks $\sigma$ and $\pi$, the concatenation $\sigma\cdot\pi$ is
defined
inductively on $\sigma$: i.e. $\epsilon\cdot\pi = \pi$ and $(e
\cdot \sigma') \cdot \pi = e\cdot (\sigma'\cdot \pi)$.  By the latter
equation, $\cdot$ can be considered associative.
We use $\Pi$ for the set of stacks.
Also, the concatenation of a process $(t,\pi)$ and a stack~$\sigma$ is
defined as $(t,\pi)\circ\sigma = (t,\pi\cdot \sigma)$.

The reduction relations are the smallest preorders on processes
 and on stacks (both denoted by $\red$) that satisfies:
\begin{description}
 \item[(cong)] $p\red p'$ implies $\sigma \cdot p \cdot \pi\red
      \sigma\cdot p'\cdot \pi$ \enspace;
 \item[(cong')] $\pi\red\pi'$ implies $t,\pi\red t,\pi'$\enspace;
 \item[(push)]
	    $(t)u, \pi   \red t,(u,\epsilon)\cdot\pi$\enspace;
 \item[(pull)]
	    $\lambda x.t,p\cdot\pi
	     \red
	     t[\bar p/x], \pi$\enspace;
 \item[(ex)]
           $\pi\cdot (c, (t,\pi_0)\cdot \pi_1)\cdot\pi' \cdot (\co
      c,(u,\sigma_0)\cdot \sigma_1) \cdot \pi''\red
      \pi\cdot (u, \sigma_0\cdot \pi_1)\cdot\pi' \cdot (
      t,\pi_0\cdot \sigma_1) \cdot \pi''
      $
      \enspace.
\end{description}


\subsection{Statics}

For a set~$S$,
$\form(S)$ is the set of $S$-formulas~$\phi$:
\[
\phi::= s \mid X \mid \phi\limp\phi\mid
\forall X \phi
\]
where $s\in S$, $X\in \pvar$. The $X$ in the
last clause is binding.
An IMLL2 formula is an element of $\form(\emptyset)$.

As abbreviations, we can introduce $\otimes$ and $\bot$:
\begin{align*}
 \phi\otimes\psi &\equiv \forall X ((\phi\limp\psi\limp X)\limp X)
 \text{ where }\phi\otimes\psi\text{ does not contain } X \text{ freely;}\\
 \bot &\equiv \forall X X\enspace.
\end{align*}

The type system employs conjunctive hypersequents.  In the previous
chapters,
different components in a hypersequent are interpreted disjunctively.
However, components in a conjunctive hypersequent are interpreted
conjunctively by $\otimes$.

 \begin{figure}
\centering
\AxiomC{}
\LL  {Ax}
\useq{\xphi}{\xphi}
\DisplayProof
%
\hfill
\AxiomC{$\mathcal H$}
\AxiomC{$\mathcal H$}
\LL{mix}
\BinaryInfC{$\mathcal H\hmid \mathcal{H'}$}
\DisplayProof
%
\newcommand{\tjx}[1]{\tj{x_{#1}}{\phi_{#1}}}
\ruleskip
  \aseq{\hyp{}\tjx{_0},\ldots,\tjx{_n},\G}{\tj{t}{\psi}}
  \LL{$\limp$I}
  \useq{\hyp{}\G}{\tj{\lambda x_0x_1\cdots
  x_n.t}{\phi_0\limp\phi_1\limp\cdots\phi_n\limp\psi}}
  \DisplayProof\\
  where $\psi$ does not have $\limp$ as its top connective.
  \ruleskip
%
\aseq{\hyp{}\G}{\tj t{\phi_0\limp\cdots\limp \phi_n \limp\psi}}
\aseq{\D_0}{\tj{u_0}{\phi_0} \hmid \cdots \hmid \D_n\tr\tj{u_n}{\phi_n}}
\LL   {$\limp$E}
\bseq{\hyp{}\G,\D_0,\ldots,\D_n}{\tj{(((t)u_0)\cdots) u_n}\psi}
\DisplayProof
\ruleskip
\aseq{\hyp{}\G}{\tj t\phi}
\LL   {$\forall$I}
\useq{\hyp{}\G}{\tj t{\forall X\phi}}
\DisplayProof (no free $X$ in $\mathcal H$ or $\G$)
\hfill
%
\aseq{\hyp{}\G}{\tj{t}{\forall X\phi}}
\LL   {$\forall$E}
\useq{\hyp{}\G}{\tj{t}{\phi[\psi/X]}}
\DisplayProof
\ruleskip
%
\aseq{\hyp{}\G}{\tj{t}{\phi}\hmid \D\tr\tj{u}{\psi}}
\LL  {com}
\useq{\hyp{}\G}{\tj{(c)t}{\psi}\hmid \D\tr\tj{(\co c)u}{\phi}}
\DisplayProof ($c$ is fresh)
\ruleskip
%
% \aseq{\hyp{}\G}{\tj{t}{\phi}\hmid \D\tr\tj{u}{\psi}}
%   \LL{$\otimes$I}
% \useq{\hyp{}\G,\D}{\tj{(t\conc u)}{\phi\otimes \psi}}
% \DisplayProof
%   %
%   \ruleskip
%   \aseq{\hyp{}\G}{\tj{t}{\phi\otimes\psi}\hmid
%   \tj{x}{\phi},\tj{y}{\psi},\D\tr \tj{u}{\theta}}
%   \LL{$\otimes$E}
%   \useq{\hyp{}\hyp'\G,\D}{\tj{\letpar t x y u}{\theta}}
%   \DisplayProof
  \caption{The typing rules of $\NMLL$.  Exchange rules are omitted.}
 \end{figure}

\subsection{Specification Using Poles}

\begin{definition}
 \label{ex:def:pole}
A pole\index{pole}~$\bbot$ is a set of stacks
which satisfies
\begin{enumerate}
 \item \label{ex:red-closed} $\pi$ is in $\bbot$ if $\pi\red \sigma$ and
       $\sigma\in\bbot$;
 \item \label{ex:conc-closed} $\pi\cdot \sigma$ is in $\bbot$ if $\pi$
       and $\sigma$ are in $\bbot$; and
 \item \label{ex-closed}
       $\pi\cdot p\cdot p'\cdot \sigma$ is in $\bbot$ if
       $\pi\cdot p'\cdot p\cdot \sigma$ is in $\bbot$.
\end{enumerate}
\end{definition}

For a set~$\mathcal Z$ of
stacks, $\mathcal Z\rightarrow\bbot$ denotes
the set of terms $t$ such that for any
stack $\pi\in\mathcal Z$,
the single-process stack $(t,\pi)\cdot\epsilon$ is in $\bbot$.

For sets of stacks $\mathcal Z_0,\ldots,\mathcal Z_n$,
we define $(\mathcal Z_0,\ldots,\mathcal Z_n)\rightarrow\bbot$ to be the
set of sequences of the form $(p_0,\ldots,p_n)$ such that
for any $\pi_i\in\mathcal Z_i$, the stack $((p_0\circ \pi_0)\cdot\cdots\cdot
(p_n\circ \pi_n))$ is in $\bbot$.


For $\phi\in\form(2^E)$ and $|\cdot|^-_0\colon\pvar\rightarrow 2^E$\kern
-2pt,
we define $\nsem{\phi}\in
2^E$ inductively:
\begin{align*}
 \nsem{\mathcal Z} =& \mathcal Z \text{ for } \mathcal Z\in 2^E\\
 \nsem{X}=& |X|_0^- \\
 %%%
 \nsem{\phi_0\limp\cdots\limp\phi_n\limp\psi}=&
 \{p_0\cdot p_1\cdot\cdots\cdot p_n\cdot\pi\mid \\ &\,\,\,
 p_0\cdot p_1\cdot\cdots\cdot p_n
 \in(\nsem{\phi_0}, \nsem{\phi_1}, \ldots, \nsem{\phi_n})\rightarrow\bbot
 \text{ and  }
 \\&\,\,\,\pi\in\nsem\psi\}\text{ where }\psi\text{ is not of the form }\psi_0\limp\psi_1\\
 %%%
 \nsem{\forall X\phi}=&
 \bigcup_{\mathcal Z\in 2^\sPi} \nsem{\phi[\mathcal Z/X]}\enspace.
\end{align*}
Using this, we define $\sem \phi=\nsem{\phi}\rightarrow\bbot$.
We define $\sem{\phi_0,\ldots,\phi_n}$ to be
$(\nsem{\phi_0},\ldots,\nsem{\phi_n})\rightarrow\bbot$.
 \begin{proposition}
  \label{nsem-tuple}
  When $\pi$ is in
  $\left(\mathcal{Z}_0,\ldots,\mathcal{Z}_m\right)\rightarrow \bbot$
  and $\sigma$ is in
  $\left(\mathcal{Z}_{m+1},\ldots,\mathcal{Z}_n\right)\rightarrow \bbot$,
  then,
  the concatenation $\pi\cdot\sigma$ is in
  $\left(\mathcal{Z}_0,\ldots,\mathcal{Z}_m,
  \mathcal{Z}_{m+1},\ldots,\mathcal{Z}_n\right)\rightarrow\bbot$.
 \end{proposition}
  \begin{proof}
   By condition~\ref{ex:conc-closed} of Definition~\ref{ex:def:pole}.
  \end{proof}

 \begin{proposition}
  \label{nsem-imp}
  If $\sigma$ is in
  $\nsem{\phi_0\limp\phi_1\limp\cdots\limp\phi_m\limp\psi}$, then
  $\sigma\cdot\pi$ is in
  $\nsem{\phi_0\limp\phi_1\limp\cdots\limp\phi_m\limp\psi}$.
 \end{proposition}
  \begin{proof}
   By asssumption, $\sigma = e_0\cdot\cdots\cdot e_m\cdot \pi$
   where $e_0\cdot\cdots\cdot e_m$ is in
   $\sem{\phi_0,\phi_1,\ldots,\phi_m}$ and $\pi$ is in $\nsem{\psi}$.
   Let $\psi$ be $\phi_{m+1}\limp\cdots\phi_n\limp\psi'$ where $\psi'$
   is not of the form $\psi'_0\limp\psi'_1$.
   Since $\pi$ is in $\nsem{\psi}$,
   $\pi$ is of the form
   \[
    e_{m+1}\cdot e_{m+2}\cdot \cdots \cdot e_n\cdot \pi'
   \]
   where $e_{m+1}\cdot e_{m+2}\cdot \cdots e_n$ is in
   $\sem{{\phi_{m+1}}, \ldots, {\phi_n}}$
   and $\pi'$ is in $\nsem{\psi'}$.
   By Proposition~\ref{nsem-tuple},
   $ e_0\cdot\cdots\cdot e_m\cdot e_{m+1}\cdot e_{m+2}\cdot \cdots \cdot e_n $ is in
   $\sem{{\phi_0},\ldots,{\phi_m},{\phi_{m+1}}, \ldots,
   {\phi_n}}$.  The statement follows.
  \end{proof}

For $\G = \tj{x_1}{\phi_1},\ldots,\tj{x_n}{\phi_n}$,
we denote by $\sem{\G}$ the set of sequences $(p_1,\dots,p_n)$
 in $\sem{\phi_1,\ldots,\phi_n}$.
For that sequence~$\vec p$, $[\vec{p}/\G]$ denotes the simultaneous substitution
$[\bar{p_i}/x_i]_{0\le i \le n}$.
For a hypersequent, we
define a set of s-executables
$
\semo{\G_0\tr\tj{t_0}{\phi_0}\hmid\cdots\hmid\G_n\tr\tj{t_n}{\phi_n}}
$
to be the set of executables of the form
$
\left(t_0[\vec{g_0}/\G_0],\pi_0\right)\cdot\cdots\cdot
\left(t_n[\vec{g_n}/\G_n],\pi_n\right)
$
 where
$\vec{g_i} \in \sem{\G_i}$ and $\pi_i\in \nsem{\phi_i}$.

 \begin{theorem}[Adequacy]
  If a hypersequent~$\mathcal H$ is derivable, the stacks in $\semo{\mathcal
  H}$ are in $\bbot$.
 \end{theorem}
  \begin{proof}
   By induction on the derivation of $\mathcal{H}$.
   \begin{description}
    \item[($\limp$I)]
	 Take any $e_0\cdot e_1\cdot \cdots \cdot e_n \cdot \pi \in
	 \nsem{\phi_0\limp\phi_1\limp\cdots \limp\phi_n\limp\psi}$,
	 $\vec g\in\sem{\G}$ and $h\in\semo{\mathcal H}$ so that
	 $e_0\cdot\cdots\cdot e_n$ is in $\sem{\phi_0,\ldots,\phi_n}$
	 and $\pi$ is in $\nsem{\psi}$.
	 We have to show that this stack is in $\bbot$:
	 \[
	  \sigma = h\cdot \left(\lambda x_0x_1\cdots x_n.t[\vec g/\G],
	 e_0\cdot \cdots e_n\cdot \pi\right)\enspace.
	 \]
	 By (pull), $\sigma$ reduces to
	 \[
	  \sigma' = h\cdot \left(t[\vec g/\G][\bar{e_i}/x_i], \pi\right)\enspace.
	 \]
	 We want to use the induction hypothesis to say $\sigma'$ is in
	 $\bbot$.
	 That means $\sigma$ is also in $\bbot$ because $\bbot$ is
	 closed for $\rev\red$.
	 In order to use the induction hypothesis, we need
	 $(\vec g,e_0,\ldots,e_n)\in\sem{\G,\phi_0,\ldots,\phi_n}$,
	 which follows from Proposition~\ref{nsem-tuple}.
    \item[($\limp$E)]
	 Take any $\pi\in\nsem{\psi}$, $\vec g\in\sem{\G}$,
	 $\vec{d_i}\in\sem{\D_i}$ for each $0\le i\le n$ and
	 $h\in\semo{\mathcal H}$.
	 We have to show that this stack is in $\bbot$:
	 \[
	  \sigma = h\cdot (((t[\vec g/\G]u'_0)\cdots) u'_n, \pi)\enspace.
	 \]
	 By (push), $\sigma$ reduces to
	 \[
	 \sigma' = h\cdot \left(t[\vec g/\G],
	 (u'_0,\epsilon)\cdot\cdots\cdot (u'_n,\epsilon)\cdot \pi\right)\enspace.
	 \]
	 By induction hypothesis on the derivation of $u_i$'s,
	 $ (u'_0,\epsilon)\cdot\cdots\cdot (u'_n,\epsilon) $ is in
	 $\sem{\phi_0,\ldots,\phi_n}$.
	 So, by Proposition~\ref{nsem-imp},
	 $(u'_0,\epsilon)\cdot\cdots\cdot (u'_n,\epsilon)\cdot \pi$ is
	 in $\nsem{\phi_0\limp\cdots\limp\phi_n\limp\psi}$.
	 By induction hypothesis on the derivation of $t$,
	 $\sigma'$ is in $\bbot$.
	 Since $\bbot$ is closed for $\rev\red$, $\sigma$ is also in
	 $\bbot$.
    \item[(com)]
	 Take any $\pi_\phi\in\nsem{\phi}$, $\pi_\psi\in\nsem\psi$,
	 $\vec g\in\sem{\G}$, $\vec d\in\sem\D$ and $h\in\semo{\mathcal
	 H}$.
	 We have to show that this stack is in $\bbot$:
	 \[
	  \sigma = h\cdot
	 ((c)t[\vec g/\G],\pi_\psi)\cdot ((\co c)u[\vec d/\D], \pi_\phi)\enspace.
	 \]
	 By (push), (cong) and (ex), $\sigma$ reduces to
	 \[
	  \sigma' = h\cdot u[\vec d/\D], \pi_\psi\cdot t[\vec g/\D],\pi_\phi\enspace.
	 \]
	 By induction hypothesis,
	 \[
	  \sigma'' = h\cdot (t[\vec g/\G],\pi_\phi)\cdot (u[\vec
	 d/\D],\pi_\psi)
	 \]
	 is in $\bbot$.
	 By condition~\ref{ex-closed} of definition of poles, $\sigma'$ is
	 also in $\bbot$.
	 Since $\bbot$ is closed for $\rev\red$, $\sigma$ is in $\bbot$,
	 too.
    \item[$\forall$I]
    \item[$\forall$E]
   \end{description}
  \end{proof}
  A stack on a set~$S$ of processes is a finite sequence of elements of
  $S$.
   \begin{proposition}
    For a set~$S$ of processes,
    it is a pole: the set of stacks that reduces to a stack on $S$.
   \end{proposition}
    \begin{proof}
     \fix{prove}
    \end{proof}

  \begin{proposition}[Specification of the axiom]
   Let $g$ have type ${\forall X\forall
   Y((X\limp Y)\otimes (Y\limp X))}$.
   Then, the process
   \[
   \fix{fill}
   \]
   reduces to a stack on
   $\{(x,\pi_X), (y,\pi_Y)\}$.
  \end{proposition}

\subsection{Consistency}

\begin{proposition}
  No term has the type $\forall X X$.
\end{proposition}
\begin{corollary}
 The system $\NMLL$ is consistent.
\end{corollary}
