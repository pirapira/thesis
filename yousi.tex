\documentclass{jsarticle}

\begin{document}
$BO@J8$NFbMF$NMW;](B

Hyper-Lambda Calculi\\
($B%O%$%Q!<%i%`%@7W;;(B)

$B;aL>(B $BJ?0f!!MN0l(B

 We propose hyper-lambda calculi.  A hyper-lambda term is a finite
 sequence of lambda terms, representing concurrent processes.  We give
 two concrete hyper-lambda calculi: synchronous and asynchronous.  Both
 employ a
 pair of communication primitives exchanging their inputs.
 In the synchronous case, both sides succeed.  In the asynchronous case,
 at least one side obtains the other side's input.
 The synchronous calculus implements message-passing communication
 and session types;
 the asynchronous calculus characterizes shared-memory waitfree
 communication.
 Among processes of a typed hyper-lambda term,
 all succeed in the synchronous case while
 at least one succeeds in the asynchronous case.
 Logically, the processes are interpreted conjunctively
 in the synchronous case but disjunctively in the asynchronous case.
 The synchronous calculus is based on Abelian logic:
 $(\phi\limp\psi)\otimes(\psi\limp\phi)$ on top of maltiplicative
 additive fragment of intuitionistic linear
 logic (without some units);
 the asynchronous calculus is based on G\"odel--Dummett logic:
 $(\phi\imp\psi)\lor(\psi\imp\phi)$ on top of intuitionistic logic.
 The hyper-lambda calculi are in Curry--Howard correspondence with the
 deduction systems for these logics.
\end{document}