\begin{acknowledge}
 The author thank my adviser Masami Hagiya for his patience and some decisive
 comments at important moments.
 Even before composing this thesis, the author became indebted to many of the
 juries and their colleagues.
 In an MLG (mathematical logic group) meeting, Hiroakira Ono asked the author
 about the constructive content of Dummett axiom, which the author answer in
 Chapter~\ref{ch:lambda}.
 Naoki Kobayashi's request for a stronger communication primitive
 resulted in Chapter~\ref{ch:exchange}.
 Ichiro Hasuo gave the author some chances of talks and
 indispensable advice on visiting the Netherlands
 and studying there.
 As well as by other juries,
 The author await valuable comments by Shinichi Honiden and Andrzej Murawski from
 practical and theoretical viewpoints.

 The work about hyper-lambda calculus is encouraged by feedbacks from
 ACAN (Algebraic and Coalgebraic Approaches to
 Non-Classical Logics Workshop) and OPLSS'11 participants,
 Pisa Proof Theory Workshop, FLOPS-2012
 and numerous other occasions.

 Grant-in-Aid for JSPS Fellows 23-6978 supported
 my second and third years of PhD, of which almost one year was spent
 in ILPS, the University of Amsterdam.
 The author thanks Maarten Marx, Maarten de Rijke and other members of ILPS for
 comfortable environments and stimulating discussions during my stay.
 Especially, it was a unique training to work on heavy problems for months with Maarten Marx, Alessandro
 Facchini and Evgeny Sherkhonov.

 The author thanks Tadeusz Litak for encouragements and
 information on relevant research.
 At the University of Tokyo,
 discussions with Tatsuya Abe, Yoshihiko Kakutani and Masahiro Hamano
 nurtured my taste on formalistic, semantic and graphical approaches.
\end{acknowledge}
