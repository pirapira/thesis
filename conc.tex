\chapter{Conclusions}

\section{Overview}

We gave mainly two inventions: hyper-lambda calculi and a new
axiomatization of Abelian logic.
Since we are just beginning to understand the computational semantics of
some concrete examples of substructural logics, the remaining space is
vast.
Reflecting the wide applications of substructural and especially
intermediate logics,
our conclusions and variety of future work span from mathematical logic, computer
science and philosophy.

\section{From Logical Perspectives}

\subsection{Generalization}

We have seen two particular hyper-lambda calculi for two logics.
\citet{alg} classified axioms according to their syntactic complexities
and identified classes of axioms that can be translated into structural
sequent calculus rules and structural hypersequent calculus rules.
According to their classification, the prelinearity axiom
$(\phi\limp\psi)\oplus(\psi\limp\phi)$ and the Amida axiom
$(\phi\limp\psi)\otimes(\psi\limp\phi)$ belong to $\mathcal P_2$, a
class whose elements can be translated into a finite set of
hypersequent structural rules.
Other classes $\mathcal N_2$ and $\mathcal P_3$ of axioms can also be
translated into finite sets of hypersequent structural rules,
thus, we expect the technique of hyper-lambda calculi applicable to
logics with these axioms on top of $\mathbf{FLe}$.

One particular logic worth trying is
the logic characterized by Kripke models with bounded width~\citep{Ciabattoni01042001}.
Since G\"odel-Dummett logic is a special case of the bounding width~1,
the generalization of width $k$ will provide waitfree computation on
weaker shared memory consistency.  Further,
an ambitious goal is to develop a general framework with which we can
develop hyper-lambda calculi for all logics characterized by axioms in
class $\mathcal P_3$.
Another target is the logic with the weak excluded-middle $\neg \phi\lor
\neg\neg \phi$ on top of intuitionistic logic.
Since the cut-elimination proof in \citep{alg} is algebraic,
we are yet to know the computational meaning of the cut-elimination.
Aforementioned framework would clarify the computational meaning of the
cut-elimination of hypersequent calculi.

\subsection{The Amida Calculus}

As we saw in Subsection~\ref{amida-cut},
the deduction system underlying the Amida calculus does not enjoy
cut-elimination in the strict sense.
Moreover, since the Amida calculus characterizes Abelian logic
(\thref{complete-to-Abelian}),
there are some inhabited types that are hard to justify constructively;
for example a linear version of the excluded middle $\phi\oplus(\phi\limp\one)$.
The first step would be removing additives and studying the
multiplicative fragment of the Amida calculus.

\subsection{G\"odel-Dummett Logic}

In logical point of view, the lambda calculus in Chapter~\ref{ch:lambda}
is too complicated for G\"odel-Dummett logic.
For example, the notion of processes are actually not needed to study
G\"odel-Dummett logic.  However, with the help of these additional
constructions,
we were able to identify the computational nature of G\"odel-Dummett
logic as that of waitfree computation.
In order to exploit our discovery, we have to extend the
characterization to the realm of proof searching.  That is, designing a
logic programming language using waitfree communication during parallel
proof searching.

\subsection{More General Fuzzy Logics}

Since we have identified the computational content of the prelinearity
axiom and obtained a lambda calculus for monoidal t-norm logic, we are
in a good position to study fuzzy logics in general.
After all, monoidal t-norm logic is the weakest fuzzy logic.
Thus, there is a possibility to obtain a lambda calculus for any fuzzy
logic by extending the lambda calculus in Chapter~\ref{ch:pole}.
The next target is \L{}ukasiewicz logic.  \L{}ukasiewicz logic is so
similar to ordinary logics that there have been attempts to formalize
mathematics on
it~\citep{Hajek:TheJournalOfSymbolicLogic:2000,hajek2005,yatabe2009}.
\citet[Theorem~9]{metcalfe2002} showed that a fragment of Abelian logic
coincides with \L{}ukasiewicz logic.

\section{From Computer Science Perspectives}

\subsection{Implementation}

In Chapter~\ref{ch:haskell}, we implemented a hyper-lambda calculus for
waitfree computation on top of Haskell~\citep{marlow2010haskell}.
The implementation is a prototype for showing that the design of \lgd\,is
implementable.
Internally, the implementation still uses locks but we should remove
locks if we try to put the library to performance sensitive uses.
In order to remove locks, we probably have to modify the runtime system
of Haskell.
Externally, the implementation lacks primitive support for more than two
party communication.  Although we can implement arbitrary $n$-party
waitfree communication using the current two-party primitive, having
a primitive serving more parties is desirable.
In order to provide more than two-party primitives,
we have to define a more complicated type class, which we expect to be
achievable.

\subsection{Further Implementation}

Since Abelian logic is incompatible with contraction or weakening,
straightforward implementation the Amida calculus on top of Haskell or OCaml
would not be a
good way to exploit the safety of the Amida calculus.
Although Clean~\citep{parle1991} offers uniqueness types,
uniqueness types only reject contraction but accept weakening, so Clean
is not suitable either.

One promising framework on which to implement the Amida calculus is linear ML%
\footnote{Although there are no
publications available, there is an implementation at
\texttt{https://github.com/pikatchu/LinearML}\enspace.},
whose type system is based on linear logic.
Another way is using the type level programming technique of Haskell.
\citet{DBLP:journals/corr/abs-1110-4163} implemented session types on top
of Haskell using the type level programming technique using
\texttt{Session} monad.  Since Haskell types can contain arbitrary trees
of symbols, they were able to encode session type information in Haskell
types.
Logically, this corresponds to having atomic formulae with complicated
structure so that we can encode session information in atomic formulae.
The advantage would be usability of existing Haskell infrastructure
including the optimizing compilers, execution environments, and
libraries.
The disadvantage, which the first author of
\citep{DBLP:journals/corr/abs-1110-4163} told me personally,
is the compiler's cryptic error messages.
Since the used Haskell compiler is not aware of session types and just reports
pattern-matching errors in the encoding of session types, the error
messages are about the encoding but not about the originally intended
session types.

% move to related work part
% Interestingly, their implementation uses heterogeneous
% lists~\citep{hetero}, which is similar to hyper-lambda terms because both
% heterogeneous lists and hyper-lambda terms are lists of programs of
% possibly different types.

\subsection{Reasoning about Hyper-Lambda Terms}

The Amida edges introduced in Section~\ref{sec:proofnets} can be applied
to other methods than proof nets.
For example, in the game semantics of lambda calculi~\citep{untyped},
strategies are represented by Nakajima trees~\citep{nakajima1975}.
Then we can expect the Amida edges on Nakajima trees to represent a
strategy of the game semantics for Abelian logic.
Also, the Amida edges works on paths, it should be compatible to the
path-based semantics by \citet{danos-regnier}.
Since the Amida edges define permutation on the ends of edges in a proof
net, and hence can be represented as a matrix, it should be compatible
to Geometry of Interaction~\citep{girard1989}.

\subsection{Mixing Synchronous and Asynchronous Communication}

Our hyper-lambda calculus in Chapter~\ref{ch:exchange} incorporates
synchronous communication and another hyper-lambda calculus in
Chapter~\ref{ch:lambda} incorporates asynchronous communication.
In order to unify these two paradigm, the most straightforward way is to
nest hypersequents
\[
 t_0\hmid (t_1 \whitemid t_2) \hmid t_3 \enspace.
\]
where $t_1$ and $t_2$ are combined conjunctively but other delimiters
are interpreted disjunctively.
However, this kind of nesting representation has a limitation because of
its tree structure: it is not clear how we can represent three threads
$t_0, t_1$ and $t_2$ where either $e_0$ and $e_1$ succeed or $e_0$ and
$e_2$ succeed.
If we take the idea of event structure~\citep{winskel1980},
we could make a deduction system where each inference step results in a
coherence space whose elements are sequents.

\subsection{Understanding Waitfreedom}

Although the definition of waitfreedom is operational
(Section~\ref{waitfreedom}),
there is a topological characterization~\citep{Herlihy99,Saks:1993vq} of
waitfreely solvable problems using contractibility.
Since we have obtained a logical characterization, studying the direct
connection between the logical and topological characterizations would
be interesting.

\section{From Philosophical Perspectives}

Mathematical logic first succeeded in formalizing mathematics.
After that, there have been many attempts to investigate analytic
philosophy
using formal logics: relevance logic, modal epistemic
logics~\citep{sep-logic-epistemic},
dynamic epistemic logic~\citep{ditmarsch2007dynamic},
inquisitive logic~\citep{ciardelli2011},
deontic logics~\citep{von1951deontic} and
so on.  Among those investigations, in some cases, substructural logics
play important roles.  For example, relevance logic is a famous
substructural logic lacking weakening and inquisitive logic is a ``weak
logic'' (a logic possibly without substitution-closedness) between
intuitionistic and classical logics~\citep{ciardelli2011}.

G\"odel-Dummett logic~\citep{dummett59} was invented for algebraic
reasons, but now we found a constructive justification of Dummett's axiom
$(\phi\imp\psi)\lor(\psi\imp\phi)$ as the result of speed-racing during
proof reduction.  Since the semantics presented in
Chapter~\ref{ch:lambda} seems somewhat too complicated,
there is room for simplification and better understanding of
G\"odel-Dummett logic.

Abelian logic can provide a way to express
exchanges and indebtedness.
The Amida axiom $(\phi\limp
\psi)\otimes(\psi\limp\phi)$
can describe two agents' exchange of $\phi$ and $\psi$ between two agents
or one agent's borrowing of $\phi$ for $\psi$ and returning $\phi$ for
$\psi$.
Since analysis of pattern of exchange is an important subject of
anthropology~\citep{kula1920}, Abelian logic can provide a
basis for describing social and cultural phenomena.

Dynamic epistemic logic~\citep{ditmarsch2007dynamic}, or its fragment
called public announcement
logic, is a modal logic that has a modality for agents' knowledge and
announcements.  However, in the current formulation of dynamic epistemic
logic, we cannot formalize the effect of repetitions\footnote{The point
is distinguishing a single announcement from repeated announcements of
the same content.  For that, we do not want contraction.} of announcements or
cancels of announcements.  In Abelian logic, we can multiply any formula
with any integer, thus there is a possibility of representing repetition
or cancels of announcements using a modal logic based on Abelian logic.

Also, since Abelian logic is modeled by Abelian groups, it can express the
sense of indebtedness or obligation in a quantitative way.
Deontic logic~\citep{von1951deontic,sep-logic-deontic} tried to capture permission,
obligation and so on.
Using Abelian logic, we hope to address how much duty is imposed.
Since the Amida calculus can reduce some proofs in Abelian logic, it could
provide a way to analyze a complicated situation where many kinds of
obligations and permissions of different significance are interwound.
