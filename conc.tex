\chapter{Conclusion}

\subsection{}

We gave two inventions: hyper-lambda calculi and Amida logic.
Our conclusions and future work spans from mathematical logic, computer
science and philosophy.

\section{From Logical Perspectives}

We have seen two particular hyper-lambda calculi for two logics.
\citet{alg} classified axioms according to their syntactic complexities
and identified classes of axioms that can be translated into structural
sequent calculus rules and structural hypersequent calculus rules.
According to their classification, the prelinearity axiom
$(\phi\limp\psi)\oplus(\psi\limp\phi)$ and Amid axiom
$(\phi\limp\psi)\otimes(\psi\limp\phi) $belong to $\mathcal P_2$, a
claess whose elements can be translated into a finite set of
hypersequent structural rules.
Other classes $\mathcal N_2$ and $\mathcal P_3$ of axioms can also be
translated into finite sets of hypersequent structural rules,
thus, we expect the technique of hyper-lambda calculi applicable to
logics with these axioms on top of $\mathbf{FLe}$.

One particular logic worth trying is
the logic characterized by Kripke models with bounded width \fix{cite
[5] from terui}.
Since G\"odel--Dummett logic is a special case of the bounding width~1,
the generalization of width $k$ will provide waitfree computation on
weaker shared memory consistency.  Further,
an ambitious goal is to develop a general framework with which we can
develop hyper-lambda calculi for all logics characterized by axioms in
class $\mathcal P_3$.
Since the cut-elimination proof in \citep{alg} is algebraic,
we are yet to know the computational meaning of the cut-elimination.
Aforementioned framework would clarify the computational meaning of the
cut-elimination of hypersequent calculi.

\section{From Computer Science Perspectives}

\subsection{Implementation}

Since Amida logic is incompatible with contraction or weakening,
straightforward implementation Amida calculus on top of Haskell or OCaml
would not be a
good way to exploint the safety of Amida calculus.
Although Clean~\citep{parle1991} offers uniqueness types,
uniqueness types only reject contraction but accept weakening, so Clean
is not suitable either.

One promising framework on which to implement Amida calculus is linear ML%
\footnote{Although there are no
publications available, there is an implementation at
\texttt{https://github.com/pikatchu/LinearML}\enspace.},
whose type system is based on linear logic.

Another way is using the type level programming technique of Haskell.
\citet{DBLP:journals/corr/abs-1110-4163} implemented session types on top
of Haskell using the type level programming technique using
\texttt{Session} monad.  Since Haskell types can contain arbitrary trees
of symbols, they were able to encode session type information in Haskell
types.
Logically, this corresponds to having atomic formulae with complicated
structure so that we can encode session information in atomic formulae.
The advantage would be usability of existing Haskell infrastructure
including the optimizing compilers, execution environments, and
libraries.
The disadvantage, which the first author of
\citep{DBLP:journals/corr/abs-1110-4163} told me personally,
is error messages hard to read.
Since Haskell compiler is not aware of session types and just reports
pattern-matching errors in the encoding of session types, the error
messages are about the encoding not about the originally intended
session types.

% move to related work part
% Interestingly, their implementation uses heterogeneous
% lists~\citep{hetero}, which is similar to hyper-lambda terms because both
% heterogeneous lists and hyper-lambda terms are lists of programs of
% possibly different types.


\subsection{Reasoning about Hyper-Lambda Terms}

\subsection{Understanding Waitfreedom}

\section{From Philosophical Perspectives}

Mathematical logic first succeeded in formalizing mathematics.
After that, there have been many attempts to investigate analytic
philosophy
using formal logics: relevance logic, modal epistemic
logics~\citep{sep-logic-epistemic},
dynamic epistemic logic~\citep{ditmarsch2007dynamic},
inquisitive logic~\citep{ciardelli2011},
deontic logics~\citep{von1951deontic} and
so on.  Among those investigations, in some cases, substructural logics
play important roles.  For example, relevance logic is a famous
substructural logic lacking weakening and inquisive logic is a weak
logic (a logic possibly without substitution-closedness) between
intuitionistic and classical logics~\citep{ciardelli2011}.
Amida logic can provide a way to express
exchanges and indebtedness.  Amida axiom $(\phi\limp
\psi)\otimes(\psi\limp\phi)$
can describe two agents' exchange of $\phi$ and $\psi$ between two agents
or one agent's borrowing of $\phi$ for $\psi$ and returning $\phi$ for
$\psi$.
Since analysis of pattern of exchange is an important subject of
anthropology~\citep{kula1920}, Amida logic can provide a
basis for describing social and cultural phenomena.
