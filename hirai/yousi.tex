\documentclass{jsarticle}

\usepackage{graphicx}
\usepackage{amssymb,amsmath}
\usepackage{amscd}


\pagestyle{empty}

\begin{document}
 \begin{center}
$BO@J8$NFbMF$NMW;](B
\vskip 1cm

Hyper-Lambda Calculi\\
($B%O%$%Q!<%i%`%@7W;;(B)

\vskip 1cm
$B;aL>(B $BJ?0f!!MN0l(B
 \end{center}

\vskip 4cm

 We propose hyper-lambda calculi, the typed lambda calculi based on
 hypersequent calculi.  A hyper-lambda term is a finite
 sequence of lambda terms, representing concurrent processes.  We give
 two concrete hyper-lambda calculi: synchronous and asynchronous.  Both
 employ a pair of communication primitives exchanging their inputs.
 In the synchronous case, both sides succeed.  In the asynchronous case,
 at least one side obtains the other side's input.
 The synchronous calculus implements message-passing communication
 and session types;
 the asynchronous calculus characterizes shared-memory waitfree
 communication.
 Among processes of a typed hyper-lambda term,
 all succeed in the synchronous case while
 at least one succeeds in the asynchronous case.
 Logically, the processes are interpreted conjunctively
 in the synchronous case but disjunctively in the asynchronous case.
 The synchronous calculus is based on Abelian logic:
 $(\phi\multimap\psi)\otimes(\psi\multimap\phi)$ on top of multiplicative
 additive fragment of intuitionistic linear
 logic (without some units);
 the asynchronous calculus is based on G\"odel--Dummett logic:
 $(\phi\supset\psi)\lor(\psi\supset\phi)$ on top of intuitionistic logic.
 The hyper-lambda calculi are in Curry--Howard correspondence with the
 deduction systems for these logics.
 We also treat another variant based on monoidal t-norm logic and
 an implementation using Haskell.

 Meanwhile, we discover a new representation of Abelian logic proofs
 called Amida nets.  Amida nets are based on a special kind of directed
 graphs called Lamarche's essential nets.
 On top of Lamarche's essential nets,
 we add a new kind of undirected edges called Amida edges.
 Similarly to the Amida lotteries used by Japanese children,
 when we follow a directed path in an Amida net, whenever we meet an
 Amida edge, we have
 to cross the Amida edge.  While Lamarche's essential nets characterize
 fragments of intuitionistic linear logic, our Amida nets characterize
 Abelian logic.  The name Amida calculus comes from the Amida lotteries.

 The logics treated in this thesis, G\"odel-Dummett, monoidal t-norm
 and Abelian logics were initially developed for algebraic interests,
 but we reveal that these logics have computational applications as type
 systems for concurrent processes.
\end{document}
