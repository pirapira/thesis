\chapter{Introduction}

At the core of computer science lies the interplay of static formalism
and dynamic behaviour.  We can find examples in typed lambda calculi,
where static formalism of type derivations interacts with dynamic
behaviour of lambda terms.
Type derivations of static objects associtating lambda terms to types.
The reduction on terms gives dynamics, defining which term reduce
to which.  Static type derivations can guarantee dynamic properties of
terms such as strong normalization \fix{cite}, deadlock freedom
\fix{cite}
and so on.

Dynamic behaviour involves time.
One simple notion of time is that of totally-ordered events where
one event happens before the other or the other before one.
This sentence is syntactically similar to the Dummett axiom that states
one proposition implies the other or the other implies one.
We investigate whether this syntactic similarity is reflected semantically.

\subsection{\fix{subsection}}

The Curry--Howard isomorphism isused to give computational content
to mathematical proofs, and on the other hand to give mathematical
guarantees to computer programs.


There have been several attempts.
The best thing we can hope is that the computational
interpretation coincides with something we know already.
That is the case indeed.

We extended the Curry--Howard correspondence to G\"odel--Dummett logic.
The Curry--Howard correspondence is \fix{add}

G\"odel--Dummett logic is an intermediate logic \fix{cite} between
classical logic and intuitionistic logic.  Every theorem in
intuitionistic logic is a theorem in G\"odel--Dummet logic and every
theorem in G\"odel--Dummett logic is a theorem in classical logic.
The inclusion is strict.  The law of excluded middle
$\varphi\vee(\varphi\supset \bot)$ is not a theorem in G\"odel--Dummett
logic and $(\varphi\supset\psi)\vee(\psi\supset\varphi)$ is not a
theorem in intuitionistic logic.

Without modalities.


In terms of Kripke semantics, the axiom is sound and complete for
requiring that the models are linearly ordered.
In terms of algebraic semantics, the axiom is sound and complete for
requiring that the truth values are linearly ordered.  \fix{ is this
usage of ``truth values'' correct? }
In terms of computation, what is it?


who said logic is about temporality.
Waitfreedom is a kind of temporality appeared in computer science.
This research shows that the proof theory is not only about abstract
notions of temporality, but also about concrete notions of temporality
appearing in computer science.

\section{Curry--Howard Correspondence for Different Logics}

Prawitz

Pfenning

Davies--Pfenning S4 A Modal Analysis of Staged Computation

Curien

Hypersequent

Game

\section{Location Aware Languages}

X10


Hypersequents were born to solve proof theoretical difficulties, but we
apply hypersequents to solve distributed computational difficulties.

\section{Properties}
