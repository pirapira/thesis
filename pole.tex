\chapter{Prelinearity as an Asynchronous Communication Scheme}

	% packages
% \usepackage{setspace}
% \doublespacing

% \usepackage{bussproofs}
% \usepackage{graphicx}
% \usepackage{amssymb}

% % title, author, etc
% \title{Some Disjunctive Tautologies as\\
%  Asynchronous Communication Schemes}
% \author{Yoichi Hirai}

% \usepackage{amsmath}	% required for `\align*' (yatex added)
% \begin{document}

% \maketitle


\section{Summary}
%%% abstract
\citet{danos-krivine} investigate
the computational semantics of disjunctive tautologies
such as the excluded middle and the symmetric excluded middle.
They state that those tautologies are synchronization schemes.
Indeed in their formalism, reduction of one term can wait
for another, making synchronization.
However,
one can find no communication among different
concurrent elements in an executable.
In the current paper, we show that logical axioms can specify how
concurrent threads exchange information.

We observe the computational behavior of prelinearity axiom
$(\phi\limp\psi)\oplus(\psi\limp\phi)$ and left weakening
on top of IMALL2, which is the fragment of second order
intuitionistic linear logic with connectives $\{\forall, \limp,
\oplus\}$.
We interpret the disjunction~$\oplus$ in the axiom
nondeterministically: whether
one thread can give information to the other or
vice versa.
In Sect.~\ref{sec:sync}, where threads wait for others,
we externally specify such nondeterministic choice.
In Sect.~\ref{sec:async}, where threads do not wait for others,
the nondeterminism appears automatically as a slower thread cannot pass
information to a faster one.
We adapt \citet{danos-krivine}'s realizability argument to our
synchronous case and give simulation from the
asynchronous case to the synchronous case.

Our computational semantics here is closely related to
\citet{hiraiflops2012}'s for G\"odel--Dummett logic,
which is axiomatized by Dummett axiom $(\phi\imp\psi)\lor(\psi\imp\phi)$
on top of intuitionistic logic.


%%% 1. introduction

% 1a operational semantics for GD case

% 1b difference from Danos--Krivine

%%% 2. terms types and _models_?
\section{Synchronous Semantics}
\label{sec:sync}

% 2a terms -- unchanged -- remove pi or anything

\subsection{Dynamics}
We consider a programming language, which is a modification of
Danos and Krivine's.
We assume a set~$\pvar$ of propositional variables whose cardinality is
countably infinite.
Channels have involution $c\mapsto\co c$ with
$c\neq \co c$ and
$\co{\co c} = c$.
A \textit{term}~$t$ is defined by BNF:
\[
 t::= x
 \mid (t)t
 \mid \lambda x.t
 \mid \ast_t
 \mid \comod c c
 \mid (t\conc t)
 \mid \mat t x t x t
\]
where $x$ is a variable and $c$ is a channel.
A \textit{stack}~$\pi$ is defined by BNF:
\[
 \pi ::= \epsilon
 \mid t\cdot \pi
 \mid \mats x t \pi x t \pi
 \enspace.
\]
We write the set of terms as $\Lambda$ and stacks~$\Pi$.
An \textit{executable}~$e$ is a multiset on $\Lambda\times\Pi$.

% 2a' reduction
A \textit{schedule}~$\sche$ is a total order on channels.
The \textit{reduction} relation~$\red$ on executables
is defined to be the smallest preorder
that satisfies:
\begin{description}
 \item[(cong)] if
      $e_0         \red e_1$
      then
      $(e_0 \conc e) \red (e_1\conc e)$\enspace;
 \item[(push)]
	    $[(t)u,\pi]      \red [t,u\cdot\pi]$      \enspace;
 \item[(store)]
	    $[\lambda x.t,u\cdot\pi]
	     \red
	     [t[\ast_u/x],      \pi]$\enspace;
 \item[(load)]
	    $[\ast_u,\pi]\red [u,\pi]$\enspace;
 \item[(dist)]
           $[(t\conc u),\pi]  \red [t,\pi\conc u,\pi]$ \enspace;
 \item[(ask)]
      $[\mat t x u y v,\pi]\red [t,\mats x u \pi y v \pi]$\enspace;
 \item[(ansL)]
           $[\inl{v}, \mats xt\pi yu\sigma] \red [t[v/x],\pi] $ \enspace;
 \item[(ansR)]
           $[\inr{w}, \mats xt\pi yu\sigma] \red [u[w/y],\sigma] $ \enspace;
 \item[(com0)]
	    $[\comod c{\co c}, t\cdot\pi\conc \comod{\co c}c,
	    u\cdot\sigma] \red
	    [u,\pi]$\enspace(if $\co c\sche c$ but $c\not\sche \co c$)\enspace;
 \item[(com1)]
	    $[\comod c{\co c}, t\cdot\pi\conc \comod{\co c}c,
	    u\cdot\sigma] \red
	    [t,\sigma]$\enspace(if $c\sche \co c$ but $\co c\not\sche c$)\enspace; and
 \item[(com2)]
	    $[\comod c{\co c}, t\cdot\pi\conc \comod{\co c}c,
	    u\cdot\sigma] \red
	    [u,\pi\conc t,\sigma]$\enspace(if $c\sche \co c$ and $\co c\sche
       c$)\enspace.
\end{description}
Rules (cong), (push), (store), (load) and (dist) come from \citet{danos-krivine}.
The operational semantics for the additive constructs comes from \citet{curien-mac}.
Below, we sometimes omit the outermost parentheses (i.e. [ and ]) for multisets.

% I want to write these rules are non-blocking here.
% but, actually, Danos and Krivines rules are also non-blocking.
% they just busy-wait.
% In order to remove busy-waiting, we need types.

% 2b type system -- remove pierce -- add com'

\subsection{Statics}
For a set~$S$,
$\form(S)$ is the set of formulas~$\phi$:
\[
\phi::= s \mid X \mid \phi\limp\phi\mid \phi\oplus\phi\mid
\forall X \phi
\]
where $s\in S$ and $X\in \pvar$.
The connective $\forall$ connects stronger than $\oplus$, which is
stronger than $\limp$.
Recurring $\limp$,
$\phi_0\limp\phi_1\limp\cdots\limp\phi_n$ is defined inductively on~$n$
as
$\phi_0\limp(\phi_1\limp\cdots\limp\phi_n)$.
A type is an element of $\form(\emptyset)$.

A \textit{sequent} $\sequent{\G}{\tj t\phi}$ consists of a
context~$\G$ and a term~$t$ associated with a type~$\phi$.
The \textit{context} is a finite sequence of variables associated with types.

A \textit{derivation} is a tree composed of the derivation rules below:\\ \noindent

\AxiomC{}
\LL  {Ax}
\useq{\xphi}{\xphi}
\DisplayProof
\ruleskip
%
\aseq{\xphi,\G}{\tj t\psi}
\LL{$\limp$I}
\useq{\G}{\tj{\lambda x.t}{\phi\limp\psi}}
\DisplayProof
\ruleskip
%
\aseq{\G}{\tj t{\phi\limp\psi}}
\aseq{\D}{\tj u\phi}
\LL   {$\limp$E}
\bseq{\G,\D}{\tj{(t)u}\psi}
\DisplayProof
\ruleskip
%
\aseq{\G}{\tj{t}{\phi}}
\LL  {$\oplus$I}
\useq{\G}{\tj{\inl{t}}{\phi\oplus\psi}}
\DisplayProof
\ruleskip
%
\aseq{\G}{\tj{t}{\psi}}
\LL  {$\oplus$I}
\useq{\G}{\tj{\inr{t}}{\phi\oplus\psi}}
\DisplayProof
\ruleskip
%
\aseq{\G}{\tj{t}{\phi\oplus\psi}}
\aseq{\D,\xphi}{\tj{u_0}{\theta}}
\aseq{\D,\ypsi}{\tj{u_1}{\theta}}
\LL  {$\oplus$E}
\tseq{\G,\D}{\tj{\mat{t}{x}{u_0}{y}{u_1}}{\theta}}
\DisplayProof
\ruleskip
%
\aseq\G{\tj t\phi}
\LL   {$\forall$I}
\useq\G{\tj t{\forall X\phi}}
\DisplayProof ($\G$ does not contain $X$ freely)
\ruleskip
%
\aseq{\G}{\tj{t}{\forall X\phi}}
\LL   {$\forall$E}
\useq{\G}{\tj{t}{\phi[\psi/X]}}
\DisplayProof
\ruleskip
%
\aseq{\tj{x}{\phi\limp\psi},\G}{\tj t\theta}
\aseq{\tj{y}{\psi\limp\phi},\D}{\tj u\theta}
\LL{Com}
\bseq{\G,\D}{\tj{(t[\comodL/x]\conc u[\comodR/y])}{\theta}}
\DisplayProof
\ruleskip
%
\aseq{\G,\tj{x}{\phi},\tj{y}{\psi},\D}{\tj{t}{\theta}}
\LL{Ex}
\useq{\G,\tj{y}{\psi},\tj{x}{\phi},\D}{\tj{t}{\theta}}
\DisplayProof

\fix{because of Com rule, it is affine?}

\subsection{Specification Using Poles}

% 2C truth values and models. make them disjunctive!
%    so that the empty set does not count as successful.
%    Also, make it closed for ``for all''
A \textit{pole}~$\bbot$ is a set of executables
which satisfies
\begin{enumerate}
 \item \label{red-closed} $e$ is in $\bbot$ if $e\red e'$ and
       $e'\in\bbot$; and
 \item \label{conc-closed} if $e$ or $f$ is in $\bbot$
       then $e\conc f\in\bbot$.
\end{enumerate}
Condition~\ref{conc-closed}. is different from that from \citet{danos-krivine}'s
definition.
There, the condition says if $e$ \textit{and} $f$ \textit{are} in
$\bbot$, then $e\conc f$ is in $\bbot$.  Our choice here is influenced
by hypersequents~\citep{avron91} and hyper-lambda
calculi~\citep{hiraiflops2012},
where
components are interpreted disjunctively.

An \textit{environment} is a pair of a stack and an executable.
The set of environments is written as~$E$.
A \textit{program} is a pair of a term an executable.
For a set~$\mathcal Z$ of environments, $\mathcal Z\rightarrow\bbot$ denotes
the set of programs $(t,e)$ such that
for any environment $(\pi,e')\in \mathcal Z$,
the executable $t,\pi\conc e \conc e'$ is in $\bbot$.
A set $\mathcal X$ of programs is called a truth value
iff there exists
such $\mathcal Z$ that $\mathcal X = \mathcal Z\rightarrow \bbot$.
We define programs and environments because if we continue using
terms and stacks, the proof of adequacy (Prop.~\ref{c:adequacy}) does
not go through the case for
($\limp$E,~\textminus), \ref{c:second}.

For $\phi\in\form(2^E)$ and $|\cdot|^-_0\colon\pvar\rightarrow 2^E$,
we define $\nsem{\phi}\colon \form(2^E)\rightarrow
2^E$ inductively as
\begin{align*}
 \nsem{\mathcal Z} &= \mathcal Z \\
 \nsem{X}&= |X|_0^- \\
 \nsem{\phi\limp\psi}&=
 \{(t\cdot\pi, e_0\conc e_1)\mid
 (t,e_0)\in\nsem\phi\rightarrow\bbot \text{ and }(\pi,e_1)\in\nsem\psi\}\\
 \nsem{\phi\oplus\psi}=& \{(\mats x{t}{\pi}y{u}{\sigma}, f)\mid\\ &
 t[v/x],\pi\conc f\conc f'\in\bbot \text{ for all } (v,f')\in\nsem{\phi}\rightarrow\bbot\text{
 and }\\ &
 u[w/y],\sigma\conc f\conc f'\in\bbot \text{ for all } (w,f')\in\nsem{\psi}\rightarrow\bbot\}
 \\
 \nsem{\forall X\phi}=&
 \bigcup_{\mathcal Z\in 2^\Pi} \nsem{\phi[\mathcal Z/X]}\enspace.
\end{align*}
Using this, we define $\sem \phi=\nsem{\phi}\rightarrow\bbot$.
We have an equality
$\nsem{\phi\limp\psi} = \{(t\cdot\pi, e_0\conc e_1)\mid
(t,e_0)\in\sem\phi\text{ and }(\pi,e_1)\in\nsem\psi\}$.
Moreover, for types~$\phi$ and $\psi$, we define $\sempair{\phi,\psi}$
as the set of triples $(t,u,e)$ of terms $t$ and $u$ and an executable~$e$
such
that
$t,\pi\conc u,\sigma\conc e\conc e_0\conc e_1
\in\bbot$ for any $(\pi,e_0)\in\nsem\phi$
and $(\sigma,e_1)\in\nsem\psi$.

\begin{proposition}
 \label{squash}
 If $(t,u,e)$ is in $\sem{\phi,\phi}$ then $((t\conc u), e)$ is in
 $\sem\phi$.
\end{proposition}
\begin{proof}
 For any $\pi\in\nsem\phi$,
 the executable $[(t\conc u),\pi\conc e]$ reduces to $[t,\pi\conc
 u,\pi\conc e]$ by (dist).
 The reduct is in $\bbot$ by assumption.
 So, the original executable is also in~$\bbot$.
\end{proof}

\begin{proposition}
 \label{comod-type}
 $(\comodL,\comodR,\emptyset)\in\sempair{\phi\limp\psi,\psi\limp\phi}$
 for any types~$\phi$ and $\psi$.
\end{proposition}
\begin{proof}
 Take any $(t\cdot\sigma,e_{0}\conc e_{1})\in\nsem{\phi\limp\psi}$
 and $(u\cdot\pi,f_{0}\conc f_{1})\in\nsem{\psi\limp\phi}$ so that
 $(t, e_{0})\in\sem{\phi}$, $(\sigma,e_{1})\in\nsem{\psi}$,
 $(u, f_{0})\in\sem{\psi}$ and $(\pi,f_{1})\in\nsem{\phi}$ hold.
 We claim that $e=\comodL,t\cdot\sigma\conc\comodR,u\cdot\pi\conc
 e_{0}\conc e_{1}\conc f_{0}\conc f_{1}$ is in
 $\bbot$.
 Depending on the schedule, $e$ might reduce to
 $ t,\pi\conc e_{0}\conc e_{1}\conc f_{0}\conc f_{1}$ and otherwise to
 $ u,\sigma\conc e_{0}\conc e_{1}\conc f_{0}\conc f_{1}$ or
 $ t,\pi\conc u,\sigma\conc e_{0}\conc e_{1}\conc f_{0}\conc f_{1}$,
 all of which are in $\bbot$ because
 $t,\pi\conc e_{0}\conc f_{1}$ and
 $u,\sigma\conc e_{1}\conc f_{0}$ are in $\bbot$.
 Since $\bbot$ is closed for $\red^{-1}$,
 we have $e\in\bbot$.
\end{proof}

% 2d the adequacy lemma
For $\G = \tj{x_1}{\phi_1},\ldots,\tj{x_n}{\phi_n}$,
we denote by $\sem{\G}$ the set of pairs $(\vec t,e)$ where
$\vec{t} = (t_1,\dots,t_n)$, $e = \bigparallel_{1\le i\le n} e_i$
 and each pair~$(t_i, e_i)$ is in $\sem{\phi_i}$.
For such tuple, $[\vec{t}/\G]$ denotes the simultaneous substitution
$[t_i/x_i]_{i}$.

Here we state adequacy.  What we want is statement~\ref{c:first}.
However, when we try to prove \ref{c:first} by induction on derivations,
the case for Com rule fails.  Thus we deal with two derivations at the
same time.

\begin{proposition}[Adequacy]
 \label{c:adequacy}
 Let these sequents be derivable:
 $  \sequent{\G}{\tj{t}{\phi}} $
 and
 $  \sequent{\D}{\tj{u}{\psi}}$\enspace.
 Under these conditions, we state that:
 \begin{enumerate}[label=(\arabic{*}), ref=\textit{(\arabic{*})}]
  \item \label{c:first} the program
	$
	(t[\vec{g}/\G], e)
	$
	is in $\sem\phi$
	 for any
	$(\vec{g},e)\in\sem{\G}$; and that
  \item \label{c:second}
	when $\G$ and $\D$ are respectively
	equal to $\tj x\theta, \hat\G$ and $\tj y\tau, \hat\D$ up to exchange,
	the following triple is in $\sempair{\phi,\psi}$:
	\[\left(
	t[v/x][\vec g/\hat \G],\quad
	u[w/y][\vec d/\hat \D],\quad
	 e \conc  f \conc e'
	\right)
	\]
	for any
	$(v,w,e')\in\sempair{\theta,\tau}$,
	$(\vec g, e)\in\sem{\hat\G}$ and
	$(\vec d, f)\in\sem{\hat\D}$.
  \end{enumerate}
\end{proposition}
\begin{proof}
 We prove both statements at the same time by induction on the sum of
 the heights of the derivations.  Here we identify sequents up to exchange.
  \begin{description}
  \item[(Ax, Ax)] When both derivations consist of only axiom rules,
       the statements follow from the definitions of $\sem{\phi}$ and
       $\sempair{\phi,\psi}$.
   \item[($\limp$I, \textminus)]
	The derivation for $t$ ends as
	\[
	\aseq{\tj{\hat x}{\phi_0},\G}{\tj{t_1}{\phi_1}}
	\useq{\G}{\tj{\lambda{\hat{x}}.t_1}{\phi_0\limp\phi_1}}
	\DisplayProof\enspace.
	\]
	\begin{enumerate}[label=\textit{(\arabic{*})}]
	 \item Take any
	       $(\vec g, e)\in\sem{\G}$
	       and
	       $(t_0\cdot\pi,e_0\conc e_1)\in\nsem{\phi_0\limp\phi_1}$
	       so that $(t_0,e_0)\in\sem{\phi_0}$ and
	       $(\pi,e_1)\in\nsem{\phi_1}$ hold.
	       We have to show that the following executable is in
	       $\bbot$:
	       \[
	       (\lambda\hat x.t_1)[\vec g/\G],
	       t_0\cdot\pi\conc e\conc e_0\conc e_1\enspace.
	       \]
	       The executable reduces to
	       \[
		t_1[\ast_{t_0}/\hat x][\vec{g}/\G],\pi\conc{e}\conc e_0\conc{e_1}\enspace.
	       \]
	       We have $(\ast_{t_0}, e_0)\in\sem{\phi_0}$ so, by
	       induction hypothesis~\ref{c:first},
	       the reduct is in $\bbot$.
	       Since $\bbot$ is closed for $\rev\red$, the original
	       executable is in $\bbot$, too.
	 \item Take any $(v,w,e')\in\sempair{\theta,\tau}$,
	       $(\vec g, e)\in\sem{\hat\G}$,
	       $(\vec d, f)\in\sem{\hat\D}$,
	       $(t_0\cdot\pi,e_0'\conc e_1')\in\nsem{\phi_0\limp\phi_1}$ and
	       $(\sigma,f'')\in\nsem\psi$ so that
	       $(t_0,e_0')\in\sem{\phi_0}$ and
	       $(\pi,e_1')\in\nsem{\phi_1}$.
	       We have to show that this executable is in $\bbot$:
	       \[
	       (\lambda \hat x. t_1)[v/x][\vec{g}/\G],t_0\pi\conc
	       u[w/y][\vec{d}/\D],\sigma\conc
	       e'\conc {e}\conc {f} \conc{e_0'}\conc{e_1'}\enspace.
	       \]
	       By (store) and (cong), the executable reduces to
	       \[
		t_1[v/x][\vec{g}/\G][\ast_{t_0}/\hat x],\pi\conc
	       u[w/y][\vec{d}/\D],\sigma\conc
	       e'\conc {e}\conc {f} \conc{e_0'}\conc{e_1'}\enspace.
	       \]
	       Since $(\ast_{t_0},e_0')$ is in $\sem{\phi_0}$,
	       by induction hypothesis~\ref{c:second}, the reduct is in
	       $\bbot$.  Since $\bbot$ is closed for $\rev\red$,
	       the original executable is also in $\bbot$.
	\end{enumerate}
   \item[($\limp$E, \textminus)]
	The derivation for $t$ ends as
	\[
	\aseq{\G_0}{\tj{t_0}{\phi'\limp\phi}}
	\aseq{\G_1}{\tj{t_1}{\phi'}}
	\bseq{\G_0,\G_1}{\tj{(t_0)t_1}{\phi}}
	\DisplayProof\enspace.
	\]
	\begin{enumerate}[label=\textit{(\arabic{*})}]
	 \item Take any
	       $(\vec{g_0},{e_0})\in\sem{\G_0}$,
	       $(\vec{g_1},{e_1})\in\sem{\G_1}$ and
	       $(\pi,e'')\in\nsem{\phi}$.
	       We have to show that this executable is in $\bbot$:
	       \[
		((t_0)t_1)[\vec{g_0}/\G_0][\vec{g_1}/\G_1],\pi\conc
	       {e_0}\conc {e_1}\conc e''.
	       \]
	       By (push), this executable reduces to
	       \[
		t_0[\vec{g_0}/\G_0], t_1[\vec{g_1}/\G_1]\cdot \pi
	       \conc {e_0}\conc {e_1}\conc e''.
	       \]
	       By induction hypothesis~\ref{c:first} on both branches,
	       we have $(t_0[\vec{g_0/\G_0}],
	       {e_0})\in\sem{\phi'\limp\phi}$
	       and
	       $(t_1[\vec{g_1}/\G_1],{e_1})\in\sem{\phi'}$.
	       By the latter, we have $(t_1[\vec{g_1}/\G_1]\cdot\pi,
	       {e_1}\conc e'')\in\nsem{\phi'\limp\phi}$.
	       By definition of $\sem{\phi'\limp\phi}$, we have shown that the
	       reduct is in $\bbot$.
	       Since $\bbot$ is closed for $\red^{-1}$,
	       the original executable is also in $\bbot$.
	 \item Variable~$x$ might be contained in $\G_0$ or $\G_1$.
	       \begin{description}
		\item[(case $\G_0 = \tj{x}{\sigma},\hat\G_0$)]
		     Take
		     $(v,w,e')\in\sempair{\sigma,\tau}$,
		     $(\vec{g_0},{e_0})\in\sem{\hat\G_0}$,
		     $(\vec{g_1},{e_1})\in\sem{\G_1}$,
		     $(\vec{d},{f})\in\sem{\hat\D}$,
		     $(\pi,e'')\in\nsem{\phi}$ and
		     $(\sigma,f'')\in\nsem{\psi}$.
		     We have to show that the following executable is in
		     $\bbot$:
		     \[
		     ((t_0)t_1)[v/x][\vec{g_0}/\hat\G_0][\vec{g_1}/\G_1],\pi\conc
		     u' ,\sigma\conc \hat e\conc {{e_1}}\conc
		     {e''}\conc{f''}\enspace.
		     \]
		     where
		     $u' = u[w/y][\vec{d}/\hat\D]$ and
		     $\hat e= e'\conc {e_0}\conc {f}$.
		     By (push) and (cong), this executable reduces to
		     \[
		     t_0[v/x][\vec{g_0}/\hat\G_0],
		     t_1[\vec{g_1}/\G_1]\cdot\pi
		     \conc
		     u' ,\sigma\conc \hat e\conc {{e_1}}\conc
		     {e''}\conc{f''}\enspace.
		     \]
		     By induction hypothesis~\ref{c:first},
		     $(t_1[\vec{g_1}/\G_1],{e_1})$ is in $\sem{\phi'}$.
		     In addition to this, $(\pi,e'')$ is in $\nsem{\phi}$,
		     making $(t_1[\vec{g_1}/\G_1], {e_1}\conc e'')$ in
		     $\sem{\phi'\limp\phi}$.
		     By induction hypothesis~\ref{c:second},
		     $(t_0[v/x][\vec{g_0}/\hat\G_0], u', \hat e)$ is in
		     $\sempair{\phi'\limp\phi,\psi}$.
		     These make the reduct a member of $\bbot$.
		     Since $\bbot$ is closed for $\rev\red$, the original
		     executable is also in $\bbot$.
		\item[(case $\G_1 = \tj{x}{\sigma},\hat\G_1$)]
		     Take any
		     $(v,w,e')\in\sempair{\sigma,\tau}$,
		     $(\vec{g_0},{e_0})\in\sem{\G_0}$,
		     $(\vec{g_1},{e_1})\in\sem{\hat\G_1}$,
		     $(\vec{d},{f})\in\sem{\hat\D}$,
		     $(\pi,e'')\in\nsem{\phi}$ and
		     $(\sigma,f'')\in\nsem{\psi}$.
		     We have to show that the following executable is in $\bbot$:
		     \[
		     ((t_0)t_1)[v/x][\vec{g_0}/\G_0][\vec{g_1}/\hat\G_1],\pi\conc
		     {e_0}
		     \conc\hat{f}\conc
		     e''
		     \]
		     where
		     $\hat f = u[w/y][\vec{d}/\hat\D],\sigma\conc
		     {f}\conc f''\conc {e_1}\conc e'$.
		     By (push) and (cong), this executable reduces
		     to
		     \[
		     t_0[\vec{g_0}/\G_0],t_1[v/x][\vec{g_1}/\hat\G_1]\cdot \pi\conc
		     {e_0}
		     \conc\hat{f}\conc
		     e''\enspace.
		     \]
		     By induction hypothesis~\ref{c:second},
		     $(t_1[v/x][\vec{g_1}/\hat\G_1],
		     u[w/y][\vec{d}/\hat\D], {e_1}\conc {f}\conc
		     e')$ is in $\sempair{\phi',\psi}$.
		     So, since $(\sigma,f'')$ is in $\nsem{\psi}$,
		     $(t_1[v/x][\vec{g_1}/\hat\G_1],\hat f)$ is in
		     $\sem{\phi'}$.
		     Moreover we had $(\pi,e'')\in\nsem{\phi}$.
		     Combined,
		     $(t_1[v/x][\vec{g_1}/\hat\G_1]\cdot\pi,\hat f\conc e'')$
		     is in $\nsem{\phi'\limp\phi}$\kern -1pt.
		     By induction hypothesis~\ref{c:first},
		     $(t_0[\vec{g_0}/\G_0],\hat e)$ is in
		     $\sem{\phi'\limp\phi}$.
		     So, the reduct is in $\bbot$, making the original
		     executable a member of~$\bbot$.
	       \end{description}
	\end{enumerate}
   \item[($\oplus$I, \textminus)]
	Without loss of generality, we can assume that the
	derivation for $t$ ends as
	\[
	\aseq{\G}{\tj{t'}{\phi_0}}
	\useq{\G}{\tj{\inl{t'}}{\phi_0\oplus\phi_1}}
	\DisplayProof\enspace.
	\]
	\begin{enumerate}[label=\textit{(\arabic{*})}]
	 \item Take any $(\vec g, e)\in\sem{\G}$ and
	       $(\mats {\hat x}v{\pi_0}{\hat y}w{\pi_1}, e'')$
	       in $\nsem{\phi_0\oplus\phi_1}$.
	       By induction hypothesis~\ref{c:first},
	       $(t'[\vec g/\G], e)$ is in $\sem{\phi_0}$.
	       By definition of $\nsem{\phi_0\oplus\phi_1}$,
	       the executable
	       $v[t'[\vec g/\G]/\hat x],\pi_0\conc{e}\conc e''$
	       is in
	       $\bbot$.
	       Since $\bbot$ is closed for $\rev\red$,
	       the executable
	       $\inl{t'}[\vec g/\G], \mats {\hat x}v{\pi_0}{\hat
	       y}w{\pi_1}
	       \conc{e}\conc e''$
	       is also in $\bbot$.
	       This shows the statement because we chose an arbitrary
	       element of $\nsem{\phi_0\oplus\phi_1}$.
	 \item Take any $(v,w,e')\in\sempair{\theta,\tau}$,
	       $(\vec g, e)\in\sem{\hat\G}$,
	       $(\vec d, f)\in\sem{\hat\D}$,
	       $(\mats{\hat x}{\hat v}{\pi_0}{\hat y}w{\pi_1}, e'')\in
	       \nsem{\phi_0\oplus\phi_1}$ and
	       $(\sigma,f'')\in\nsem{\psi}$.
	       We have to show that the following executable is in
	       $\bbot$:
	       \[
		(\inl t)[v/x][\vec g/\hat\G], \mats{\hat x}{\hat v}{\pi_0}{\hat
	       y}w{\pi_1}\conc
	       u',\sigma\conc
	       \hat e\conc e''\conc f''
	       \]
	       where
	       $u' = u[w/y][\vec d/\hat\D]$ and $\hat e=e'\conc  e\conc
	       f$.
	       By (ansL), the executable reduces to
	       $\hat v[t'[v/x][\vec g/\hat\G]/\hat x],\pi_0 \conc
	       u',\sigma\conc
	       \hat e\conc e''\conc f''
	       $.
	       By induction hypothesis~\ref{c:second},
	       $(t'[v/x][\vec g/\hat \G], u', \hat e)$ is in
	       $\sempair{\phi_0,\psi}$.
	       So, $(t'[v/x][\vec g/\hat\G], [u',\sigma\conc \hat e\conc
	       f''])$ is in $\sem{\phi_0}$.
	       By definition of $\nsem{\phi_0\oplus\phi_1}$,
	       the reduct is in $\bbot$.
	       Since $\bbot$ is closed for $\rev\red$,
	       the original executable is also in $\bbot$.
	\end{enumerate}
   \item[($\oplus$E, \textminus)]
	The derivation for $t$ ends in
	\[
	\aseq{\G_0}{\tj{t}{\phi_0\oplus\phi_1}}
	\aseq{\G_1,\tj{\hat{x}}{\phi_0}}{\tj{t_0}{\phi}}
	\aseq{\G_1,\tj{\hat{y}}{\phi_1}}{\tj{t_1}{\phi}}
	\tseq{\G_0,\G_1}{\tj{\mat{t}{\hat x}{t_0}{\hat y}{t_1}}{\phi}}
	\DisplayProof\enspace.
	\]
	\begin{enumerate}[label=\textit{(\arabic{*})}]
	 \item Take any $(\vec{g_0},{e_0})\in\sem{\G_0}$,\quad
	       $(\vec{g_1},{e_1})\in\sem{\G_1}$
	       and
	       $(\pi,e')\in\nsem{\phi}$.
	       We have to show that the executable
	       \[
	       (\mat t{\hat{x}}{t_0}{\hat{y}}{t_1})[\vec{ g_0}/\G_0][\vec{g_1}/\G_1],
	       \pi\conc {e_0}\conc e'\conc {e_1}
	       \]
	       is in $\bbot$.
	       By (ask), the executable reduces to
	       \[
		t[\vec{g_0}/\G_0],
	       \mats{\hat{x}}{t_0[\vec{g_1}/\G_1]}{\pi}
	       {\hat{y}}{t_1[\vec{g_1}/\G_1]}{\pi}\conc{e_0}\conc
	       {e_1}\enspace.
	       \]
	       We claim that
	       $(\mats{\hat{x}}{t_0[\vec{g_0/\G_0}]}{\pi}{\hat{y}}{t_1}{\pi},{e_1}\conc
	       e')$ is in $\nsem{\phi\oplus\psi}$ and that
	       $(t[\vec{g_0}/\G_0],{e_0})$ is in $\sem{\phi\oplus\psi}$.  The first
	       claim is shown by induction hypothesis~\ref{c:first}
	       stating that $(t_0[v/\hat{x}][\vec{g_1}/\G_1], {e_1}\conc
	       e'')$ is in $\sem{\phi}$ for any $(v,e'')\in\sem{\phi_0}$
	       and similarly to~$t_1$.
	       The second claim follows from induction
	       hypothesis~\ref{c:first} on $t$.
	       By the two claims and by the definition of
	       $\sem{\phi\oplus\psi}$,
	       we have shown that the reduct is in $\bbot$ and
	       thence that the original executable is in $\bbot$.
	 \item Variable $x$ might be in $\G_0$ or $\G_1$.
	       \begin{description}
		\item[(case $\G_0 = \tj x\theta,\hat\G_0$)]
		     Take any
		     $(v,w,e')\in\sempair{\theta,\tau}$,
		     $(\vec{g_0},{e_0})\in\sem{\hat\G_0}$,
		     $(\vec{g_1},{e_1})\in\sem{\G_1}$,
		     $(\vec d, f)\in\sem{\hat\D}$,
		     $(\pi,e')\in\nsem{\phi}$ and
		     $(\sigma,f')\in\nsem\psi$.
		     We have to show that the following executable is in
		     $\bbot$:
		     \begin{eqnarray*}
		     (\mat{t'}{\hat x}{t_0}{\hat
		     y}{t_1})[v/x][\vec{g_0}/\hat\G_0][\vec{g_1}/\G_1],\pi\conc\\
		     u[w/y][\vec d/\hat\D],\sigma\conc e'\conc
		     {e_0}\conc {e_1}\conc  f\conc e'\conc
		     f'.
		     \end{eqnarray*}
		     By (ask) and (cong), this executable reduces to
		     \begin{eqnarray*}
		     (t'[v/x][\vec{g_0}/\hat\G_0],
		      \mats{\hat x}{t_0[\vec{g_1}/\G_1]}{\pi}{\hat y}{t_1[\vec{g_1}/\G_1]}{\pi}
		      \conc\\
		     u[w/y][\vec d/\hat\D],\sigma\conc e'\conc
		     {e_0}\conc {e_1}\conc  f\conc e'\conc
		     f'\enspace.
		     \end{eqnarray*}
		     By induction hypothesis~\ref{c:second},
		     we have
		     \[
		     (t'[v/x][\vec{g_0}/\hat\G_0], u[w/y][\vec d/\hat\D],
		     e'\conc {e_0}\conc f) \in\sempair{\phi_0\oplus\phi_1,\psi}\enspace.
		     \]
		     Also, we defined $(\sigma,f')$ to be an element of
		     $\nsem\psi$.
		     By definition of $\sempair{\phi_0\oplus\phi_1,\psi}$,
		     the reduct is in $\bbot$.  Since $\bbot$ is closed for
		     $\rev\red$,
		     the original executable is also in $\bbot$.
		\item[(case $\G_1=\tj{x}{\theta},\hat\G_1$)]
		     Take any
		     $(v,w,e')\in\sempair{\theta,\tau}$,
		     $(\vec{g_0},{e_0})\in\sem{\G_0}$,
		     $(\vec{g_1},{e_1})\in\sem{\hat\G_1}$,
		     $(\vec{d},{f})\in\sem{\hat\D}$,
		     $(\pi,e')\in\nsem{\phi}$ and
		     $(\sigma,f')\in\nsem{\psi}$.
		     We have to show that the following executable is in
		     $\bbot$:
		     \begin{eqnarray*}
		      (\mat{t'}{\hat x}{t_0}{\hat
		       y}{t_1})[\vec{g_0}/\G_0][v/x][\vec{g_1}/\G_1],\pi\conc \\
		      u[w/y][\vec d/\hat\D],\sigma\conc e'\conc
		       {e_0}\conc {e_1}\conc f\conc e'\conc f'\enspace.
		     \end{eqnarray*}
		     By (ask) and (cong), this executable reduces to
		     \begin{align*}
		      e_{\mathrm r} =&
		       t'[\vec{g_0}/\G_0],
		       \mats{\hat x}
		       {(t_0[x/v][\vec{g_1}/\G_1])}{\pi}{\hat y}
		       {t_1[x/v][\vec{g_1}/\G_1]}{\pi}\conc \\
		      &u[w/y][\vec d/\hat\D],\sigma\conc e'\conc
		       {e_0}\conc {e_1}\conc  f\conc e'\conc f'\enspace.
		     \end{align*}
		     By induction hypothesis~\ref{c:second}, for any
		     $(v_0',e_0')\in\sem{\phi_0}$,
		     the triple
		     $(t_0[x/v][\vec{g_1}/\G_1][v_0'/\hat x],
		     u[w/y][\vec d/\hat \D], e'\conc {e_1}\conc
		     e_0'\conc  f)$ is in $\sempair{\phi,\psi}$ so
		     that
		     $(t_0[x/v][\vec{g_1}/\hat\G_1][v_0'/\hat x],
		     [u[w/y][\vec d/\hat D],\sigma\conc e'\conc
		     {e_1}\conc e_0'\conc {f}\conc f''])$ is in
		     $\sem\phi$.
		     We have a symmetric fact for $t_1$.
		     Thus,
		     \begin{align*}
		     (
		      & \mats{\hat x}
		       {(t_0[x/v][\vec{g_1}/\G_1])}{\pi}{\hat y}
		       {t_1[x/v][\vec{g_1}/\G_1]}{\pi}, \\ &
		      [u[w/y][\vec{d}/\hat\D]\conc e'\conc {e_1}\conc
		       f\conc f'']
		     )
		     \end{align*}
		     is in $\nsem{\phi_0\oplus\phi_1}$.
		     By induction hypothesis~\ref{c:first},
		     $(t[\vec{g_0}/\G_0],{e_0})$ is in
		     $\sem{\phi_0\oplus\phi_1}$.
		     By definition of $\sem{\phi_0\oplus\phi_1}$,
		     the reduct $e_{\mathrm r}$ is in $\bbot$.
		     Since $\bbot$ is closed for $\rev\red$,
		     the original executable is also in $\bbot$.
	       \end{description}
	\end{enumerate}
   \item[(Com, \textminus)]
	The derivation for $t$ ends in
	\[
	\aseq{\tj{\hat x}{\phi_0\limp\phi_1},\G_0}{\tj{t_0}{\phi}}
	\aseq{\tj{\hat y}{\phi_1\limp\phi_0},\G_1}{\tj{t_1}{\phi}}
	\bseq{\G_0,\G_1}{\tj{(t_0[\comodL/\hat x]\conc t_1
	[\comodR/\hat y])}{\phi}}
	\DisplayProof\enspace.
	\]
	\begin{enumerate}[label=\textit{(\arabic{*})}]
	\item Take any $(\vec{g_0},{e_0})\in\sem{\G_0}$,
	      $(\vec{g_1},{e_1})\in\sem{\G_1}$ and
	      $(\pi,e')\in\nsem{\phi}$.
	      By Prop.~\ref{comod-type}, we have
	      $(\comodL,\comodR,\emptyset)\in\sempair{\phi_0\limp\phi_1,\phi_1\limp\phi_0}$.
	      By induction hypothesis~\ref{c:second}, the terms
	      $t'_0=t_0[\comodL/\hat{x}][\vec{g_0}/\G_0]$ and
	      $t'_1=t_1[\comodR/\hat{y}][\vec{g_1}/\G_1]$
	      satisfy
	      $(t_0',t_1',{e_0}\conc
	      {e_1})\in\sempair{\phi,\phi}$.
	      By Prop.~\ref{squash}, we have $(t'_0\conc t'_1,
	      {e_0}\conc {e_1})\in\sem{\phi}$.
	\item
	      Without loss of generality,
	      we can assume $(\tj x\theta)\in\G_0$ so that
	      $\G_0 =\tj x\theta, {\hat\G_0}$ up to exchange.
	      Take any
	      $(v,w,e')\in\sempair{\theta,\tau}$,
	     $(\vec{g_0},{e_0})\in\sem{\hat\G_0}$,
	     $(\vec{g_1},{e_1})\in\sem{\G_1}$,
	     $(\vec d, f)\in\sem{\hat\D}$,
	     $(\pi,e'')\in\nsem{\phi}$ and
	     $(\sigma,f'')\in\nsem{\psi}$.
	     We have to show that this executable is in $\bbot$:
	     \begin{align*}
	      & (t_0'\conc
	      t'_1),\pi\conc
	      u',\sigma\conc e'\conc
	      {e_0}\conc {e_1}\conc  f\conc e''\conc f''
	     \end{align*}
	     where $t_0' = t_0[\comodL/\hat
	     x][\vec{g_0}/\hat\G_0][v/x]$,
	     $t'_1 = t_1[\comodR/\hat y][\vec{g_1}/\G_1]$ and $u' = u[w/y][\vec{d}/\hat\D]$.
	     By (dist) and (cong), this executable reduces to
	     \[
	      t'_0,\pi\conc t'_1,\pi\conc u',\sigma\conc e'\conc
	     {e_0}\conc {e_1}\conc{f}\conc e''\conc f''\enspace.
	     \]
	     Consider the derivation consisting of an axiom
	     $\sequent{\tj{\hat x}{\phi_0\limp\phi_1}}
	     {\tj{\hat x}{\phi_0\limp\phi_1}}$.
	     This derivation is shorter than the derivation for $u$
	     unless the derivation for $u$ also consists of a single axiom.
	     We can use the induction hypothesis~\ref{c:second} on
	     this derivation and the derivation of $u$, or, if the
	     derivation for $u$ also an axiom, we can do the same
	     argument as in (Ax, Ax) case to prove
	     statement~\ref{c:second}.
	     By this, $(\comodL, t_1',{e_1})$ is in
	     $\sempair{\phi_0\limp\phi_1,\phi}$
	     because of Prop.~\ref{comod-type}.
	     So, program $p =(\comodL,(t_1',\pi\conc{e_1}))$ is in
	     $\sem{\phi_0\limp\phi_1}$.
	     By induction hypothesis~\ref{c:second} for the derivation
	     for $t_0$, especially using the program~$p$ for
	     substituting~$\hat x$, we can show that
	     $(t_0',u',(t_1',\pi\conc {e_1}\conc {e_0}\conc
	     e'\conc {f}))$ is in $\sempair{\phi,\psi}$.
	     Thus, the reduct is in $\bbot$.
	     Since $\bbot$ is closed for $\rev\red$, the original
	     executable is also in $\bbot$.
	\end{enumerate}
  \item[($\forall$I, \textminus)]
       The derivation for $t$ ends in
       \aseq{\G}{\tj t\phi}
       \useq{\G}{\tj t{\forall X\phi}}
       \DisplayProof.
       \begin{enumerate}[label=\textit{(\arabic{*})}]
	\item Take any $(\vec g,{e})\in\sem{\G}$.
	      Since $\G$ does not contain $X$ freely,
	      $\sem\G$ does not change whatever
	      $\nsem{X}_0$ is.
	      On the other hand, since the induction
	      hypothesis~\ref{c:first}
	      holds for arbitrary $\nsem{X}_0$,
	      the program $t[\vec g/\G]$ is in
	      $\bigcap_{\mathcal Z\in 2^\Pi}\sem{\phi[\mathcal Z/X]}$.
	      Since $(\bigcap_{\mathcal Z\in 2^\Pi}\sem{\phi[\mathcal
	      Z/X]})\cdot (\bigcup_{\mathcal Z\in
	      2^\Pi}\nsem{\phi[\mathcal Z/X]})$ is a subset of $\bbot$,
	      the program is in $\sem{\forall X\phi} = \left(\bigcup_{\mathcal
	      Z\in 2^\Pi}\nsem{\phi[\mathcal Z/X]}\right) \rightarrow
	      \bbot$.
	\item Take any
	      $(v,w,e')\in\sempair{\theta,\tau}$,
	      $(\vec g, e)   \in\sem{\hat\G}$,
	      $(\pi,e'')\in\nsem{\forall X\phi}$ and
	      $(\sigma,f'')\in\nsem{\psi}$.
	      We can replace $X$ in the derivation of $t$ with
	      another propositional variable~$X'$ that does not
	      occur in the derivation of $u$.
	      We are going to use the induction hypothesis on the
	      renamed derivation with the terms and stacks taken above.
	      Since $\theta$ and $\hat\G$ do not contain
	      $X$ freely, we have
	      $(v,w,e')\in\sem{\theta[X'/X],\tau}$,
	      $(\vec{g},{e})\in\sem{\hat\G[X'/X]}$ and $(\pi,e'')\in
	      \nsem{\forall X'\phi[X'/X]}$.
	      By induction hypothesis~\ref{c:second}
	      on the renamed derivation,
	      the terms
	      $t' = t[v/x][\vec g/\G]$ and
	      $u' = u[w/y][\vec d/\D]$ satisfy
	      $(t',u', e\conc  f)\in\sem{\phi[X'/X],\psi'}$ for any $\nsem{X'}_0$.
	      That is, for $\mathcal Z\in 2^E$ that makes
	      $(\pi,e'')\in\nsem{\phi[X'/X][\mathcal Z/X']}$ and
	      $(\sigma,f'')\in\nsem{\psi[\mathcal Z/X']}$,
	      $t'\pi\conc u',\sigma\conc e''\conc f''$ is in $\bbot$,
	      making $t',\pi\conc u',\sigma$ an element of
	      $\bbot$..
       \end{enumerate}
  \item[($\forall$E, \textminus)]
       The derivation for $t$ ends in
       \aseq{\G}{\tj t{\forall X\phi}}
       \useq{\G}{\tj t{\phi[\psi/X]}}
       \DisplayProof.
       Both statements follow immediately from the induction hypotheses
       because $\nsem{\phi[\psi/X]}$ is a subset of $\nsem{\forall
       X\phi}$.
   \item[(Other cases)]
	We can swap $t$ and $u$ to find a symmetric case above.
 \end{description}
 \end{proof}
Note that \ref{c:first} uses \ref{c:second} in the (Com,~\textminus)
case
and \ref{c:second} uses \ref{c:first} in the ($\limp$E,~\textminus).

\begin{proposition}
 \label{prop:exec-on-pole}
 The following set is a pole: the set of executables that reduces to
 an executable containing an element of a fixed set of executables.
\end{proposition}

\begin{proposition}
 \label{prop:spec}
 Let
 $\sequent{}{}{\tj c
 {\forall X\forall A\forall B
 [((A\limp B)\limp X)
  \limp((B\limp A)\limp X)
  \limp X]}}$
 be
 derivable.
 For all terms $\rho,\sigma, r$ and for all stacks $\pi, \pi_A$ and
 $\pi_B$,
 if $[(\rho)d  ,\pi]\red [d,a\cdot \pi_B]$ and
    $[(\sigma)f,\pi]\red [f,b\cdot \pi_A]$ hold for all $d$ and $f$,
 then
 $[(c)(\rho)\sigma,\pi],S_\epsilon$ reduces to a multiset containing an
 element of
 $\{(a,\pi_A),(b,\pi_B)\}$.
\end{proposition}
\begin{proof}
 We denote by $\bbot$ the set of executables that reduce to a multiset
 containing an element of $\{(a,\pi_A), (b,\pi_B)\}$.
 By Prop.~\ref{prop:exec-on-pole},
 this is a pole.
 Take $\nsem{A}_0 =\{(\pi_A,\emptyset)\}$, $\nsem{B}_0 =
 \{(\pi_B,\emptyset)\}$ and $\nsem{X}_0=\{(\pi,\emptyset)\}$.
 We have $a\in\sem{A}$ and $b\in\sem{B}$.
 We claim $(\rho,\emptyset)\in\sem{(A\limp B)\limp X}$.
 Take any $(d,e)\in\sem{A\limp B}$.
 By assumption and (cong),
 $[(\rho)d,\pi\conc e]$ reduces to $[d,a\cdot\pi_B\conc e]$,
 where the reduct is in $\bbot$ because $(d,e)$ is in $\sem{A\limp B}$.
 Since $\bbot$ is closed for $\rev\red$,
 $[(\rho)d,\pi\conc e]$ is also in $\bbot$, making
 $(\rho,\emptyset)$ a member of $\sem{(A\limp B)\limp X}$.
 Likewise,
 $(\sigma,\emptyset)$ is a member of $\sem{(B\limp A)\limp X}$.
\end{proof}

% not really useful
% \begin{proposition}
%  The set of executables which do not reduce to $\emptyset$ forms a pole.
% \end{proposition}
% \begin{proof}
%  This is a special case of Prop.~\ref{prop:exec-on-pole}
% \end{proof}

Another important property of an executable is
non-abortfullness
which states that the executable has no reduction sequence leading to
$\emptyset,S$.
The name comes from hyper-lambda calculus formulation in~\citep{hiraiflops2012},
where a term sometimes turns into $\mathsf{abort}$ and there the property
stated that a hyper-term has no reduction sequence leading to a hyper-term
full of $\mathsf{abort}'s$.
This property is necessary for terms to denote
constructive content of proofs.

\begin{proposition}[Non-abortfullness]
 Let $\tr\tj t\psi$ be derivable, then,
 $[t,\pi]$ does not reduce to $\emptyset$ for
 any $\pi$.
\end{proposition}
\begin{proof}
 By definition of $\red$.
\end{proof}

\subsection{Adding Exponentials}
\fix{add if resources allow}

\section{Asynchronous Semantics}
\label{sec:async}

A \textit{term}~$t$ is defined by BNF:
\[
 t::= x
 \mid (t)t
 \mid t\conc t
 \mid \lambda x.t
 \mid \ast_t
 \mid \comod c c
 \mid \reader  c
\]
where $x$ is a variable and $c$ is a channel.
A \textit{stack}~$\pi$ is defined by BNF:
\[
 \pi ::= \epsilon
 \mid t\cdot \pi
 \enspace.
\]
We write the set of terms as $\Lambda$ and stacks~$\Pi$.
A \textit{store} is a partial mapping from locations to
terms, equivalently, a mapping from locations to terms and $\bot$'s.
The \textit{empty store}~$S_\epsilon$ maps any channel to $\bot$.
For a store $S$, we define a store $S[c\colon\phi\mapsto x]$ to be
the same as $S$ except that $S[c\colon\phi\mapsto x](c\colon\phi\mapsto
x)$ is $x$.
A \textit{store} is a partial mapping from locations to
terms, equivalently, a mapping from locations to terms and $\bot$'s.
The \textit{empty store}~$S_\epsilon$ maps any channel to $\bot$.
For a store $S$, we define a store $S[c\colon\phi\mapsto x]$ to be
the same as $S$ except that $S[c\colon\phi\mapsto x](c\colon\phi\mapsto
x)$ is $x$.
An \textit{executable} is a finite multiset on $\Lambda \times \Pi$,
paired with a store.

% 2a' reduction
We have reduction relation~$\red$ on executables,
which is defined to be the smallest binary preorder
that satisfies:
\begin{description}
 \item[(cong)] if
	    $[t,\pi],         S \red [ t',\pi'],        S'$
	    then
	    $[t,\pi \conc e], S \red [ t',\pi'\conc e], S'$\enspace;
 \item[(push)]
	    $[(t)u,\pi],S       \red [t,u\cdot\pi],S$;
 \item[(dist)]
	    $[t\conc u,\pi],S   \red [t,\pi\conc u,\pi],S$\enspace;
 \item[(store)]
	    $[\lambda x.t,u\cdot\pi],S
	     \red
	     [t[\ast_u/x],      \pi],S$\enspace;
 \item[(load)]
	    $[\ast_u,\pi],S\red[u,\pi],S$\enspace;
 \item[(write)]
	    $
	    [\comod c{\co c}, t\cdot\pi], S[\co
	    c\colon\psi\mapsto\bot]
	    \red
	    [\reader c, \pi],
	    S[\co c\colon\psi\mapsto t]
	    $\enspace;
 \item[(write')]
	    $
	    [\comod c{\co c}, t\cdot\pi], S[\co
	    c\colon\psi\mapsto u]
	    \red
	    [\reader c, \pi],
	    S[\co c\colon\psi\mapsto u]
	    $\enspace;
 \item[(read)]$
	    [\reader c ,\pi],
	    S[c\colon\phi\mapsto u]
	    \red
	    [u,\pi],
	    S[c\colon\phi\mapsto u]
	    $\enspace; and
 \item[(fail)]
	    $
	    [\reader c,\pi],
	    S[c\colon\phi\mapsto \bot]
	    \red
	    \emptyset,S[c\colon\phi\mapsto \bot]
	    $
	    \enspace.
\end{description}
Differently from the synchronous case in Sect.~\ref{sec:sync},
we do not use an external schedule relation on channels.
Instead, nondeterminism appears spontaneously.
We can implement something similar to Lafont's example~\fix{cite},
showing that confluence does not hold.
Suppose $\G\tr\tj t\phi$ and $\G\tr\tj u\phi$ are both derivable.
Then, by induction on the derivation,
both $\G,\tj x{\phi\limp\phi}\tr\tj{(x)t}\phi$
and $\G,\tj y{\phi\limp\phi}\tr\tj{(y)u}\phi$ are derivable
for $x$ and $y$ not appearing in $\G$, $u$, or $t$.
By the communication rule,
$\G\tr\tj{(\comod c{\co c})t\conc(\comod {\co c} c)u}\phi$ is derivable.
But $[(\comod c{\co c})t\conc (\comod{\co c} c)u,
\pi],S_\epsilon$
can reduce both to $[t,\pi], S_\epsilon[c\mapsto u, \co c\mapsto t]$
and to $[u,\pi], S_\epsilon[c\mapsto u,\co c\mapsto t]$.
Since $t$ and $u$ were taken arbitrarily, we cannot distinguish terms
of the same type if we equate a term with what it reduces to.

\begin{proposition}
 If $e\red e'$ in the synchronous case then $e, S_\epsilon \red e', S'$
 in the asynchronous case for some store~$S'$.
\end{proposition}
\fix{prove}

%%%%%%%% some texts

\section{Comparison with the Hypersequent Formulation}


\section*{where to put them}

``Undecidability of second order linear logic'' has
\begin{align*}
 \phi\otimes \psi&\equiv \forall X((\phi\limp\psi\limp X)\limp X) \\
 {1}      &\equiv \forall X(X\limp X)\\
 \exists X\phi   &\equiv \forall Y((\forall X(\phi\limp Y))\limp Y)
\end{align*}

Not all disjunctive tautologies are valid in G\"odel--Dummett logic:
$(P\limp Q)\lor(Q\limp R)$.

Non-abortfullness is not a pole.
In order to change this, we have to do something more subtle.
