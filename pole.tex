\chapter{Prelinearity Axiom as an Asynchronous Communication Scheme}
\label{ch:pole}

\section{Summary}
%%% abstract
\citet{danos-krivine} state that
disjunctive tautologies\footnote{A \textit{tautology}\index{tautology}
is a theorem of classical propositional logic.}
such as the excluded middle $\phi\vee(\phi\imp\bot)$
and the symmetric excluded middle $(\phi\imp\psi)\vee(\psi\imp\phi)$
represent synchronization schemes.
Indeed in their formalism, reduction of one term can wait
for another term, thus making synchronization.
However,
one can find no communication among different
concurrent elements of an executable.
In this chapter, we show that logical axioms can specify how
concurrent processes exchange information asynchronously.

We investigate the computational behavior of the prelinearity axiom
$(\phi\limp\psi)\oplus(\psi\limp\phi)$ and left weakening
on top of IMALL2, which is the fragment of second order
intuitionistic linear logic with connectives $\{\forall, \limp,
\oplus\}$.  In other words, our type system is based on the second-order
monoidal t-norm logic\index{logic!monoidal t-norm}
(MTL\index{MTL|see{logic, monoidal t-norm}})~\citep{Esteva2001271}.
We interpret the disjunction~$\oplus$ in the prelinearity axiom
nondeterministically: whether
one process can give information to the other or
vice versa.
In Section~\ref{sec:async}, where processes do not make synchronization,
nondeterminism appears automatically as a slower process cannot pass
information to a faster one.
In Section~\ref{sec:sync}, where processes make synchronization,
we externally specify the nondeterministic choice.
We adapt \citet{danos-krivine}'s realizability argument to our
synchronous case and give simulation from the
synchronous case to the asynchronous case.

The asynchronous semantics here is closely related to
the hyper-lambda calculus for G\"odel-Dummett logic presented in
\citet{hirailpar}.
On top of intuitionistic logic,
G\"odel-Dummett logic is axiomatized by Dummett axiom
$(\phi\imp\psi)\lor(\psi\imp\phi)$, which is similar to the prelinearity axiom.

Our aim in this chapter is to reason about asynchronous communication on
shared memory using parametricity argument.
However, since processes can write into and read from shared memory,
a straightforward approach would lead us to solving mixed-variant domain
equations or using the step-index technique~\citep{ahmed2006}.
Although dealing with general references and parametricity
is manageable~\citep{birkedal2009relational},
we choose a simpler, indirect approach.
In Section~\ref{sec:async} we deal with the
operational
semantics involving asynchronous communication using stores.
In Section~\ref{sec:sync} we apply the classical realizability argument to
an operational semantics where
communication is made
synchronously.
In Sections~\ref{sec:discuss} and \ref{sec:conclude}, we discuss and conclude.

%%% 2. terms types and _models_?
\section{Asynchronous Semantics}
\label{sec:async}

We apply Danos and Krivine's classical realizability
argument~\citep{danos-krivine} to a
lambda calculus involving
communicating processes.  Although the formulation is similar to
\citep{danos-krivine}, our formulation allows different processes to
pass around lambda terms.

In this section, we give an abstract machine involving
asynchronous shared memory communication, which is similar to the
hyper-lambda calculus for G\"odel-Dummett logic (\citet{hirailpar}).
We consider a programming language, which is a modification of
Danos and Krivine~\citep{danos-krivine}'s.

\subsection{Dynamics}
We assume a set~$\pvar$ of propositional variables whose cardinality is
countably infinite.
We also assume countably infinitely many locations with involution
$c\mapsto \co c$
satisfying $\co c\neq c$ and $\co{\co c} = c$.
Metavariable~$c$ runs over locations.
An \textit{a-term}\index{term!a-term} (asynchronous term)~$t$ is defined
by BNF:
\begin{align*}
 t::= \,&x
 \mid (t)t
 \mid \lambda x.t
 \mid \mat t x t x t
 \mid \inl t
 \mid \inr t
 \mid \\ &
 (t\conc t)
 \mid \comod c c
 \mid \reader c
 \mid \abort
\end{align*}
where $x$ is a variable and $c$ is a location.  The variable occurrences
except in the first clause are binding.
An \textit{a-stack}\index{stack!a-stack} (asynchronous stack)~$\pi$ is
defined by a BNF:
\[
 \pi ::= \epsilon
 \mid t\cdot \pi
 \mid \mats x t \pi x t \pi
 \enspace.
\]
We write the set of a-terms as $\aLambda$ and a-stacks~$\aPi$.
A \textit{store}\index{store} maps a location to an a-term, $\bot$ or $\top$.
The \textit{empty store}\index{store!empty}%
\index{empty store|see{store}}~$S_\epsilon$ maps any location to $\bot$.
For a store $S$, we define the updated store $S[c\mapsto x]$ to be
the same as $S$ except that $S[c\mapsto x](c)$ is $x$ whatever $S(c)$ is.
An \textit{a-executable}\index{executable!a-executable} (asynchronous executable)
is a finite multiset on $\aLambda \times \aPi$,
paired with a store.

% 2a' reduction
We define a binary relation~$\ared$ on a-executables to be the smallest binary preorders
that satisfy:
\begin{description}
 \item[(cong)] if
	    $[t,\pi],         S \ared [ t',\pi'],        S'$
	    then
	    $[t,\pi \conc e], S \ared [ t',\pi'\conc e], S'$\enspace;
 \item[(abort)] $[\abort,\pi] \ared \emptyset$\enspace;
 \item[(push)]
	    $[(t)u,\pi],S       \ared [t,u\cdot\pi],S$;
 \item[(store)]
	    $[\lambda x.t,u\cdot\pi],S
	     \ared
	     [t[u/x],      \pi],S$\enspace;
 \item[(ask)]
      $[\mat t x u y v,\pi], S\ared [t,\mats x u \pi y v \pi], S$\enspace;
 \item[(ansL)]
      $[\inl{v}, \mats xt\pi yu\sigma],S \ared [t[v/x],\pi],S $ \enspace;
 \item[(ansR)]
      $[\inr{w}, \mats xt\pi yu\sigma],S \ared [u[w/y],\sigma],S $ \enspace;
 \item[(write)]
	    $
	    [\comod c{\co c}, t\cdot\pi], S[\co c\mapsto\bot]
	    \ared
	    [\reader c, \pi],
	    S[\co c\mapsto t]
	    $\enspace;
 \item[(write')]
	    $
	    [\comod c{\co c}, t\cdot\pi], S[\co c\mapsto \top]
	    \ared
	    [\reader c, \pi],
	    S[\co c\mapsto \bot]
	    $\enspace;
 \item[(read)]$
	    [\reader c ,\pi],
	    S[c\mapsto u]
	    \ared
	    [u,\pi],
	    S[c\mapsto u]
	    $\enspace;
 \item[(fail)]
	    $
	    [\reader c,\pi],
	    S[c\mapsto \bot]
	    \ared
	    \emptyset,S[c\mapsto \top]
	    $
	    \enspace;
 \item[(local-global)] if $e_0, S_0\ared e_1, S_1$ then $e_0, S_0\ared e_1,
      S_1$\enspace; and
 \item[(dist)]
	    $[(t\conc u),\pi\conc e],S   \ared [t,\pi\conc u,\pi\conc
      e\conc e],S$\enspace.
\end{description}
Differently from the synchronous case in Section~\ref{sec:sync},
we do not use an external schedule relation on locations.
Instead, nondeterminism appears spontaneously.
Indeed, we can implement something similar to Lafont's
example~\citep[B.1]{girard1989proofs}.
Suppose $\tr\tj t\phi$ and $\tr\tj u\phi$ are both derivable.
Then,
$\tr\tj{(\comod c{\co c})t\conc(\comod {\co c} c)u}\phi$ is derivable.
From there $[((\comod c{\co c})t\conc (\comod{\co c} c)u),
\pi],S_\epsilon$
can reduce both to $[t,\pi], S_\epsilon$
and to $[u,\pi], S_\epsilon$ (Figure~\ref{fig:lafont}).
\begin{figure}
 \begin{subfigure}[b]{0.45\textwidth}
  \begin{align*}
   & [((\comodL)t\conc (\comodR) u),\pi], S_\epsilon \\
   \ared & [(\comodL)t,\pi\conc (\comodR)u,\pi], S_\epsilon \\
   \ared & [\reader c, \pi\conc (\comodR)u,\pi], S_\epsilon[d\mapsto
   t] \\
   \ared & [(\comodR)u,\pi], S_\epsilon[c\mapsto \top, \co c\mapsto t] \\
   \ared & [\reader {\co c}, \pi], S_\epsilon[c\mapsto \bot, \co c\mapsto t] \\
   \ared & [t,\pi], S_\epsilon[c\mapsto \bot,\co c\mapsto \bot]\enspace.
  \end{align*}
 \end{subfigure}
 \begin{subfigure}[b]{0.45\textwidth}
  \begin{align*}
   & [((\comodL)t\conc (\comodR) u),\pi], S_\epsilon \\
   \ared & [(\comodL)t,\pi\conc (\comodR)u,\pi], S_\epsilon \\
   \ared & [(\comodL)t,\pi\conc \reader {\co c},\pi], S_\epsilon[c\mapsto u] \\
   \ared & [(\comodL)t,\pi], S_\epsilon[c\mapsto u, \co c\mapsto \top] \\
   \ared & [\reader c,\pi], S_\epsilon[c\mapsto u,\co c\mapsto \bot] \\
   \ared & [u,\pi], S_\epsilon[c\mapsto \bot,\co c\mapsto \bot]\enspace.
  \end{align*}
 \end{subfigure}
 \caption[A non-confluent example similar to Lafont's example.]
 {A non-confluent example similar to Lafont's example~\citep[B.1]{girard1989proofs}.  Both sides start
 from the same configuration but reduce to different configurations.
 Moreover, since terms $t$ and $u$ can be taken arbitrarily, if we
 equate the terms related by $\ared$, we have to conclude that arbitrary
 two terms of the same type are equal.}
 \label{fig:lafont}
\end{figure}

\subsection{Statics}
For a set~$S$,
$\form(S)$ is the set of $S$-\textit{formulae}\index{formula!$S$}~$\phi$:
\[
\phi::= s \mid X \mid \phi\limp\phi\mid \phi\oplus\phi\mid
\forall X \phi
\]
where $s\in S$ and $X\in \pvar$. The $X$ in the
last clause is binding.
The connective $\forall$ connects stronger than $\oplus$, which is
stronger than $\limp$.
Repeated $\limp$'s,
$\phi_0\limp\phi_1\limp\cdots\limp\phi_n$, are defined inductively on~$n$
as
$\phi_0\limp(\phi_1\limp\cdots\limp\phi_n)$.
A \textit{type}\index{type} is an element of $\form(\emptyset)$.

A \textit{sequent}\index{sequent} $\sequent{\G}{\tj t\phi}$ consists of a
context~$\G$ and an s-term~$t$ associated with a type~$\phi$.
The \textit{context}\index{context} is a finite sequence of variables associated
with types where the same variable does not appear more than once.
When we write a concatenation of contexts $\G,\D$, we assume no variables
appear in both $\G$ and $\D$.
A \textit{hypersequent}\index{hypersequent} is a finite sequence of sequents.
In a hypersequent, we also assume that no variable appears more than once in
the contexts.

An $\NMTL$ \textit{derivation}\index{derivation!MTL2} is a tree composed of the derivation
rules in Figure~\ref{fig:pole-typing}
whose top rules are Ax.
A hypersequent is \textit{derivable}\index{derivable} in $\NMTL$ when there is an $\NMTL$
derivation ending at the hypersequent.
An s-term~$t$ has type~$\phi$ when $\tr\tj{t}{\phi}$ is derivable.
A type~$\phi$ is provable when there is an s-term~$t$ with type~$\phi$.
There is a well-known substructural logic called monoidal t-norm logic
(MTL), which validates the prelinearity axiom
$(\phi\limp\psi)\oplus(\psi\limp\phi)$.  Actually, $\NMTL$ characterizes
MTL.

\newcommand{\hyp}[1]{\mathcal{H}#1\hmid}
 \begin{figure}
 \centering
  \AxiomC{$\mathcal H$}
  \LL{EW}
  \useq{\hyp{}\G}{\tj\abort\phi}
  \DisplayProof
  \hfill
  \AxiomC{$\mathcal H\hmid \G\tr\tj{t}\phi \hmid \D\tr\tj{u}\psi
  \hmid \mathcal H'$}
  \LL{EE}
  \UnaryInfC{$\mathcal H\hmid \D\tr\tj{u}\psi\hmid \G\tr\tj{t}\phi
  \hmid \mathcal H'$}
  \DisplayProof
  \ruleskip
  \AxiomC{$\mathcal H\hmid\tr\tj{t}\phi \hmid\tr\tj{u}\phi$}
  \LL{EC}
  \UnaryInfC{$\mathcal H\hmid \tr\tj{(t\conc{u})}\phi $}
  \DisplayProof
  \ruleskip
\AxiomC{}
\LL  {Ax}
\useq{\xphi}{\xphi}
\DisplayProof
  \hfill
\aseq{\hyp{}\G,\tj{x}{\phi},\tj{y}{\psi},\D}{\tj{t}{\theta}}
\LL{IE}
\useq{\hyp{}\G,\tj{y}{\psi},\tj{x}{\phi},\D}{\tj{t}{\theta}}
\DisplayProof
  \hfill
  \aseq{\hyp{}\G}{\tj t \phi}
  \LL{IW}
  \useq{\hyp{}\tj{x}{\psi},\G}{\tj t\phi}
  \DisplayProof
\ruleskip
%
\aseq{\hyp{}\xphi,\G}{\tj t\psi}
\LL{$\limp$I}
\useq{\hyp{}\G}{\tj{\lambda x.t}{\phi\limp\psi}}
\DisplayProof
\hfill
%
\aseq{\hyp{}\G}{\tj t{\phi\limp\psi}}
\aseq{\hyp'\D}{\tj u\phi}
\LL   {$\limp$E}
\bseq{\hyp{}\hyp'\G,\D}{\tj{(t)u}\psi}
\DisplayProof
\ruleskip
%
\aseq{\hyp{}\G}{\tj{t}{\phi}}
\LL  {$\oplus$I}
\useq{\hyp{}\G}{\tj{\inl{t}}{\phi\oplus\psi}}
\DisplayProof
\hfill
%
\aseq{\hyp{}\G}{\tj{t}{\psi}}
\LL  {$\oplus$I}
\useq{\hyp{}\G}{\tj{\inr{t}}{\phi\oplus\psi}}
\DisplayProof
\ruleskip
%
\aseq{\hyp{}\G}{\tj{t}{\phi\oplus\psi}}
\aseq{\hyp'\D,\xphi}{\tj{u_0}{\theta}}
\aseq{\hyp'\D,\ypsi}{\tj{u_1}{\theta}}
\LL  {$\oplus$E}
\tseq{\hyp{}\hyp'\G,\D}{\tj{\mat{t}{x}{u_0}{y}{u_1}}{\theta}}
\DisplayProof
\ruleskip
%
\aseq{\hyp{}\G}{\tj t\phi}
\LL   {$\forall$I}
\useq{\hyp{}\G}{\tj t{\forall X\phi}}
\DisplayProof (no free $X$ in $\mathcal H$ or $\G$)
\hfill
%
\aseq{\hyp{}\G}{\tj{t}{\forall X\phi}}
\LL   {$\forall$E}
\useq{\hyp{}\G}{\tj{t}{\phi[\psi/X]}}
\DisplayProof
\ruleskip
%
\aseq{\hyp{}\tj{x}{\phi\limp\psi},\G}{\tj t\theta}
\aseq{\hyp'\tj{y}{\psi\limp\phi},\D}{\tj u\tau}
\LL{Com}
\bseq{\hyp{}\hyp'\G}
  {\tj{t[\comodL/x]}{\theta}\hmid
   \D\tr\tj{u[\comodR/y]}{\tau}}
\DisplayProof
\ruleskip
%
  \caption{Typing derivation rules of $\NMTL$.  $\mathcal H$ and
  $\mathcal H'$ stand for hypersequents. In EW, $\mathcal H$ cannot be empty.}
  \label{fig:pole-typing}
 \end{figure}
For example, there is an s-term typed with the prelinearity axiom (Figure~\ref{fig:prelin-term}).

 \begin{figure}
  \centering
  \AxiomC{}
  \LL{Ax}
  \useq{\tj{x}{X\limp Y}}{\tj{x}{X\limp Y}}
  \AxiomC{}
  \LL{Ax}
  \useq{\tj{y}{Y\limp X}}{\tj{y}{Y\limp X}}
  \LL{Com}
  \bseq{}{\tj{{\comodL}}{X\limp Y}
  \hmid\tr\tj{{\comodR}}{Y\limp X}}
  \LL{$\oplus$I}
  \useq{}{\tj{\inl{\comodL}}{(X\limp Y)\oplus(Y\limp X)}
  \hmid\tr\tj{\comodR}{Y\limp X}}
  \LL{$\oplus$I}
  \useq{}{\tj{\inl{\comodL}}{(X\limp Y)\oplus(Y\limp X)}
  \hmid\tr\tj{\inr{\comodR}}{(X\limp Y)\oplus(Y\limp X)}}
  \LL{EC}
  \useq{}{\tj{(\inl{\comodL}\conc\inr{\comodR})}{(X\limp Y)\oplus(Y\limp
  X)}}
  \LL{$\forall$I}
  \useq{}{\tj{(\inl{\comodL}\conc\inr{\comodR})}{
  \forall Y((X\limp Y)\oplus(Y\limp X))}}
  \LL{$\forall$I}
  \useq{}{\tj{(\inl{\comodL}\conc\inr{\comodR})}{
  \forall X\forall Y((X\limp Y)\oplus(Y\limp X))}}
  \DisplayProof
  \caption{A derivation tree typing an s-term with the prelinearity
  axiom: the s-term $(\inl{\comodL}\conc\inr{\comodR})$ has
  type $\forall X\forall Y((X\limp Y)\oplus(Y\limp X))$.}
  \label{fig:prelin-term}
 \end{figure}

%%%%%%% due to space limitation
% Note that Com rule introduces a restricted form of weakening on the left
% as in Fig.~\ref{fig:weakening}.
%  \begin{figure}
%   \centering
% \AxiomC{}
% \useq{\phi\limp\phi}{\phi\limp\phi}
% \aseq{\G}{\phi}
% \bseq{\phi\limp\phi,\G}{\phi}
% \AxiomC{}
% \useq{\phi\limp\phi}{\phi\limp\phi}
% \aseq{\D}{\phi}
% \bseq{\phi\limp\phi,\D}{\phi}
% \bseq{\G,\D}{\phi}
% \DisplayProof
%   \caption{Com rule introduces a restricted form of weakening on the
%   left even without using the weakening rule~(IW) explicitly.  }
%   \label{fig:weakening}
%  \end{figure}

\begin{corollary}[Prelinearity as a Communication Scheme]
 \label{spec:prelin-cor}
 Assume that an a-term~$g$ have type $
		    {\forall X\forall
 Y((X\limp Y)\oplus (Y\limp X))}.
		   $
 Then, the s-executable
 $
 [g, \mats{z}{z}{x\cdot\pi_Y}{w}{w}{y\cdot\pi_X}]
 $
 reduces to a multiset containing an element of
 $\{(x,\pi_X), (y,\pi_Y)\}$.
\end{corollary}
\begin{proof}
  By Props.~\ref{spec:prelin} and \ref{simu} below.
\end{proof}

\section{Synchronous Semantics}
\label{sec:sync}

We consider a programming language, which is a modification of
Danos and Krivine~\citep{danos-krivine}'s.
We assume a set~$\pvar$ of propositional variables whose cardinality is
countably infinite.
We also assume countably infinitely many locations with involution
satisfying $\co c\neq c$ and $\co{\co c} = c$.
Metavariable~$c$ runs over locations.
An \textit{s-term}\index{term!s-term} (synchronous term)~$t$ is defined
by a BNF:
\begin{align*}
 t::= \,&x
 \mid (t)t
 \mid \lambda x.t
 \mid \mat t x t x t
 \mid \inl t
 \mid \inr t
 \mid \\ &
 (t\conc t)
 \mid \comod c c
 \mid \abort
\end{align*}
where $x$ is a variable and $c$ is a location.  The variable occurrences
except in the first clause are binding.
An \textit{s-stack}\index{stack!s-stack} (synchronous stack)~$\pi$ is
defined by a BNF:
\[
 \pi ::= \epsilon
 \mid t\cdot \pi
 \mid \mats x t \pi x t \pi
 \enspace.
\]
We write the set of s-terms as $\sLambda$ and s-stacks~$\sPi$.
For s-terms~$s,t$ and a variable~$x$, $s[t/x]$ denotes the result of
substitution of $t$ for free occurrences of $x$ in $s$.  When more than one
substitutions are concatenated, e.g. $s[t/x][t'/y]$, the substitutions
are applied at the same time not one after another.
When we use this
kind of simultaneous substitution, we always make sure that the same
variable $x$ does not appear more than once after the $/$'s,
e.g.~$s[t/x][t'/x]$ never happens.

A \textit{process} is an element of $\sLambda\times\sPi$.
An \textit{s-executable}\index{executable!s-executable} (synchronous executable)
is a multiset of processes.
We ruthlessly use $\conc$ both for delimiting elements in an
s-executable and for addition of multisets.
We denote the empty multiset by~$\emptyset$.

% 2a' reduction
A \textit{schedule}\index{schedule} is a total preorder on locations.
Given a fixed schedule~$\sche$,
we define binary relations $\sred'$ and $\sred$
on s-executables
to be the smallest preorders
that satisfy:
\begin{description}
 \item[(cong)] if
      $e_0         \sred' e_1$
      then
      $e_0 \conc e \sred' e_1\conc e$\enspace;
 \item[(abort)] $[\abort,\pi]\sred' \emptyset$ \enspace;
 \item[(push)]
	    $[(t)u,\pi]      \sred' [t,u\cdot\pi]$      \enspace;
 \item[(store)]
	    $[\lambda x.t,u\cdot\pi]
	     \sred'
	     [t[u/x],      \pi]$\enspace;
 \item[(ask)]
      $[\mat t x u y v,\pi]\sred' [t,\mats x u \pi y v \pi]$\enspace;
 \item[(ansL)]
           $[\inl{v}, \mats xt\pi yu\sigma] \sred' [t[v/x],\pi] $ \enspace;
 \item[(ansR)]
           $[\inr{w}, \mats xt\pi yu\sigma] \sred' [u[w/y],\sigma] $ \enspace;
 \item[(com0)]
           $[\comod c{\co c}, t\cdot\pi\conc \comod{\co c}c,
           u\cdot\sigma] \sred'
           [u,\pi]$\enspace(if $\co c\sche c$ but $c\not\sche \co c$)\enspace;
 \item[(com1)]
	    $[\comod c{\co c}, t\cdot\pi\conc \comod{\co c}c,
	    u\cdot\sigma] \sred'
	    [t,\sigma]$\enspace(if $c\sche \co c$ but $\co c\not\sche c$)\enspace;
 \item[(com2)]
	    $[\comod c{\co c}, t\cdot\pi\conc \comod{\co c}c,
	    u\cdot\sigma] \sred'
	    [u,\pi\conc t,\sigma]$\enspace(if $c\sche \co c$ and $\co c\sche
       c$)\enspace;
 \item[(local-global)] If $e_0\sred'e_1$ then $e_0\sred e_1$\enspace; and
 \item[(dist)]
           $[(t\conc u),\pi\conc e]  \sred [(t,\pi)\conc e\conc (u,\pi)\conc
      e]$ \enspace.
\end{description}
We say $e$ reduces to $e'$ when $e\sred e'$ holds.
Below, we sometimes omit the outermost parentheses (i.e. [ and ]) for multisets.
Rules (cong), (push) and (store) come from
\citet{danos-krivine}.  (dist) also appears there but we changed it so
that the proof for adequacy (Theorem~\ref{c:adequacy}) goes through in the
case (EC, \textminus).

% I want to write these rules are non-blocking here.
% but, actually, Danos and Krivines rules are also non-blocking.
% they just busy-wait.
% In order to remove busy-waiting, we need types.

% 2b type system -- remove pierce -- add com'


% 2C truth values and models. make them disjunctive!
%    so that the empty set does not count as successful.
%    Also, make it closed for ``for all''
\begin{definition}
 \label{def:pole}
A \textit{pole}\index{pole}~$\bbot$ is a set of s-executables
which satisfies
\begin{enumerate}
 \item \label{red-closed} $e$ is in $\bbot$ if $e\sred e'$ and
       $e'\in\bbot$; and
 \item \label{conc-closed} $e\conc f$ is in $\bbot$ if $e$ or $f$ is in $\bbot$.
\end{enumerate}
\end{definition}
In the definition of poles,
condition~\ref{conc-closed}.~is different from that of Danos and Krivine's~\citep{danos-krivine}.
There, the condition says if $e$ \textit{and} $f$ \textit{are} in
$\bbot$, then $e\conc f$ is in $\bbot$.  Our disjunctive choice here is influenced
by hypersequents~\citep{avron91} and hyper-lambda
calculi~(Chapter~\ref{ch:lambda}), where
components are interpreted disjunctively.
Computationally,
we only guarantee that at least one process of an s-executable
$[t_0,\pi_0\conc \cdots \conc t_n,\pi_n]$ succeeds.

An \textit{environment}\index{environment} is a pair of an s-stack and
an s-executable.
The set of environments is written as~$E$.
A \textit{program}\index{program} is a pair of an s-term and an s-executable.
For a set~$\mathcal Z$ of environments, $\mathcal Z\rightarrow\bbot$ denotes
the set of programs $(t,e)$ such that
for any environment $(\pi,e')\in \mathcal Z$,
the s-executable $[t,\pi\conc e \conc e']$ is in $\bbot$.
We use programs and environments because if we continued using
s-terms and s-stacks, the proof of adequacy (Theorem~\ref{c:adequacy})
would fail
in the case
($\limp$E,~\textminus),~\ref{c:second}.

For $\phi\in\form(2^E)$ and $|\cdot|^-_0\colon\pvar\rightarrow 2^E$\kern
-2pt,
we define $\nsem{\phi}\in
2^E$ inductively on $\phi$:
\begin{align*}
 \nsem{\mathcal Z} =& \mathcal Z \text{ for } \mathcal Z\in 2^E\\
 \nsem{X}=& |X|_0^- \\
 \nsem{\phi\limp\psi}=&
 \{(t\cdot\pi, e_0\conc e_1)\mid
 (t,e_0)\in\nsem\phi\rightarrow\bbot \text{ and }(\pi,e_1)\in\nsem\psi\}\\
\nsem{\phi\oplus\psi}=& \{(\mats x{t}{\pi}y{u}{\sigma}, f)\mid\\ &
\,\,\,  t[v/x],\pi\conc f\conc f'\in\bbot \text{ for all } (v,f')\in\nsem{\phi}\rightarrow\bbot\text{
 and }\\ &
\,\,\, u[w/y],\sigma\conc f\conc f'\in\bbot \text{ for all } (w,f')\in\nsem{\psi}\rightarrow\bbot\}
 \\
 \nsem{\forall X\phi}=&
 \bigcup_{\mathcal Z\in 2^\sPi} \nsem{\phi[\mathcal Z/X]}\enspace.
\end{align*}
Using this, we define $\sem \phi=\nsem{\phi}\rightarrow\bbot$.
We have an equality
$\nsem{\phi\limp\psi} = \{(t\cdot\pi, e_0\conc e_1)\mid
(t,e_0)\in\sem\phi\text{ and }(\pi,e_1)\in\nsem\psi\}$.
Moreover, for types~$\phi$ and $\psi$, we define $\sempair{\phi,\psi}$
as the set of triples $(t,u,e)$ of s-terms $t$ and $u$ and an s-executable~$e$
such
that
$[t,\pi\conc u,\sigma\conc e\conc e_0\conc e_1]
\in\bbot$ for any $(\pi,e_0)\in\nsem\phi$
and $(\sigma,e_1)\in\nsem\psi$.

\begin{proposition}
 \label{comod-type}
 $(\comodL,\comodR,\emptyset)\in\sempair{\phi\limp\psi,\psi\limp\phi}$
 for any types~$\phi$ and $\psi$.
\end{proposition}
\begin{proof}
 Take any $(t\cdot\sigma,e_{0}\conc e_{1})\in\nsem{\phi\limp\psi}$
 and $(u\cdot\pi,f_{0}\conc f_{1})\in\nsem{\psi\limp\phi}$ such that
 $(t, e_{0})\in\sem{\phi}$, $(\sigma,e_{1})\in\nsem{\psi}$,
 $(u, f_{0})\in\sem{\psi}$ and $(\pi,f_{1})\in\nsem{\phi}$ hold.
 We claim that $e=[\comodL,t\cdot\sigma\conc\comodR,u\cdot\pi\conc
 e_{0}\conc e_{1}\conc f_{0}\conc f_{1}]$ is in
 $\bbot$.
 Depending on the schedule, $e$ might reduce to
 $ [t,\pi\conc e_{0}\conc e_{1}\conc f_{0}\conc f_{1}]$,
 $ [u,\sigma\conc e_{0}\conc e_{1}\conc f_{0}\conc f_{1}]$ or
 $ [t,\pi\conc u,\sigma\conc e_{0}\conc e_{1}\conc f_{0}\conc f_{1}]$,
 all of which are in $\bbot$ by condition~\ref{conc-closed}.~of
 Definition~\ref{def:pole} because
 $[t,\pi\conc e_{0}\conc f_{1}]$ and
 $[u,\sigma\conc e_{1}\conc f_{0}]$ are in $\bbot$.
 Since $\bbot$ is closed for $\sred^{-1}$,
 we have $e\in\bbot$.
\end{proof}

% 2d the adequacy lemma
For $\G = \tj{x_1}{\phi_1},\ldots,\tj{x_n}{\phi_n}$,
we denote by $\sem{\G}$ the set of pairs $(\vec t,e)$ where
$\vec{t} = (t_1,\dots,t_n)$, $e = \bigparallel_{1\le i\le n} e_i$
 and each pair~$(t_i, e_i)$ is in $\sem{\phi_i}$.
For that $\vec t$, $[\vec{t}/\G]$ denotes the simultaneous substitution
$[t_i/x_i]_{0\le i \le n}$.
For a hypersequent, we
define a set of s-executables
$
\semo{\G_0\tr\tj{t_0}{\phi_0}\hmid\cdots\hmid\G_n\tr\tj{t_n}{\phi_n}}
$
to be the set of executables of the form $\bigparallel_{0\le i\le
n}\left(t_i[\vec{g_i}/\G_i],\pi_i\conc e_i\conc f_i\right)$ where
$(\vec{g_i},e_i) \in \sem{\G_i}$ and $(\pi_i,f_i)\in \nsem{\phi_i}$.

Here we state adequacy.  What we will use later is statement~\ref{c:first}.
However, when we try to prove \ref{c:first} by induction on derivations,
the case for Com rule cannot be proved due to insufficient induction
hypotheses.
Thus we deal with two derivations at the
same time.

\begin{theorem}[Adequacy of NMTL2]
 \label{c:adequacy}
 Let hypersequents $\mathcal H$ and $\mathcal I$ be derivable.
 We state:
 \begin{enumerate}[label=(\arabic{*}), ref=\textit{(\arabic{*})}]
  \item \label{c:first} any s-executable in
	$
	\semo{\mathcal H}
	$
	is also in $\bbot$; and
  \item \label{c:second}
	when $\mathcal H$ and $\mathcal I$ are respectively
	equal to $\hat{\mathcal H}\hmid\tj x\theta, \hat\G\tr \tj{t}\phi$
	and $\hat{\mathcal I}\hmid \tj y\tau, \hat\D\tr\tj u\psi$ up to
	exchange,
	the following triple is in $\sempair{\phi,\psi}$:
	\[\left(
	t[v/x][\vec g/\hat \G],\quad
	u[w/y][\vec d/\hat \D],\quad
	 e \conc  f \conc e' \conc e_{\hat{\mathcal H}} \conc
	e_{\hat{\mathcal I}}
	\right)
	\]
	given
	$\eH \in \semo{\hatma H}$,
	$\eI \in \semo{\hatma I}$,
	$(v,w,e')\in\sempair{\theta,\tau}$,
	$(\vec g, e)\in\sem{\hat\G}$ and
	$(\vec d, f)\in\sem{\hat\D}$.
  \end{enumerate}
\end{theorem}
\begin{proof}
 We prove both statements at the same time by induction on the sum of
 the heights of the derivations of $\mathcal H$ and $\mathcal I$.  Here
 we identify hypersequents up to exchange.
  \begin{description}
  \item[(Ax, Ax)] When both derivations consist of only axiom rules,
       the statements follow from the definitions of $\sem{\phi}$ and
       $\sempair{\phi,\psi}$.
   \item[(IW, \textminus)]
	The derivation for $t$ ends as
	\[
	 \aseq{\hyp'\G'}{\tj{t}{\phi}}
	\LL{W}
	\useq{\hyp'\tj{\hat{x}}{\hat{\phi}}, \G'}{\tj{t}{\phi}}
	\DisplayProof\enspace.
	\]
	\begin{enumerate}[label=\textit{(\arabic{*})}]
	 \item Take any
	       $\takehyper{\mathcal{H'}}$,
	       $(\vec{g'}, e)\in\sem{\G'}$ and $(\hat t,\hat
	       e)\in\sem{\hat\phi}$.
	       Since $\hat x$ does not appear freely in $t$,
	       the term in question $t[\hat t/\hat x][\vec{g'}/\G']$ is
	       equal to $t[\vec{g'}/\G']$.
	       By the induction hypothesis~\ref{c:first},
	       $(t[\vec{g'}/\G'],e\conc e_{\mathcal H'})$ is in $\sem{\phi}$.
	       By condition~\ref{conc-closed}.~of Definition~\ref{def:pole},
	       the program in question $(t[\hat t/\hat x][\vec{g'}/\G'],
	       e\conc \hat e\conc e_{\mathcal H'})$ is also in $\sem{\phi}$.
	 \item
	       When $x$ and $\hat x$ are different, we can do the same
	      as for \ref{c:first}.
	      Otherwise, $x$ is equal to $\hat x$ and $\theta$ is equal
	      to $\hat\phi$.
	      Take any
	      $\takehyper{\hatma H}$,
	      $\takehyper{\hatma I}$,
	      $(v,w,e')\in\sempair{\theta,\tau}$,
	      $(\vec g,e)\in\sem{\hat\G}$,
	      $(\vec d,f)\in\sem{\hat\D}$,
	      $(\pi,e'')\in\nsem{\phi}$ and
	      $(\sigma,f'')\in\nsem{\psi}$.
	      We have to show that the following executable is in
	      $\bbot$:
	      \[
	       \bar e = t[v/\hat x][\vec{g}/\hat{\G}],\pi\conc
	      u[w/y][\vec d/\hat \D], \sigma\conc e\conc f\conc e'\conc
	      e''\conc f''\conc e_{\hatma H}\conc e_{\hatma I}\enspace.
	      \]
	      By the induction hypothesis~\ref{c:first},
	      $
	       [t[\vec g/\hat \G],\pi\conc e\conc e''\conc e_{\hatma H}]
	      $
	      is in $\bbot$.  Moreover, since $\hat x$ does not appear
	      in $t$,  $[t[v/\hat x][\vec g/\hat \G],\pi\conc e\conc
	      e''\conc e_{\hatma H}]$
	      is in $\bbot$.  By condition~\ref{conc-closed}.~of
	      Definition~\ref{def:pole},
	      $\bar e$ is also in $\bbot$.
	\end{enumerate}
   \item[(EW, \textminus)]
	\begin{enumerate}[label=\textit{(\arabic{*})}]
	 \item  By (abort) reduction.
	 \item Also by (abort) reduction, but when $x$ appears in the
	       context for $\abort$, we have to do the same argument as
	       in (W, \textminus), \ref{c:second}.
	\end{enumerate}
   \item[(EC, \textminus)]
	The derivation for $t$ ends as
	\[
	 \aseq{\hyp'}{\tj{t_0}{\phi}\hmid \tr\tj{t_1}{\phi}}
	\useq{\hyp'}{\tj{(t_0\conc t_1)}{\phi}}
	\DisplayProof\enspace.
	\]
	\begin{enumerate}[label=\textit{(\arabic{*})}]
	 \item Take any $e_{\mathcal H'}\in \semo{\mathcal H'}$ and
	       $(\pi,f)\in\nsem{\phi}$.
	       We have to show that this executable is in $\bbot$:
	       \[
	       [e_{\mathcal H'}\conc (t_0\conc t_1), \pi\conc f]\enspace.
	       \]
	       By (dist), this executable reduces to
	       \[
	       [e_{\mathcal H'}\conc e_{\mathcal H'}\conc
	       t_0,\pi \conc t_1, \pi\conc f\conc f]\enspace,
	       \]
	       which is in $\bbot$ because, by the induction hypothesis, the
	       following executable is in $\bbot$:
	       \[
	       [e_{\mathcal H'}\conc
	       (t_0\conc t_1), \pi\conc f\conc f]\enspace.
	       \]
	 \item Similar to \ref{c:first}.
	\end{enumerate}
   \item[($\limp$I, \textminus)]
	The derivation for $t$ ends as
	\[
	\aseq{\hyp'\tj{\hat x}{\phi_0},\G}{\tj{t_1}{\phi_1}}
	\useq{\hyp'\G}{\tj{\lambda{\hat{x}}.t_1}{\phi_0\limp\phi_1}}
	\DisplayProof\enspace.
	\]
	\begin{enumerate}[label=\textit{(\arabic{*})}]
	 \item Take any
	       $\takehyper{\mathcal H'}$,
	       $(\vec g, e)\in\sem{\G}$
	       and
	       $(t_0\cdot\pi,e_0\conc e_1)\in\nsem{\phi_0\limp\phi_1}$
	       so that $(t_0,e_0)\in\sem{\phi_0}$ and
	       $(\pi,e_1)\in\nsem{\phi_1}$ hold.
	       We have to show that the following s-executable is in
	       $\bbot$:
	       \[
	       (\lambda\hat x.t_1)[\vec g/\G],
	       t_0\cdot\pi\conc e\conc e_0\conc e_1\conc e_{\mathcal
	       H'}\enspace.
	       \]
	       By (store) and (cong), the s-executable reduces to
	       \[
		t_1[{t_0}/\hat x][\vec{g}/\G],\pi\conc{e}\conc
	       e_0\conc{e_1}\conc e_{\mathcal H'}\enspace.
	       \]
	       We have $({t_0}, e_0)\in\sem{\phi_0}$ so, by
	       the induction hypothesis~\ref{c:first},
	       the reduct is in $\bbot$.
	       Since $\bbot$ is closed for $\rev\sred$, the original
	       s-executable is in $\bbot$, too.
	 \item Take any
	       $\takehyper{\hatma H}$,
	       $\takehyper{\hatma I}$,
	       $(v,w,e')\in\sempair{\theta,\tau}$,
	       $(\vec g, e)\in\sem{\hat\G}$,
	       $(\vec d, f)\in\sem{\hat\D}$,
	       $(t_0\cdot\pi,e_0'\conc e_1')\in\nsem{\phi_0\limp\phi_1}$ and
	       $(\sigma,f'')\in\nsem\psi$ so that
	       $(t_0,e_0')\in\sem{\phi_0}$ and
	       $(\pi,e_1')\in\nsem{\phi_1}$.
	       We have to show that this s-executable is in $\bbot$:
	       \[
	       (\lambda \hat x. t_1)[v/x][\vec{g}/\G],t_0\pi\conc
	       u[w/y][\vec{d}/\D],\sigma\conc
	       e'\conc {e}\conc {f} \conc{e_0'}\conc{e_1'}\conc
	       e_{\hatma H}\conc e_{\hatma I}\enspace.
	       \]
	       By (store) and (cong), the s-executable reduces to
	       \[
		t_1[v/x][\vec{g}/\G][{t_0}/\hat x],\pi\conc
	       u[w/y][\vec{d}/\D],\sigma\conc
	       e'\conc {e}\conc {f} \conc{e_0'}\conc{e_1'}\conc
	       e_{\hatma H}\conc e_{\hatma I}\enspace.
	       \]
	       Since $({t_0},e_0')$ is in $\sem{\phi_0}$,
	       by the induction hypothesis~\ref{c:second}, the reduct is in
	       $\bbot$.  Since $\bbot$ is closed for $\rev\sred$,
	       the original s-executable is also in $\bbot$.
	\end{enumerate}
   \item[($\limp$E, \textminus)]
	The derivation for $t$ ends as
	\[
	\aseq{\hyp{_0}\G_0}{\tj{t_0}{\phi'\limp\phi}}
	\aseq{\hyp{_1}\G_1}{\tj{t_1}{\phi'}}
	\bseq{\hyp{_0}\hyp{_1}\G_0,\G_1}{\tj{(t_0)t_1}{\phi}}
	\DisplayProof\enspace.
	\]
	\begin{enumerate}[label=\textit{(\arabic{*})}]
	 \item Take any
	       $\eHz\in\semo{\mathcal H_0}$,
	       $\eHo\in\semo{\mathcal H_1}$,
	       $(\vec{g_0},{e_0})\in\sem{\G_0}$,
	       $(\vec{g_1},{e_1})\in\sem{\G_1}$ and
	       $(\pi,e'')\in\nsem{\phi}$.
	       We have to show that this s-executable is in $\bbot$:
	       \[
		((t_0)t_1)[\vec{g_0}/\G_0][\vec{g_1}/\G_1],\pi\conc
	       {e_0}\conc {e_1}\conc e''\conc \eHz\conc \eHo.
	       \]
	       By (push) and (cong), this s-executable reduces to
	       \[
		t_0[\vec{g_0}/\G_0], t_1[\vec{g_1}/\G_1]\cdot \pi
	       \conc {e_0}\conc {e_1}\conc e''\conc \eHz\conc \eHo.
	       \]
	       By the induction hypothesis~\ref{c:first} on both branches,
	       we have $(t_0[\vec{g_0}/\G_0],
	       {e_0\conc \eHz})\in\sem{\phi'\limp\phi}$
	       and
	       $(t_1[\vec{g_1}/\G_1],{e_1\conc \eHo})\in\sem{\phi'}$.
	       By the latter, we have $(t_1[\vec{g_1}/\G_1]\cdot\pi,
	       {e_1}\conc e'')\in\nsem{\phi'\limp\phi}$.
	       By definition of $\sem{\phi'\limp\phi}$, we have shown that the
	       reduct is in $\bbot$.
	       Since $\bbot$ is closed for $\sred^{-1}$,
	       the original s-executable is also in $\bbot$.
	 \item Variable~$x$ might be contained in $\G_0$, $\G_1$,
	       $\mathcal H_0$ or $\mathcal H_1$.
	       In the last two cases, the proof is similar to that of
	       \ref{c:first}.
	       \begin{description}
		\item[(case $\G_0 = \tj{x}{\theta},\hat\G_0$)]
		     Take any
		     $\eHz\in\semo{\mathcal H_0}$,
		     $\eHo\in\semo{\mathcal H_1}$,
		     $(v,w,e')\in\sempair{\theta,\tau}$,
		     $(\vec{g_0},{e_0})\in\sem{\hat\G_0}$,
		     $(\vec{g_1},{e_1})\in\sem{\G_1}$,
		     $(\vec{d},{f})\in\sem{\hat\D}$,
		     $(\pi,e'')\in\nsem{\phi}$ and
		     $(\sigma,f'')\in\nsem{\psi}$.
		     We have to show that the following s-executable is in
		     $\bbot$:
		     \[
		     ((t_0)t_1)[v/x][\vec{g_0}/\hat\G_0][\vec{g_1}/\G_1],\pi\conc
		     u' ,\sigma\conc \hat e\conc {{e_1}}\conc
		     {e''}\conc{f''}\conc \eHo\enspace.
		     \]
		     where
		     $u' = u[w/y][\vec{d}/\hat\D]$ and
		     $\hat e= e'\conc {e_0}\conc \eHz\conc {f}$.
		     By (push) and (cong), this s-executable reduces to
		     \[
		     t_0[v/x][\vec{g_0}/\hat\G_0],
		     t_1[\vec{g_1}/\G_1]\cdot\pi
		     \conc
		     u' ,\sigma\conc \hat e\conc {{e_1}}\conc
		     {e''}\conc{f''}\conc \eHo\enspace.
		     \]
		     By the induction hypothesis~\ref{c:first},
		     $(t_1[\vec{g_1}/\G_1],{e_1\conc \eHo})$ is in $\sem{\phi'}$.
		     In addition to this, $(\pi,e'')$ is in $\nsem{\phi}$,
		     making $(t_1[\vec{g_1}/\G_1]\cdot\pi, {e_1\conc \eHo}\conc e'')$ in
		     $\nsem{\phi'\limp\phi}$.
		     By the induction hypothesis~\ref{c:second},
		     $(t_0[v/x][\vec{g_0}/\hat\G_0], u', \hat e)$ is in
		     $\sempair{\phi'\limp\phi,\psi}$.
		     These make the reduct a member of $\bbot$.
		     Since $\bbot$ is closed for $\rev\sred$, the original
		     s-executable is also in $\bbot$.
		\item[(case $\G_1 = \tj{x}{\theta},\hat\G_1$)]
		     Take any
		     $\eHz\in\semo{\mathcal H_0}$,
		     $\eHo\in\semo{\mathcal H_1}$,
		     $(v,w,e')\in\sempair{\theta,\tau}$,
		     $(\vec{g_0},{e_0})\in\sem{\G_0}$,
		     $(\vec{g_1},{e_1})\in\sem{\hat\G_1}$,
		     $(\vec{d},{f})\in\sem{\hat\D}$,
		     $(\pi,e'')\in\nsem{\phi}$ and
		     $(\sigma,f'')\in\nsem{\psi}$.
		     We have to show that the following s-executable is in $\bbot$:
		     \[
		     ((t_0)t_1)[v/x][\vec{g_0}/\G_0][\vec{g_1}/\hat\G_1],\pi\conc
		     {e_0\conc \eHz}
		     \conc\hat{f}\conc
		     e''
		     \]
		     where
		     $\hat f = u[w/y][\vec{d}/\hat\D],\sigma\conc
		     {f}\conc f''\conc {e_1\conc \eHo}\conc e'$.
		     By (push) and (cong), this s-executable reduces
		     to
		     \[
		     t_0[\vec{g_0}/\G_0],t_1[v/x][\vec{g_1}/\hat\G_1]\cdot \pi\conc
		     {e_0\conc \eHz}
		     \conc\hat{f}\conc
		     e''\enspace.
		     \]
		     By the induction hypothesis~\ref{c:second},
		     $(t_1[v/x][\vec{g_1}/\hat\G_1],
		     u[w/y][\vec{d}/\hat\D], {e_1\conc \eHo}\conc {f}\conc
		     e')$ is in $\sempair{\phi',\psi}$.
		     So, since $(\sigma,f'')$ is in $\nsem{\psi}$,
		     $(t_1[v/x][\vec{g_1}/\hat\G_1],\hat f)$ is in
		     $\sem{\phi'}$.
		     Moreover we have $(\pi,e'')\in\nsem{\phi}$.
		     Combined,
		     $(t_1[v/x][\vec{g_1}/\hat\G_1]\cdot\pi,\hat f\conc e'')$
		     is in $\nsem{\phi'\limp\phi}$\kern -1pt.
		     By the induction hypothesis~\ref{c:first},
		     $(t_0[\vec{g_0}/\G_0],\hat e)$ is in
		     $\sem{\phi'\limp\phi}$.
		     So, the reduct is in $\bbot$, making the original
		     s-executable a member of~$\bbot$.
	       \end{description}
	\end{enumerate}
   \item[($\oplus$I, \textminus)]
       Without loss of generality, we assume that the
       derivation for $t$ ends as
       \[
       \aseq{\hyp{'}\G}{\tj{t'}{\phi_0}}
       \useq{\hyp{'}\G}{\tj{\inl{t'}}{\phi_0\oplus\phi_1}}
       \DisplayProof\enspace.
       \]
       \begin{enumerate}[label=\textit{(\arabic{*})}]
        \item Take any
	      $\takehyper{\mathcal{H}'}$,
	      $(\vec g, e)\in\sem{\G}$ and
              $(\mats {\hat x}v{\pi_0}{\hat y}w{\pi_1}, e'')$
              in $\nsem{\phi_0\oplus\phi_1}$.
              By the induction hypothesis~\ref{c:first},
              $(t'[\vec g/\G], e\conc e_{\mathcal H'})$ is in
	      $\sem{\phi_0}$.
              By definition of $\nsem{\phi_0\oplus\phi_1}$,
              the s-executable
              $[v[t'[\vec g/\G]/\hat x],\pi_0\conc{e\conc e_{\mathcal
	      H'}}\conc e'']$
              is in
              $\bbot$.
              Since $\bbot$ is closed for $\rev\sred$,
              the s-executable
	      \[
	      \inl{t'}[\vec g/\G], \mats {\hat x}v{\pi_0}{\hat
              y}w{\pi_1}
              \conc{e\conc e_{\mathcal H}}\conc e''
	      \]
              is also in $\bbot$.
              This shows the statement because we chose an arbitrary
              element of $\nsem{\phi_0\oplus\phi_1}$.
        \item
	     If $x$ is in $\mathcal H$, the same argument as
	      \ref{c:first} suffices. Otherwise,
	     take any
	      $\takehyper{\mathcal H}$
	      $(v,w,e')\in\sempair{\theta,\tau}$,
              $(\vec g, e)\in\sem{\hat\G}$,
              $(\vec d, f)\in\sem{\hat\D}$,
              \[
	       (\mats{\hat x}{\hat v}{\pi_0}{\hat y}w{\pi_1}, e'')\in
              \nsem{\phi_0\oplus\phi_1}
	      \] and
              $(\sigma,f'')\in\nsem{\psi}$.
              We have to show that the following s-executable is in
              $\bbot$:
              \[
               (\inl t)[v/x][\vec g/\hat\G], \mats{\hat x}{\hat v}{\pi_0}{\hat
              y}w{\pi_1}\conc
              u',\sigma\conc
              \hat e\conc e''\conc f''
              \]
              where
              $u' = u[w/y][\vec d/\hat\D]$ and $\hat e=e'\conc  e\conc
	      \conc e_{\mathcal H} \conc
              f$.
              By (ansL), the s-executable reduces to
              $\hat v[t'[v/x][\vec g/\hat\G]/\hat x],\pi_0 \conc
              u',\sigma\conc
              \hat e\conc e''\conc f''
              $.
              By the induction hypothesis~\ref{c:second},
              $(t'[v/x][\vec g/\hat \G], u', \hat e)$ is in
              $\sempair{\phi_0,\psi}$.
              So, $(t'[v/x][\vec g/\hat\G], [u',\sigma\conc \hat e\conc
              f''])$ is in $\sem{\phi_0}$.
              By definition of $\nsem{\phi_0\oplus\phi_1}$,
              the reduct is in $\bbot$.
              Since $\bbot$ is closed for $\rev\sred$,
              the original s-executable is also in $\bbot$.
	      % Similar to \ref{c:first}.
       \end{enumerate}
   \item[($\oplus$E, \textminus)]
       The derivation for $t$ ends in
       \[
       \aseq{\hyp{_0}\G_0}{\tj{t'}{\phi_0\oplus\phi_1}}
       \aseq{\hyp{_1}\G_1,\tj{\hat{x}}{\phi_0}}{\tj{t_0}{\phi}}
       \aseq{\hyp{_1}\G_1,\tj{\hat{y}}{\phi_1}}{\tj{t_1}{\phi}}
       \tseq{\hyp{_0}\hyp{_1}\G_0,\G_1}{\tj{\mat{t'}{\hat x}{t_0}{\hat
	y}{t_1}}{\phi}}
       \DisplayProof\enspace.
       \]
       \begin{enumerate}[label=\textit{(\arabic{*})}]
        \item Take any
	      $\eHz\in\semo{\mathcal H_0}$,
	      $\eHo\in\semo{\mathcal H_1}$,
	      $(\vec{g_0},{e_0})\in\sem{\G_0}$,\quad
              $(\vec{g_1},{e_1})\in\sem{\G_1}$
              and
              $(\pi,e')\in\nsem{\phi}$.
              We have to show that the s-executable
              \[
              (\mat {t'}{\hat{x}}{t_0}{\hat{y}}{t_1})[\vec{ g_0}/\G_0][\vec{g_1}/\G_1],
              \pi\conc {e_0\conc \eHz}\conc e'\conc {e_1\conc \eHo}
              \]
              is in $\bbot$.
              By (ask), the s-executable reduces to
              \[
               t'[\vec{g_0}/\G_0],
              \mats{\hat{x}}{t_0[\vec{g_1}/\G_1]}{\pi}
              {\hat{y}}{t_1[\vec{g_1}/\G_1]}{\pi}\conc{e_0\conc \eHz
              \conc e'}\conc
              {e_1\conc \eHo}\enspace.
              \]
              We claim that
              $(\mats{\hat{x}}{t_0[\vec{g_0/\G_0}]}{\pi}{\hat{y}}{t_1}{\pi},{e_1\conc
              \eHo}\conc
              e')$ is in $\nsem{\phi\oplus\psi}$ and that
              $(t'[\vec{g_0}/\G_0],{e_0\conc \eHz})$ is in $\sem{\phi\oplus\psi}$.  The first
              claim is shown by the induction hypothesis~\ref{c:first}
              stating that $(t_0[v/\hat{x}][\vec{g_1}/\G_1], {e_1\conc \eHo}\conc
              e'')$ is in $\sem{\phi}$ for any $(v,e'')\in\sem{\phi_0}$
              and similarly to~$t_1$.
              The second claim follows from induction
              hypothesis~\ref{c:first} on $t'$.
              By the two claims and by the definition of
              $\sem{\phi\oplus\psi}$,
              we have shown that the reduct is in $\bbot$ and
              thence that the original s-executable is in $\bbot$.
        \item Variable $x$ might be in $\G_0$, $\G_1$,
	       $\mathcal H_0$ or $\mathcal H_1$.
	       In the last two cases, the proof is similar to that of
	       \ref{c:first}.
              \begin{description}
               \item[(case $\G_0 = \tj x\theta,\hat\G_0$)]
                    Take any
		    $\eHz\in\semo{\mathcal H_0}$,
		    $\eHo\in\semo{\mathcal H_1}$,
                    $(v,w,e')\in\sempair{\theta,\tau}$,
                    $(\vec{g_0},{e_0})\in\sem{\hat\G_0}$,
                    $(\vec{g_1},{e_1})\in\sem{\G_1}$,
                    $(\vec d, f)\in\sem{\hat\D}$,
                    $(\pi,e'')\in\nsem{\phi}$ and
                    $(\sigma,f'')\in\nsem\psi$.
                    We have to show that the following s-executable is in
                    $\bbot$:
                    \begin{eqnarray*}
                    (\mat{t'}{\hat x}{t_0}{\hat
                    y}{t_1})[v/x][\vec{g_0}/\hat\G_0][\vec{g_1}/\G_1],\pi\conc\\
                    u[w/y][\vec d/\hat\D],\sigma\conc e'\conc
                    {e_0\conc \eHz}\conc {e_1\conc \eHo}\conc  f\conc e''\conc
                    f''.
                    \end{eqnarray*}
                    By (ask) and (cong), this s-executable reduces to
                    \begin{eqnarray*}
                    (t'[v/x][\vec{g_0}/\hat\G_0],
                     \mats{\hat x}{t_0[\vec{g_1}/\G_1]}{\pi}{\hat y}{t_1[\vec{g_1}/\G_1]}{\pi}
                     \conc\\
                    u[w/y][\vec d/\hat\D],\sigma\conc e'\conc
                    {e_0\conc \eHz}\conc {e_1\conc \eHo}\conc  f\conc e''\conc
                    f''\enspace.
                    \end{eqnarray*}
                    By the induction hypothesis~\ref{c:second},
                    we have
                    \[
                    (t'[v/x][\vec{g_0}/\hat\G_0], u[w/y][\vec d/\hat\D],
                    e'\conc {e_0\conc \eHz}\conc f) \in\sempair{\phi_0\oplus\phi_1,\psi}\enspace.
                    \]
                    Also, we defined $(\sigma,f'')$ to be an element of
                    $\nsem\psi$.
                    By definition of $\sempair{\phi_0\oplus\phi_1,\psi}$,
                    the reduct is in $\bbot$.  Since $\bbot$ is closed for
                    $\rev\sred$,
                    the original s-executable is also in $\bbot$.
               \item[(case $\G_1=\tj{x}{\theta},\hat\G_1$)]
                    Take any
		    $\eHz\in\semo{\mathcal H_0}$,
		    $\eHo\in\semo{\mathcal H_1}$,
                    $(v,w,e')\in\sempair{\theta,\tau}$,
                    $(\vec{g_0},{e_0})\in\sem{\G_0}$,
                    $(\vec{g_1},{e_1})\in\sem{\hat\G_1}$,
                    $(\vec{d},{f})\in\sem{\hat\D}$,
                    $(\pi,e'')\in\nsem{\phi}$ and
                    $(\sigma,f'')\in\nsem{\psi}$.
                    We have to show that the following s-executable is in
                    $\bbot$:
                    \begin{eqnarray*}
                     (\mat{t'}{\hat x}{t_0}{\hat
                      y}{t_1})[\vec{g_0}/\G_0][v/x][\vec{g_1}/\G_1],\pi\conc \\
                     u[w/y][\vec d/\hat\D],\sigma\conc e'\conc
                      {e_0\conc \eHz}\conc {e_1\conc \eHo}\conc f\conc e''\conc f''\enspace.
                    \end{eqnarray*}
                    By (ask) and (cong), this s-executable reduces to
                    \begin{align*}
                     e_{\mathrm r} =&
                      t'[\vec{g_0}/\G_0],
                      \mats{\hat x}
                      {t_0'}{\pi}{\hat y}
                      {t_1'}{\pi}\conc \\
                     &u[w/y][\vec d/\hat\D],\sigma\conc e'\conc
                      {e_0\conc \eHz}\conc {e_1\conc \eHo}\conc  f\conc e''\conc f''
                    \end{align*}
		    where $t_0' = t_0[v/x][\vec{g_1}/\G_1]$ and $t_1' =
		    t_1'[v/x][\vec{g_1}/\G_1]$.
                    By the induction hypothesis~\ref{c:second},
                    the triple
                    $(t_0'[v_0'/\hat x],
                    u[w/y][\vec d/\hat \D], e'\conc {e_1\conc \eHo}\conc
                    e_0'\conc  f)$ is in $\sempair{\phi,\psi}$ for any
                    $(v_0',e_0')\in\sem{\phi_0}$ so
                    that
                    $(t_0'[v_0'/\hat x],
                    [u[w/y][\vec d/\hat D],\sigma\conc e'\conc
                    {e_1\conc \eHo}\conc e_0'\conc {f}\conc f''])$ is in
                    $\sem\phi$.
                    We have a symmetric fact for $t_1$.
                    Thus, the environment
                    \begin{align*}
                    (&
                     \mats{\hat x}
                      {t_0'}{\pi}{\hat y}
                      {t_1'}{\pi},\\
                     & [u[w/y][\vec{d}/\hat\D],\sigma\conc e'\conc {e_1\conc \eHo}\conc
                      f\conc e''\conc f'']
                    )
                    \end{align*}
                    is in $\nsem{\phi_0\oplus\phi_1}$\kern -2pt.
                    By the induction hypothesis~\ref{c:first},
                    $(t'[\vec{g_0}/\G_0],{e_0\conc \eHz})$ is in
                    $\sem{\phi_0\oplus\phi_1}$.
                    By definition of $\sem{\phi_0\oplus\phi_1}$,
                    the reduct $e_{\mathrm r}$ is in $\bbot$.
                    Since $\bbot$ is closed for $\rev\sred$,
                    the original s-executable is also in $\bbot$.
              \end{description}
       \end{enumerate}
   \item[(Com, \textminus)]
	The derivation for $t$ ends in
	\[
	\aseq{\hyp{_0}\tj{\hat x}{\phi_0\limp\phi_1},\G_0}{\tj{t_0}{\phi}}
	\aseq{\hyp{_1}\tj{\hat y}{\phi_1\limp\phi_0},\G_1}{\tj{t_1}{\phi'}}
	\bseq{\hyp{_0}\hyp{_1}\G_0}{\tj{t_0[\comodL/\hat x]}{\phi}\hmid
	\G_1\tr \tj{t_1
	[\comodR/\hat y]}{\phi'}}
	\DisplayProof\enspace.
	\]
	\begin{enumerate}[label=\textit{(\arabic{*})}]
	\item Take any
	      $\eHz\in\semo{\mathcal H_0}$,
	      $\eHo\in\semo{\mathcal H_1}$,
	      $(\vec{g_0},{e_0})\in\sem{\G_0}$ and
	      $(\vec{g_1},{e_1})\in\sem{\G_1}$.
	      By Proposition~\ref{comod-type}, we have
	      $(\comodL,\comodR,\emptyset)\in\sempair{\phi_0\limp\phi_1,\phi_1\limp\phi_0}$.
	      By the induction hypothesis~\ref{c:second}, the s-terms
	      $t'_0=t_0[\comodL/\hat{x}][\vec{g_0}/\G_0]$ and
	      $t'_1=t_1[\comodR/\hat{y}][\vec{g_1}/\G_1]$
	      satisfy
	      $(t_0',t_1',{e_0\conc \eHz}\conc
	      {e_1\conc \eHo})\in\sempair{\phi,\phi'}$.
	\item
	     If $\tj{x}{\theta}$ is in $\mathcal H_0$ or $\mathcal H_1$,
	     we can apply the same argument as \ref{c:first}.
	     Otherwise,
	      without loss of generality,
	      we can assume
	      $\G_0 =\tj x\theta, {\hat\G_0}$ up to exchange.
	      Take any
	      $(v,w,e')\in\sempair{\theta,\tau}$,
	     $(\vec{g_0},{e_0})\in\sem{\hat\G_0}$,
	     $(\vec{g_1},{e_1})\in\sem{\G_1}$,
	     $(\vec d, f)\in\sem{\hat\D}$,
	     $(\pi,e'')\in\nsem{\phi}$,
	     $(\pi',e''')\in\nsem{\phi'}$
	     and
	     $(\sigma,f'')\in\nsem{\psi}$.
	     After defining $t_0' = t_0[\comodL/\hat
	     x][\vec{g_0}/\hat\G_0][v/x]$,
	     $t'_1 = t_1[\comodR/\hat y][\vec{g_1}/\G_1]$ and $u' =
	     u[w/y][\vec{d}/\hat\D]$,
	     we have to show that this s-executable is in $\bbot$:
	     \[
	      t'_0,\pi\conc t'_1,\pi'\conc u',\sigma\conc e'\conc
	     {e_0\conc \eHz}\conc {e_1\conc \eHo}\conc{f}\conc e''\conc
	     e'''\conc f''\enspace.
	     \]
	     Consider the derivation consisting of an axiom
	     $\sequent{\tj{\hat x}{\phi_0\limp\phi_1}}
	     {\tj{\hat x}{\phi_0\limp\phi_1}}$.
	     This derivation is shorter than the derivation for
	     $\mathcal H$.
	     We can use the induction hypothesis~\ref{c:second} on
	     this derivation and the derivation of $u$.
	     By this and Proposition~\ref{comod-type}, $(\comodL, t_1',{e_1\conc \eHo})$ is in
	     $\sempair{\phi_0\limp\phi_1,\phi}$.
	     So, program $p =(\comodL,[t_1',\pi'\conc{e_1\conc \eHo\conc
	     e'''}])$ is in
	     $\sem{\phi_0\limp\phi_1}$.
	     By the induction hypothesis~\ref{c:second} for the derivations
	     for $t_0$ and $u$, especially using the program~$p$ for
	     substituting~$\hat x$, we can show that
	     $(t_0',u',(t_1',\pi'\conc {e_1\conc \eHo\conc e'''}\conc {e_0\conc
	     \eHz}\conc
	     e'\conc {f}))$ is in $\sempair{\phi,\psi}$.
	     Thus, the reduct is in $\bbot$.
	     Since $\bbot$ is closed for $\rev\sred$, the original
	     s-executable is also in $\bbot$.
	\end{enumerate}
  \item[($\forall$I, \textminus)]
       The derivation for $t$ ends in
       \aseq{\hyp{'}\G}{\tj t\phi}
       \useq{\hyp{'}\G}{\tj t{\forall X\phi}}
       \DisplayProof.
       \begin{enumerate}[label=\textit{(\arabic{*})}]
	\item Take any
	      $\takehyper{\mathcal{H}'}$ and
	      $(\vec g,{e})\in\sem{\G}$.
	      Since $\G$ does not contain $X$ freely,
	      $\sem\G$ does not change whatever
	      $\nsem{X}_0$ is.
	      By induction
	      hypothesis~\ref{c:first}
	      for arbitrary $\nsem{X}_0$,
	      the program $(t[\vec g/\G], e\conc e_{\mathcal H'})$ is in
	      $\bigcap_{\mathcal Z\in 2^\sPi}\sem{\phi[\mathcal Z/X]}$,
	      which is a subset of $\left(\bigcup_{\mathcal
	      Z\in 2^\sPi}\nsem{\phi[\mathcal Z/X]}\right) \rightarrow
	      \bbot$.
	\item We can replace $X$ in the derivation of $t$ with
	      a propositional variable~$X'$ that does not
	      occur in the derivation of $u$.
	      % The rest is similar to \ref{c:first}.
	      We are going to use the induction hypothesis on the
	      renamed derivation with the s-terms and s-stacks taken above.
	      Since $\theta$ and $\hat\G$ do not contain
	      $X$ freely, we have
	      $(v,w,e')\in\sem{\theta[X'/X],\tau}$,
	      $(\vec{g},{e\conc e_{\mathcal H}})\in\sem{\hat\G[X'/X]}$ and $(\pi,e'')\in
	      \nsem{\forall X'\phi[X'/X]}$.
	      By the induction hypothesis~\ref{c:second}
	      on the renamed derivation,
	      the s-terms
	      $t' = t[v/x][\vec g/\G]$ and
	      $u' = u[w/y][\vec d/\D]$ satisfy
	      $(t',u', e\conc  f\conc e_{\mathcal H})\in\sem{\phi[X'/X],\psi'}$ for any $\nsem{X'}_0$.
	      That is, for $\mathcal Z\in 2^E$ that makes
	      $(\pi,e'')\in\nsem{\phi[X'/X][\mathcal Z/X']}$ and
	      $(\sigma,f'')\in\nsem{\psi[\mathcal Z/X']}$,
	      $t'\pi\conc u',\sigma\conc e''\conc f''$ is in $\bbot$,
	      making $t',\pi\conc u',\sigma$ an element of
	      $\bbot$..
       \end{enumerate}
  \item[($\forall$E, \textminus)]
       Both statements follow from the induction hypotheses
       because $\nsem{\phi[\psi/X]}$ is a subset of $\nsem{\forall
       X\phi}$.
   \item[(Other cases)]
	We can swap $t$ and $u$ and find a symmetric case above.
 \end{description}
 \end{proof}
Note that \ref{c:first} uses \ref{c:second} in (Com,~\textminus)
and \ref{c:second} uses \ref{c:first} in ($\limp$E,~\textminus) and
(IW,~\textminus) in the proof of Theorem~\ref{c:adequacy}.

\begin{proposition}
 \label{prop:exec-on-pole}
 Given a set~$P$ of processes,
 the following set is a pole: the set of s-executables that reduces to
 an s-executable containing an element of $P$.
\end{proposition}

\begin{proposition}[Prelinearity as a Communication Scheme]
 \label{spec:prelin}
 Assume that an s-term $g$ have type $
{\forall X\forall
 Y((X\limp Y)\oplus (Y\limp X))}.
$
 Then, the s-executable
 \[
 [g, \mats{z}{z}{x\cdot\pi_Y}{w}{w}{y\cdot\pi_X}]
 \]
 reduces to a multiset containing an element of
 $\{(x,\pi_X), (y,\pi_Y)\}$.
\end{proposition}
\begin{proof}
 We denote by $\bbot$ the set of s-executables that reduce to a multiset
 containing an element of $\{(x,\pi_X), (y,\pi_Y)\}$.
 By Proposition~\ref{prop:exec-on-pole}, $\bbot$ is a pole.
 Take $\nsem{X}_0 = \{(\pi_X,\emptyset)\}$ and
      $\nsem{Y}_0 = \{(\pi_Y,\emptyset)\}$.
 We have $x\in\sem{X}$ and $y\in\sem{Y}$.
 By definition of $\nsem{\cdot}$,
 $\mats{z}{z}{x\cdot\pi_Y}{w}{w}{y\cdot\pi_X}$ is in $\nsem{(X\limp
 Y)\oplus (Y\limp X)}$\kern -1.5pt.
 By adequacy (Theorem~\ref{c:adequacy}), $\left[g,
 \mats{z}{z}{x\cdot\pi_Y}{w}{w}{y\cdot\pi_X}\right]$ is in $\bbot$.
\end{proof}

\begin{proposition}
 \label{simu}
 Let $e$ be an s-executable.
 If $e\sred e'$ then $e, S_\epsilon \ared e', S_\epsilon$.
\end{proposition}
\begin{proof}
 Most reduction rules are common.  Rules (com0) and (com1) are
 simulated in Figure~\ref{fig:lafont}.  Rule (com2) can be simulated similarly.
\end{proof}
This transfers the semantics of prelinearity
(Proposition~\ref{spec:prelin}) to the asynchronous case.

\begin{example}[A term performing information exchange.]
 \label{spec-ex}
 Since $g = (\inl{\comodL}\conc\inr{\comodR})$ has the type
 $\forall X\forall Y((X\limp Y)\oplus (Y\limp X))$ (Figure~\ref{fig:prelin-term}),
 $g$ satisfies the condition of Proposition~\ref{spec:prelin}.
 Indeed, by (dist),
 $\left[g, \mats{z}{z}{x\cdot\pi_Y}{w}{w}{y\cdot\pi_X}\right]$
 reduces to
 \begin{align*}
  \bigl[& \left(\inl{\comodL},
  \mats{z}{z}{x\cdot\pi_Y}{w}{w}{y\cdot\pi_X}\right) \\ \conc
  & \left(\inr{\comodR}, \mats{z}{z}{x\cdot\pi_Y}{w}{w}{y\cdot\pi_X}\right)\bigr]\enspace.
 \end{align*}
 By (ansL), (ansR) and (cong), the s-term above reduces to
 $[\comodL, x\cdot\pi_Y\conc \comodR,y\cdot\pi_X]$.
 Depending on the schedule,  the s-term above reduces to either
 $[x,\pi_X]$ or $[y,\pi_Y]$ or $[x,\pi_X\conc y,\pi_Y]$.
\end{example}

% After looking at \thref{spec-ex},
% one may wonder whether the statement of \thref{spec:prelin} can be
% strenghthened to force $g$ to reduce to a nonempty multiset on
% $\{(x,\pi_X), (y,\pi_Y)\}$ without any garbage elements.
% However, when there are more than two processes, some garbages can be
% produced.
% \fix{complete this paragraph or abandon it; but it is an important one}

% Another important property of an s-executable is
% non-abortfullness
% which states that the s-executable has no reduction sequence leading to
% $\emptyset,S$.
% The name comes from hyper-lambda calculus formulation by \citet{hiraiflops2012},
% where a s-term sometimes turns into $\mathsf{abort}$ and there the property
% stated that a hyper-term has no reduction sequence leading to a hyper-term
% full of $\mathsf{abort}'s$.
% This property is necessary for s-terms to denote
% constructive content of proofs.

% \begin{proposition}[Non-abortfullness]
%  Let $t$ be a s-term and $\pi$ an s-stack, then,
%  $[t,\pi]$ does not reduce to $\emptyset$.
% \end{proposition}
% \begin{proof}
%  By definition of $\sred$.
% \end{proof}



\section{Characterization of Monoidal t-Norm Logic}
\label{sec:as-logic}

There is a well-known substructural logic called monoidal t-norm logic
(MTL), which validates the prelinearity axiom
$(\phi\limp\psi)\oplus(\psi\limp\phi)$.  Actually, $\NMTL$ characterizes
MTL (Propositions.~\ref{prop:sound} and \ref{prop:comp}).

The MTL formulae can be defined as%
\footnote{From the literature~\citep{handbook:fuzzy}, the connectives
are translated as $\&$ into $\otimes$ and $\rightarrow$ into $\limp$.}%
:
\[
 \phi::= \mathbf{0}\mid X\mid \phi\limp\psi \mid \phi\otimes \psi \mid \phi \oplus \psi\enspace.
\]
As an abbreviation, we can introduce $\phi\land\psi =  (\phi\otimes(\phi\limp\psi))\oplus
 (\psi\otimes(\psi\limp\phi))$\footnote{
The translation of $\wedge$ is a tailor-made one for MTL,
 which is taken from \citep[p.~48]{handbook:fuzzy}.
 The origin of the translation
can be traced back to \citet[Lemma~6.5]{Cintula200740}.}%
.
We take a hypersequent formulation of MTL by
\citet{baaz2004analytic}, whose rules we show in Figure~\ref{fig:hlbck}.
 \begin{figure}[h]
  \centering
  \AxiomC{}
  \LL{(ax)}
  \useq{X}{X}
  \DisplayProof
  \hfill
  \AxiomC{}
  \LL{(ax)}
  \useq{\mathbf 0}{\phi}
  \DisplayProof
  \hfill
  \aseq{\mathcal H\hmid\D_0,\G_0}{\phi}
  \aseq{\mathcal H\hmid \D_1,\G_1}{\psi}
  \LL{(com)}
  \bseq{\mathcal H\hmid \D_0,\D_1}{\phi\hmid\G_0,\G_1\tr \psi}
  \DisplayProof
  \ruleskip
  \aseq{\mathcal H\hmid\G}{\phi}
  \aseq{\mathcal H\hmid \phi,\D}{\psi}
  \LL{(cut)}
  \bseq{\mathcal H\hmid\G,\D}{\psi}
  \DisplayProof
  \hfill
  \aseq{\mathcal H\hmid \G}{\phi}
  \LL{(w,l)}
  \useq{\mathcal H\hmid \G,\psi}{\phi}
  \DisplayProof
  \ruleskip
  \aseq{\mathcal H\hmid\G}{\mathbf{0}}
  \LL{(w,r)}
  \useq{\mathcal H\hmid \G}{\phi}
  \DisplayProof
  \hfill
  \AxiomC{$\mathcal H$}
  \LL{(EW)}
  \useq{\mathcal H\hmid \G}{\phi}
  \DisplayProof
  \ruleskip
  \aseq{\mathcal H\hmid\G,\phi,\psi}{\chi}
  \LL{($\otimes$, l)}
  \useq{\mathcal H\hmid \G,\phi\otimes\psi}{\chi}
  \DisplayProof \hfill
  \aseq{\mathcal H\hmid\G}{\phi}
  \aseq{\mathcal H\hmid\D}{\psi}
  \LL{($\otimes$, r)}
  \bseq{\mathcal H\hmid \G,\D}{\phi\otimes \psi}
  \DisplayProof \ruleskip
  \aseq{\mathcal H\hmid\G}{\phi}
  \aseq{\mathcal H\hmid\D,\psi}{\chi}
  \LL{($\limp$,l)}
  \bseq{\mathcal H\hmid \G,\D,\phi\limp\psi}{\chi}
  \DisplayProof \hfill
  \aseq{\mathcal H\hmid\G,\phi}{\psi}
  \LL{($\limp$, r)}
  \useq{\mathcal H\hmid\G}{\phi\limp\psi}
  \DisplayProof \ruleskip
  \aseq{\mathcal H\hmid\G,\phi}{\chi}
  \aseq{\mathcal H\hmid\G,\psi}{\chi}
  \LL{($\oplus$, l)}
  \bseq{\mathcal H\hmid\G,\phi\oplus\psi}{\chi}
  \DisplayProof
  \hfill
  \aseq{\mathcal H\hmid \G}{\phi_i}
  \LL{($\oplus$, r)}
  \useq{\mathcal H\hmid \G}{\phi_0\oplus \phi_1}
  \DisplayProof

  \caption[The propositional rules in $\HLBCK$~\citep{baaz2004analytic}, which
  characterizes monoidal t-norm logic.]
  {The propositional rules in $\HLBCK$~\citep{baaz2004analytic}, which
  characterizes monoidal t-norm logic.  Axioms are not shown in
  \citet{baaz2004analytic} so we took them from \citet{ono-komori-1985}.
  Exchange rules  are implicit: the contexts and hypersequents are
  treated as finite sets rather than sequences.
  If a formula is provable with the (cut) rule,
  the formula is also provable without the (cut) rule by
  \citet[Theorem~3.2]{baaz2004analytic}.}
  \label{fig:hlbck}
 \end{figure}

We translate MTL formulae to MTL2 formulae by
induction on $\phi$:
\begin{align*}
 \mathbf{0}^\bullet &= \forall XX \nonumber\\
 X^\bullet &= X \nonumber\\
 (\phi\limp\psi)^\bullet &= \phi^\bullet \limp \psi^\bullet\nonumber\\
 (\phi\otimes\psi)^\bullet & = \forall
 X((\phi^\bullet\limp\psi^\bullet\limp X)\limp X) \text{ where }X\text{ do
 not appear in }\phi\otimes\psi\nonumber\\
 (\phi\oplus\psi)^\bullet &= \phi^\bullet \oplus \psi^\bullet\quad.
\end{align*}
We state that any MTL formula~$\phi$ is a theorem in $\HLBCK$ iff
 $\phi^\bullet$ is a theorem in $\NMTL$.
 In order to reason inductively on hypersequent derivations, we extend
 the translation $(\cdot)^\bullet$ to sequents and hypersequents.
 We define $(\G\tr\phi)^\bullet$ to be $\G^\bullet\limp\phi^\bullet$
 where
 $\G^\bullet$ is the $\otimes$-conjunction of translations of the
 elements of $\G$.
 We define
 $\left(\G_0\tr\phi_0\hmid\cdots\hmid\G_n\tr\phi_n\right)^\bullet$ to be
 $(\G_0\tr\phi_0)^\bullet\hmid\cdots \hmid (\G_n\tr\phi_n)^\bullet$.
A substitution instance of an MTL2 formula~$\phi$ is an MTL formula obtained by
substituting an MTL formula for each bound propositional variable in $\phi$.
 \begin{proposition}
  \label{prop:sound-ind}
  If a hypersequent is derivable in $\NMTL$, all substitution instances
  of the hypersequent are derivable in $\HLBCK$.
 \end{proposition}
  \begin{proof}
   By induction on $\NMTL$ derivations.
  \end{proof}

 \begin{proposition}[Soundness]
  \label{prop:sound}
  When the translation $\phi^\bullet$ is provable in $\NMTL$, the
  original $\phi$ is provable in $\HLBCK$.
 \end{proposition}
  \begin{proof}
   We define an MTL formula~$\phi^\circ$ inductively on an MTL
   formula~$\phi$ so that $\phi^\circ$ is a substitution instance of
   $\phi^\bullet$ and $\phi^\circ\limp\phi$ is provable in
   $\HLBCK$.
   The definition of $\phi^\circ$ follows:
   \begin{align*}
    \mathbf 0^\circ&= \mathbf 0\\
    X^\circ &=X \\
    (\phi\limp\psi)^\circ&=\phi^\circ\limp\psi^\circ\\
    (\phi\otimes\psi)^\circ &=
    (\phi^\circ\limp\psi^\circ\limp(\phi^\circ\otimes\psi^\circ))\limp(\phi^\circ\otimes\psi^\circ)\\
    (\phi\oplus\psi)^\circ &= \phi^\circ \oplus \psi^\circ\enspace.
   \end{align*}
   Both claims are immediate.  Assume $\phi^\bullet$
   is provable in $\NMTL$.  By the first claim and Proposition~\ref{prop:sound-ind},
   $\phi^\circ$ is provable in $\HLBCK$.  By the second claim, $\phi$ is
   provable in $\HLBCK$.
  \end{proof}

\begin{proposition}[Completeness]
 \label{prop:comp}
 If a hypersequent $\mathcal H$ is
 derivable in $\HLBCK$ without using (cut), then the translation $\mathcal H^\bullet$ is
 derivable in $\NMTL$.
\end{proposition}
\begin{proof}
 By induction on derivations of the cut-free fragment of $\HLBCK$.
 All cases are straightforward.  For branching rules, we have to apply
 (EC) because in $\HLBCK$, the components in a hypersequent is distributed
 additively while in $\NMTL$ multiplicatively.  And above and below
 (EC), we have to use ($\limp$I) and ($\limp$E) so that the
 restricted form of (EC) rule in $\NMTL$ is applicable.
\end{proof}

\section{Discussion and Future Work}
\label{sec:discuss}
This work is technically similar to Danos and Krivine's~\citep{danos-krivine}.
One large difference is the existence of (com$i$), (read) and (write)
reductions, where terms are passed from processes to other
processes.
In \citep{danos-krivine}, the only reduction rule involving multiple
processes is (dist), which spawns processes that never communicate.
Sharing their purpose
``interpretation of logical rules as programming
instructions''~\citep{danos-krivine},
we continue to seek generalization to other substructural
logics.

The asynchronous semantics given here is similar to that of a
hyper-lambda calculus \lgd\,by
\citet{hiraiflops2012}.
We conjecture that our classical realizability argument here is
applicable to \lgd\,as well, but for that, due to contraction rule,
we have to generalize the statement of adequacy to involve arbitrarily
many derivations.

There are recent
developments~\citep{pfenning2010,wadler2012propositions} over
Curry-Howard correspondence between
the linear logic and process
calculi.
Since their type systems do not incorporate the prelinearity axiom
(without modalities $!$ or $?$),
we conjecture that we can extend their type systems and their process
calculi
with the prelinearity
axiom representing the same kind of communication schemes as we have
shown.

H\'{a}jek's basic fuzzy logic (BL) is very similar to
MTL~\citep{chvalovsky2012}.
Actually from MTL, BL can be obtained by adding
$(\phi\otimes(\phi\limp\psi)) \limp (\psi\otimes(\psi\limp\phi))$ (or a
symmetric axiom).  Thus, we have to find the computational meaning of
the additional axiom in order to find a lambda calculus for BL.

\section{Conclusion}
\label{sec:conclude}
We developed a lambda calculus for MTL2 and
analyzed the prelinearity axiom using
the classical realizability.  The terms typed with prelinearity are
asynchronous communication schemes.

