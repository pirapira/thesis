\chapter{Parametricity for Prelinearity}

	% packages
% \usepackage{setspace}
% \doublespacing

% \usepackage{bussproofs}
% \usepackage{graphicx}
% \usepackage{amssymb}

% % title, author, etc
% \title{Some Disjunctive Tautologies as\\
%  Asynchronous Communication Schemes}
% \author{Yoichi Hirai}

% \usepackage{amsmath}	% required for `\align*' (yatex added)
% \begin{document}

% \maketitle

\newcommand{\ltypes}[1]{\mathsf{Ltypes}({#1})}
\newcommand{\lvars}[1]{\mathsf{Lvars}({#1})}
\newcommand{\rtypes}[1]{\mathsf{Rtypes}({#1})}
\newcommand{\rterms}[1]{\mathsf{Rterms}({#1})}

\renewcommand{\vec}{\overrightarrow}

\newcommand{\sequent}[2]{{#1}\tr{#2}}
\newcommand{\aseq}[2]{\AxiomC{$\sequent{#1}{#2}$}}
\newcommand{\useq}[2]{\UnaryInfC{$\sequent{#1}{#2}$}}
\newcommand{\bseq}[2]{\BinaryInfC{$\sequent{#1}{#2}$}}

\newcommand{\Haseq}[3]{\AxiomC{$\sequent{#2}{#3}\hmid{#1}$}}
\newcommand{\Huseq}[3]{\UnaryInfC{$\sequent{#2}{#3}\hmid{#1}$}}
\newcommand{\Hbseq}[3]{\BinaryInfC{$\sequent{#2}{#3}\hmid{#1}$}}

\newcommand{\Hy}{\mathcal{H}}
\newcommand{\Ky}{\mathcal{K}}
\newcommand{\limp}{\multimap}


\subsection{Summary}
%%% abstract
This work is a refinement of
Danos and Krivine's ``Disjunctive Tautologies as
Synchronisation Schemes.''
We show that some autologies,
namely the therems of G\"odel--Dummett logic,
can be viewed as asynchronous communication schemes.
We call these asynchronous because the operational semantics
of the communicating terms does not allow a term to wait for another
term.
In contrast to Danos and Krivine's \fix{cite} formulation,
where computation can wait for another,
our calculus is weaker and resides in a particular class of
distributed computation called waitfreedom.
This work is a reformulation of Hirai~\fix{cite} in the setting of
second-order logic.
This allowed parametricity argument.
Also, differently from Hirai's \fix{cite} hypersequent formulation,
we succceed in describing the computational behaviour of the Dummett
(a.k.a symmetric excluded middle) in the implicational form: this
difference is to be discussed in Sect. \fix{?}.

%%% 1. introduction

% 1a operational semantics for GD case

% 1b difference from Danos--Krivine

%%% 2. terms types and _models_?
\section{Synchronous Semantics}

% 2a terms -- unchanged -- remove pi or anything

\newcommand{\conc}{\parallel}
\newcommand{\comod}[2]{\ast^{\rightarrow {#2}}_{\leftarrow{#1}}}
\newcommand{\reader}[1]{\ast_{\leftarrow{#1}}}

We consider a programming language, which is a modification of
Danos and Krivine's.
We assume a set~$\pvar$ of propositional variables whose cardinality is
countably infinite.
For a set~$S$,
the set $\form(S)$ denotes the set of formulas:
\[
\phi::= s \mid X \mid \phi\limp\phi\mid \forall X \phi
\]
where $s\in S$ and $X\in \pvar$.
A type is an element of $\form(\emptyset)$.
Channels have involution $c\mapsto\co c$ with
$c\neq \co c$ and
$\co{\co c} = c$.
A \textit{term}~$t$ is defined by BNF:
\[
 t::= x
 \mid (t)t
 \mid \lambda x.t
 \mid \ast_t
 \mid \comod c c
\]
where $x$ is a variable and $c$ is a channel.
A \textit{stack}~$\pi$ is defined by BNF:
\[
 \pi ::= \epsilon
 \mid t\cdot \pi
 \enspace.
\]
We write the set of terms as $\Lambda$ and stacks~$\Pi$.
An \textit{executable} is a finite multiset on $\Lambda \times \Pi$.

% 2a' reduction
\newcommand{\red}{\succ}
The reduction relation~$\red$ on executables
is defined to be the smallest preorder
that satisfies:
\begin{description}
 \item[(cong)] if
	    $e_0         \red e_1$
	    then
	    $[e_0 \conc e] \red [ e_1\conc e]$  \enspace;
 \item[(push)]
	    $[(t)u,\pi]      \red [t,u\cdot\pi]$      \enspace;
 \item[(store)]
	    $[\lambda x.t,u\cdot\pi]
	     \red
	     [t[\ast_u/x],      \pi]$\enspace;
 \item[(load)]
	    $[\ast_u,\pi],S\red[u,\pi]$\enspace;
 \item[(com)]
	    $[\comod c{\co c}, t\cdot\pi\conc \comod{\co c}c,
	    u\cdot\sigma] \red
	    [u,\pi\conc t,\sigma]$\enspace.
 % this kind of failures lead to something strange
 % it gets affine
 % \item[(com0)]
 % 	    $[\comod c{\co c}, t\cdot\pi\conc \comod{\co c}c,
 % 	    u\cdot\sigma] \red
 % 	    [u,\pi]$\enspace.
 % \item[(com1)]
 % 	    $[\comod c{\co c}, t\cdot\pi\conc \comod{\co c}c,
 % 	    u\cdot\sigma] \red
 % 	    [t,\sigma]$\enspace.
\end{description}
From Danos and Krivine's formalization~\fix{cite},
we removed terms of the form $ t\conc t $ and
reductions
	    $[t\conc u,\pi]  \red [t,\pi\conc u,\pi]$
because they duplicate terms in $\pi$ and hence destroy the linearity.

% I want to write these rules are non-blocking here.
% but, actually, Danos and Krivines rules are also non-blocking.
% they just busy-wait.
% In order to remove busy-waiting, we need types.

% 2b type system -- remove pierce -- add com'

\fix{describe a sequent; we do not allow multiple occurrences of the
same variable}
\fix{say, up to external exchange, we identify}

The derivation rules follow.\\ \noindent

\newcommand{\xphi}{\tj{x}{\phi}}

\AxiomC{}
\LL  {Ax}
\useq{\xphi}{\xphi}
\DisplayProof
\ruleskip
%
\Haseq\Hy{\xphi,\G}{\tj t\psi}
\LL{$\limp$I}
\Huseq\Hy{\G}{\tj{\lambda x.t}{\phi\limp\psi}}
\DisplayProof
\ruleskip
%
\Haseq{\Hy_0}{\G}{\tj t{\phi\limp\psi}}
\Haseq{\Hy_1}{\D}{\tj u\phi}
\LL   {$\limp$E}
\Hbseq{\Hy_0\hmid\Hy_1}{\G,\D}{\tj{(t)u}\psi}
\DisplayProof
\ruleskip
%
\Haseq{\Hy}\G{\tj t\phi}
\LL   {$\forall$I}
\Huseq{\Hy}\G{\tj t{\forall X\phi}}
\DisplayProof
\ruleskip
%
\Haseq{\Hy}{\G}{\tj{t}{\forall X\phi}}
\LL   {$\forall$E}
\Huseq{\Hy}{\G}{\tj{t}{\phi[\psi/X]}}
\DisplayProof
\ruleskip
%
\newcommand{\comodL}{\comod c{\co c}}
\newcommand{\comodR}{\comod{\co c}c}
\Haseq{\Hy_0}{\tj{x}{\phi\limp\psi},\G}{\tj t\theta}
\Haseq{\Hy_1}{\tj{y}{\psi\limp\phi},\G}{\tj u\tau}
\LL{Com}
\BinaryInfC{$
\Hy_0\hmid\Hy_1\hmid
\sequent\G{t[\comodL/x]}
\hmid
\sequent{\D}{u[\comodR/y]}
$}
\DisplayProof

\fix{and exchange rules}


\section{Specification Using Poles}

% 2C truth values and models. make them disjunctive!
%    so that the empty set does not count as successful.
%    Also, make it closed for ``for all''
A \textit{pole}~$\bbot$ is a set of executables
which satisfies
\begin{enumerate}
 \item $e$ is in $\bbot$ if $e'\in\bbot$ and
       $e\red e'$; and
 \item if $(m,S)\in\bbot$ and $(m',S)\in\bbot$
       then $(m\conc m',S)\in\bbot$.
\end{enumerate}

An \textit{environment} is a pair of a stack and an executable.
The set of environments is written as~$E$.
A \textit{program} is a pair of a term an executable.
For a set~$\mathcal Z$ of environments, $\mathcal Z\limp\bbot$ denotes
the set of programs $(t,e)$ such that
for any environment $(\pi,e')\in \mathcal Z$,
the executable $t,\pi\conc e \conc e'$ is in $\bbot$.
A set $\mathcal X$ of programss is called a truth value
iff there exists
such $\mathcal Z$ that $\mathcal X = \mathcal Z\limp \bbot$.

\renewcommand{\sem}[1]{|{#1}|}
\newcommand{\nsem}[1]{\sem{#1}^-}
\newcommand{\sempair}[1]{\sem{(#1)}}
For $F\in\form(2^E)$ and $|\cdot|^-_0\colon\pvar\rightarrow 2^E$,
we define $\nsem{\phi}\colon \form(2^E)\rightarrow
2^E$ inductively as
\begin{align*}
 \nsem{\mathcal Z} &= \mathcal Z \\
 \nsem{X}&= |X|_0^- \\
 \nsem{\phi\limp\psi}&=
 \{(t\cdot\pi, e_0\conc e_1)\mid
 (t,e_0)\in\nsem\phi\rightarrow\bbot \text{ and }(\pi,e_1)\in\nsem\psi\}
 \\
 \nsem{\forall X\phi}&=
 \bigcup_{\mathcal Z\in 2^\Pi} \nsem{\phi[\mathcal Z/X]}
\end{align*}
Using this, we define $\sem \phi=\nsem{\phi}\rightarrow\bbot$.
We have an equality
$\nsem{\phi\limp\psi} = \{(t\cdot\pi, e_0\conc e_1)\mid
(t,e_0)\in\sem\phi\text{ and }(\pi,e_1)\in\nsem\psi\}$, which we denote
$\sem{\phi}\cdot\nsem{\psi}$.
Moreover, for types~$\phi$ and $\psi$, we define $\sempair{\phi,\psi}$
as the set of pairs $((t,e_0),(u,e_1))$ of programs such
that
$t,\pi_\phi\conc u,\pi_\psi\conc e_0\conc e_1\conc e_2\conc e_3
\in\bbot$ for any $(\pi_\phi,e_2)\in\nsem\phi$
and $(\pi_\psi,e_3)\in\nsem\psi$.

\begin{proposition}
 \label{comod-type}
 $((\comodL,\emptyset), (\comodR,\emptyset))\in\sempair{\phi\limp\psi,\psi\limp\phi}$
 for any types~$\phi$ and $\psi$.
\end{proposition}
\begin{proof}
 Take any $(t\cdot\pi_\psi,e_{00}\conc e_{01})\in\nsem{\phi\limp\psi}$
 and $(u\cdot\pi_\phi,e_{10}\conc e_{11})\in\nsem{\psi\limp\phi}$ so that
 $(t, e_{00})\in\sem{\phi}$, $(\pi_\psi,e_{01})\in\nsem{\psi}$,
 $(u, e_{10})\in\sem{\psi}$, $(\pi_\phi,e_{11})\in\nsem{\phi}$.
 We claim $e=\comodL,t\cdot\pi_\psi\conc\comodR,u\cdot\pi_\phi\conc
 e_{00}\conc e_{01}\conc e_{10}\conc e_{11}$ is in
 $\bbot$.
 By (com), $e\red t,\pi_\phi\conc u,\pi_\psi\conc e_{00}\conc
 e_{01}\conc e_{10}\conc e_{11}$, which is in $\bbot$ because
 we have $t,\pi_\phi\conc e_{00}\conc e_{11}\in\bbot$ and
 $u,\pi_\psi\conc e_{01}\conc e_{10}\in\bbot$.
 Since $\bbot$ is closed for $\red^{-1}$,
 we have $e\in\bbot$.
\end{proof}

% 2d the adequacy lemma
\fix{describe the notation $[\vec d/\G]$ and $\sem{\G}$}

\fix{add what we really want to show is this short proposition: ...
}
We deal with two derivations because otherwise the proof for the
Com rule fails.
We include $t_{ij}$ terms in the statements because otherwise
the proof for $\limp$E rule fails.

\begin{proposition}[Adequacy]
 Let these hypersequents be derivable:
 \begin{align*}
  \Hy = &\sequent{\G_0}{\tj{t_0}{\phi_0}} \hmid
  \sequent{\G_1}{\tj{t_1}{\phi_1}} \hmid \cdots \hmid
  \sequent{\G_{m}}{\tj{t_{m}}{\phi_{m}}}
  &\text{ and}\\
  \Ky = &\sequent{\D_0}{\tj{u_0}{\psi_0}} \hmid
  \sequent{\D_1}{\tj{u_1}{\psi_1}} \hmid \cdots \hmid
  \sequent{\D_{n}}{\tj{u_{n}}{\psi_{n}}}\enspace.
 \end{align*}
 Under these conditions, we make two statements.
 \begin{enumerate}[label=(\arabic{*}), ref=\textit{(\arabic{*})}]
  \item \label{c:first} the executable
	\[
	\bigparallel_{0\le i\le m}
	\left(t_i[\vec{g_i}/\G_i], \pi_i\conc e_i\conc \vec{f_i}\right)
	\]
	is in $\bbot$
	 for any
	$(\vec{g_i},\vec{f_i})\in\sem{\G_i}$ and $(\pi_i, e_i)\in\nsem{\phi_i}$; and that
  \item \label{c:second}
	when $\G_{\hat\imath}$ and $\D_{\hat\jmath}$ are respectively
	equal to $\tj x\phi, \G$ and $\tj y\psi, \D$ up to exchange,
	the following executable is in $\bbot$:
	\begin{align*}
	 &
	 t_{\hat\imath}[v/x][\vec{g}/\G],\pi_{\hat\imath}
	 \conc e'\conc e'_{\hat\imath} \conc \vec e\conc
	 u_{\hat\jmath}[w/y][\vec d /\D],
	 \sigma_{\hat\jmath}\conc f'\conc f'_{\hat\jmath}\conc \vec f \\ \conc
	&\bigparallel_{i\neq\hat\imath} \left(t_i[\vec{g_i}/\G_i],
	\pi_i\conc e'_i\conc \vec{e_i}\right)
	 \quad \conc
	\bigparallel_{j\neq\hat\jmath} \left(u_j[\vec{d_j}/\D_j],
	\sigma_j\conc f'_j\conc {\vec{f_j}}\right)
	\end{align*}
	for any
	$((v,e'),(w,f'))    \in \sempair{\phi,\psi}$,
	$(\vec{g},\vec{e})  \in \sem{\G}$,
	$(\vec d,\vec{f})   \in \sem\D$,
	$(\vec{g_i},\vec{e_i})\in \sem{\G_i}$ for $i\neq\hat\imath$,
	$(\vec{d_j},\vec{f_j})\in \sem{\D_j}$ for $j\neq\hat\jmath$,
	$(\pi_i,e'_i)    \in \nsem{\phi_i}$   for $0\le i\le m$ and
	$(\sigma_j,f'_j) \in \nsem{\psi_j}$   for $0\le j\le n$.
  \end{enumerate}
 \end{proposition}
 \begin{proof}
 We prove both statements at the same time by induction on the sum of
 the heights of the derivations.
  \begin{description}
  \item[(Ax, Ax)] When both derivations consist of only axiom rules,
       the statements are tautologies.
   \item[($\limp$I, \textminus)]
       In this case, the derivation for~$t$ ends as
       \Haseq{\Hy'}{\tj {z}{\hat\phi},\G_0}{\tj {t'}{\hat\psi}}
       \Huseq{\Hy'}{\G_0}{\tj{\lambda {z}.t'}{\hat\phi\limp{\hat\psi}}}
       \DisplayProof.
       \begin{enumerate}[label=\textit{(\arabic{*})}]
	\item Take any
	      $(\vec{g_i},\vec{f_i})\in\sem{\G_i}$,
	      $(v\cdot\pi,e_0\conc e_1)\in\nsem{\hat\phi\limp\hat\psi}$
	      and
	      $(\pi_i,e_i)\in\sem{\phi_i}$ for $1\le i\le m$ so that
	      $(v,e_0)\in\sem{\hat\phi}$ and
	      $(\pi,e_1)\in\nsem{\hat\psi}$.
	      Then, by (load),
	      we have $(\ast_v,e_0)\in\sem{\hat\phi}$.
	      We need to show
	      \[
	       (\lambda z.t')[\vec{g_0}/\G_0], v\cdot \pi\conc
	      e_0\conc e_1\conc f\in\bbot
	      \]
	      where $f=\bigparallel_{1\le i\le
	      m}\left(t_i[\vec{g_i}/\G_i],\pi_i\conc e_i\right)$.
	      By (store), the executable reduces to
	      $t'[\vec{g_0}/\G_0][\ast_v/z],\pi\conc e_0\conc e_1\conc
	      f$,
	      which is in $\bbot$ by induction hypothesis~\ref{c:first}.
	      Since $\bbot$ is closed for $\red^{-1}$, the executable in
	      question is in $\bbot$.
	\item
	     Either $\G_0$ or other $\G_i$ contains $\xphi$.
	     \begin{description}
	      \item[($\hat\imath = 0$)] In this case, we have $\G_0 = \xphi, \G$.
	      \item[($\hat\imath > 0$)] 
	     \end{description}
	     \fix{fill}
       \end{enumerate}
  \item[($\limp$E, \textminus)]
	 The derivation for $t$ ends as
       \[
	     \Haseq{\Hy_0}{\G_{00}}{\tj{t_{00}}{\hat\phi\limp\phi_0}}
	     \Haseq{\Hy_1}{\G_{01}}{\tj{t_{01}}{\hat\phi}}
	     \Hbseq{\Hy_0\hmid \Hy_1}{\G_{00},\G_{01}}{\tj{(t_{00})t_{01}}{\phi_0}}
	     \DisplayProof
       \]
	     and the derivation for $u$ ends at sequent
	     $\sequent{\tj{y}{\psi},\D}{\tj{u_0}{\psi_0}}\hmid \Ky$.
	 We have $t_0=(t_{00})t_{01}$.
       \begin{enumerate}[label=\textit{(\arabic{*})}]
	\item \fix{after (2)}
	      Take any $\vec{d_0}\in\sem{\D_0}$,
	      $\vec{d_1}\in\sem{\D_1}$,
	      $\vec g\in\sem\G$ and $\pi\in\nsem{\phi'}$.
	      By induction hypothesis~\ref{c:first}, we have
	      $t_0'=t_0[\vec{d_0}/\D_0][\vec
	      g/\G]\in\sem{\phi_1\limp\phi'}$
	      and
	      $t_1'=t_1[\vec{d_1}/\D_1][\vec
	      g/\G]\in{\sem{\phi_1}}$.
	      Since variables in $\D_0$ and $\D_1$ are
	      disjoint, we have
	      $t'=((t_0)t_1)[\vec{d_0}/\D_0][\vec{d_1}/\D_1][\vec
	      g/\G] = (t_0')t_1'$.
		    By (push), $(t_0')t_1',\pi\red t_0',t_1'\cdot\pi$,
		    which is in $\bbot$ because
		    $t_0'\in\sem{\phi_1\limp\phi'}$ and
		    $t_1'\cdot\pi\in\nsem{\phi_1\limp\phi'} =
		    \sem{\phi_1}\cdot\nsem{\phi'}$.
		    Since $\bbot$ is closed for $\red^{-1}$,
		    $(t_0')t_1',\pi$ is in $\bbot$.
		    Since $\pi$ was taken arbitrarily,
		    we have $(t_0')t_1'\in\sem{\phi'}$.
	\item $\xphi$ might be in $\G_{00}$ or $\G_{01}$ or another
	      $\G_{\hat\imath}$ ($\hat\imath >0$).
	     \begin{description}
	      \item[($\tj{x}{\phi}\in\G_{00}$)]
		   \fix{use the new}
		    Let $\D_0$ be equal to $\tj{x}{\phi},\overline{\D_0}$ up
		    to exchange.
		    The derivation for $t$ ends in\[
		    \aseq{\tj{x}{\phi},\overline{\D_0}}{\tj{t_0}{\phi_1\limp\phi'}}
		    \aseq{\D_1}{\tj{t_1}{\phi_1}}
		    \bseq{\tj{x}{\phi},\overline{\D_0},\D_1}{\tj{(t_0)t_1}{\phi'}}
		    \DisplayProof\enspace.
						  \]
		    Take any $(v,w)\in\sem{\phi,\psi}$,
		             $\vec{d_0}\in\sem{\overline{\D_0}}$,
		             $\vec{d_1}\in\sem{\D_1}$,
		             $\vec{d'}\in\sem{\D'c}$,
		             $\vec g\in\sem{\G}$,
		             $\pi\in\nsem{\phi'}$ and
		             $\sigma\in\nsem{\psi'}$.
		    By induction hypothesis~\ref{c:second},
		    $t_0'=t_0[v/x][\vec{d_0}/\overline{\D_0}][\vec
		    g/\G]$ and
		    $u' = u[w/y][\vec{d'}/\D'][\vec g/\G]$ satisfy
		    $(t_0',u')\in\sem{\phi_1\limp\phi',\psi'}$.
		    By induction hypothesis~\ref{c:first},
		    $t_1'=t_1[\vec{d_1}/\D_1][\vec g/\G]$ is in
		    $\sem{\phi_1}$.
		    We have $t' =
		    ((t_0)t_1)[v/x][\vec{d_0}/\overline{\D_0}][\vec{d_1}/\D_1][\vec
		    g/\G] = (t_0')t_1'$.
		    By (push) and (cong), $(t_0')t_1',\pi\conc
		    u',\sigma\red t_0',t_1'\cdot\pi\conc u',\sigma$,
		    which is in $\bbot$ because
		    $(t_0',u')\in\sem{\phi_1\limp\phi',\psi'}$ and
		    $t_1'\cdot\pi\in \sem{\phi_1}\cdot\nsem{\phi'} =
		    \nsem{\phi_1\limp\phi'}$, $u'\in\sem{\psi'}$
		    and $\sigma\in\nsem{\psi'}$.
		    Since $\bbot$ is closed for $\red^{-1}$,
		    $(t_0')t_1',\pi\conc u',\sigma$ is in $\bbot$.
	      \item[($\xphi\in\G_{01}$)]
		   Let $\G_{01}$ be equal to $\xphi,\G'_{01}$.
		   Then the derivation ends in
		   \[
		   \Haseq{\Hy_0}{\G_{00}}{\tj{t_{00}}{\hat\phi\limp\phi_0}}
		   \Haseq{\Hy_1}{\xphi,\G'_{01}}{\tj{t_{01}}{\hat\phi}}
		   \Hbseq{\Hy_{0}\hmid
		   \Hy_{1}}{\G_{00},\xphi,\G'_{01}}{\tj{(t_{00})t_{01}}{\phi_0}}
		   \DisplayProof\enspace.
		   \]
		   Take any
		   $((v,e'),(w,f'))               \in\sempair{\phi,\psi}$,
		   $(\vec{g_{00}},\vec{e_{00}})   \in\sem{\G_{00}}$,
		   $(\vec{g_{01}},\vec{e_{01}})   \in\sem{\G_{01}}$,
		   $(\vec d,      \vec{f})        \in\sem{\D}$,
		   $(\vec{g_i},   \vec{e_i})      \in\sem{\G_{i}}$ for
		   $i>0$,
		   $(\vec{d_j},\vec{f_j})         \in\sem{\D_j}$ for
		   $i>0$,
		   $(\pi_i,e'_i)                  \in\nsem{\phi_i}$
		   for $0\le i\le m$ and
		   $(\sigma_j,f'_j)               \in\nsem{\psi_i}$ for
		   $0\le j\le n$.
		   Then, we have to show
		   that the following executable is in $\bbot$:
		   \[
		    ((t_{00})t_{01})[v/x][\vec{g_{00}}/\G_{00}][\vec{g_{01}}/\G_{01}],\pi_0\conc
		   e'_0\conc e_0\conc e_1\in\bbot
		   \]
		   where
		   \begin{align*}
		    e_0 =&
		    \vec{e_{00}}\conc
		    \bigparallel_{i\in\Hy_0}(t_i[\vec{g_i}/\G_i],\pi_i\conc
		    e'_i \conc \vec{e_i})\text{ and}\\
		    e_1 =&
		    e'\conc \vec{e_{01}}\conc
		    \bigparallel_{i\in\Hy_1}(t_i[\vec{g_i}/\G_i],\pi_i
		    \conc e'_i\conc \vec{e_i}) \conc \\ &
		    u_0[w/y][\vec d/\D],\sigma_0\conc f'_0\conc
		    \vec{f}\conc
		    \bigparallel_{1\le j\le n} \kern -8pt \left(u_j[\vec{d_j}/\D_j],
		    \sigma_j\conc f'_j\conc {\vec{f_j}}\right).
		   \end{align*}
		   The executable in question reduces to
		   \[
		    t_{00}[\vec{g_{00}}/\G_{00}],t_{01}[v/x][\vec{g_{01}}/\G_{01}]\cdot\pi_0\conc
		   e'_0\conc e_0\conc e_1\in\bbot\enspace.
		   \]
		   By induction hypothesis~\ref{c:second},
		   $(t_{01}[v/x][\vec{g_{01}}/\G_{01}],e_1)$ is in
		   $\sem{\hat\phi}$.
		   Since $(\pi_0,e'_0)$ is in $\nsem{\phi_0}$,
		   the environment
		   $(t_{01}[v/x][\vec{g_{01}}/\G_{01}]\cdot\pi_0,
		   e_1\conc e'_0)$ is in $\nsem{\hat\phi\limp\phi_0}$.
		   By induction hypothesis~\ref{c:first},
		   the program
		   $ t_{00}[\vec{g_{00}}/\G_{00}], e_0 $ is in
		   $\sem{\hat\phi\limp\phi_0}$.
		   As a result, we have shown the statement.
	      \item[($\xphi\in\G_i$ where $i>0$)]
	     \end{description}
       \end{enumerate}
   \item[(Com, \textminus)]
       The derivation for $t$ ends in
       \[
       \aseq{\tj{\hat x}{\theta\limp\tau},\D_0}{\tj{t_0}{\phi'}}
       \aseq{\tj{\hat y}{\tau\limp\theta},\D_1}{\tj{t_1}{\phi'}}
       \bseq{\D_0,\D_1}{\tj{(t_0[\comodL/\hat x]\conc t_1[\comodR/\hat
       y])}{\phi'}}
       \DisplayProof\enspace.
       \]
       \begin{enumerate}[label=\textit{(\arabic{*})}]
	\item Take any
	      $\vec{d_0}\in\sem{\D_0}$,
	      $\vec{d_1}\in\sem{\D_1}$,
	      $\vec{g}  \in\sem{\G}$.
	      By Prop.~\ref{comod-type}, we have
	      $(\comodL,\comodR)\in\sem{\theta\limp\tau,\tau\limp\theta}$.
	      By induction hypothesis~\ref{c:second}, the terms
	      $t_0'=t_0[\comodL/\hat x][\vec{d_0}/\D_0][\vec g/\G]$
	      and
	      $t_1'=t_1[\comodR/\hat y][\vec{d_1}/\D_1][\vec g/\G]$
	      satisfy
	      $(t_0',t_1')\in\sem{\phi',\phi'}$.
	      By Prop.~\ref{squash}, we have $t_0'\conc
	      t_1'\in\sem{\phi'}$.
	      This term $t_0'\conc t_1'$ is equal to
	      $(t_0[\comodL/\hat x]\conc t_1[\comodR/\hat
	      y])[\vec{d_0}/\D_0][\vec{d_1}/\D_1][\vec g/\G]$.
	\item Without loss of generality,
	      we can assume $(\tj x\phi)\in\D_0$ so
	      $\D_0 =\tj x\phi, \overline{\D_0}$ up to exchange.
	      Take any
	      $(v,w)\in\sem{\phi,\psi}$,
	      $\vec{d_0}\in\sem{\overline{\D_0}}$,
	      $\vec{d_1}\in\sem{\overline{\D_1}}$,
	      $\vec{d'}\in\sem{\D'}$,
	      $\vec g\in \sem\G$,
	      $\pi\in\nsem{\phi'}$ and
	      $\sigma\in\nsem{\psi'}$.
	      By induction hypothesis~\ref{c:second},
	      the terms $t_0'=t_0[v/x][\vec{d_0}/\overline{\D_0}][\vec
	      g/\G]$
	      and
	      $u'=u[w/y][\vec{d_1}/\D_1][\vec g/\G]$
	      satisfy
	      $(t_0',u')\in\sem{\phi',\psi'}$.
	      By induction hypothesis~\ref{c:first},
	      $t_1'=t_1[\vec{d_1}/\D_1][\vec g/\G]$ is in $\sem{\phi'}$.
	      From these, respectively, we obtain
	      $t_0',\pi\conc u',\sigma\in\bbot$
	      and
	      $t_1',\pi\in\bbot$.
	      Since $\bbot$ is closed for $\conc$,
	      we have $t_0',\pi\conc t_1',\pi\conc u',\sigma\in\bbot$.
	      By (dist) and (cong),
	      $(t_0'\conc t_1'),\pi\conc u',\sigma$ reduces
	      to this executable.
	      So $(t_0'\conc t_1',u')\in\sem{\phi',\psi'}$
	      where $t_0'\conc t_1' = (t_0[\comodL/\hat x]\conc
	      t_1[\comodR/\hat y])[\vec{d_0}/\D_0][\vec{d_1}/\D_1][\vec g/\G]$.
       \end{enumerate}
  \item[($\forall$I, \textminus)]
       \fix{renew}
       The derivation for $t$ ends in
       \aseq{\hat\D}{\tj t\phi}
       \useq{\hat\D}{\tj t{\forall X\phi}}
       \DisplayProof.
       \begin{enumerate}[label=\textit{(\arabic{*})}]
	\item Take any $\vec d\in\sem{\hat\D}$ and $\vec g\in\sem{\G}$.
	      Since neither $\hat\D$ nor $\G$ contains $X$ freely,
	      the sets $\sem{\hat\D}$ and $\G$ do not change whatever
	      $\nsem{X}_0$ is.
	      On the other hand, since the indcution
	      hypothesis~\ref{c:first}
	      holds for arbitrary $\nsem{X}_0$,
	      the term $t[\vec d/\hat\D][\vec g/\G]$ is in
	      $\bigcap_{\mathcal Z\in 2^\Pi}\sem{\phi'[\mathcal Z/X]}$.
	      Since $(\bigcap_{\mathcal Z\in 2^\Pi}\sem{\phi'[\mathcal
	      Z/X]})\cdot (\bigcup_{\mathcal Z\in
	      2^\Pi}\nsem{\phi'[\mathcal Z/X]})$ is a subset of $\bbot$,
	      the term is in $\sem{\forall X\phi'} = \left(\bigcup_{\mathcal
	      Z\in 2^\Pi}\nsem{\phi'[\mathcal Z/X]}\right) \rightarrow
	      \bbot$.
	\item Take any
	      $(v,w)\in\sem{\phi,\psi}$,
	      $\vec{d_0}\in\sem{\D_0}$,
	      $\vec{d_1}\in\sem{\D_1}$,
	      $\vec g   \in\sem{\G}$,
	      $\pi_{\phi'}\in\nsem{\forall X\phi'}$ and
	      $\pi_{\psi'}\in\nsem{\psi'}$.
	      We can replace $X$ in the derivation of $t$ with
	      another propositional variable~$X'$ that does not
	      occur in the derivation of $u$.
	      We are going to use the induction hypothesis on the
	      renamed derivation with the terms and stacks taken above.
	      Since $\phi,\D_0$ and $\G$ do not contain
	      $X$ freely, we have
	      $(v,w)\in\sem{\phi[X'/X],\psi}$,
	      $\vec{d_0}\in\sem{\D_0[X'/X]}$,
	      $\vec{g}\in\sem{\G[X'/X]} = \G$ and $\pi_{\phi'}\in
	      \nsem{\forall X'\phi'[X'/X]}$.
	      By induction hypothesis on the renamed derivation,
	      the terms
	      $t' = t[v/x][\vec{d_0}/\D_0][\vec g/\G]$ and
	      $u' = u[w/y][\vec{d_1}/\D_1][\vec g/\G]$ satisfy
	      $(t',u')\in\sem{\phi'[X'/X],\psi'}$ for any $\nsem{X'}_0$.
	      That is, for any $\mathcal Z\in 2^\Pi$,
	      $\pi\in\nsem{\phi'[X'/X][\mathcal Z/X']}$ and
	      $\sigma\in\nsem{\psi'[\mathcal Z/X']}$,
	      $t'\pi\conc u',\sigma$ is in $\bbot$,
	      making $t',\pi_{\phi'}\conc u',\pi_{\sigma'}$ an elment of
	      $\bbot$ because
	      $\pi_{\psi'}\in\nsem{\psi'}=\nsem{\psi'[\mathcal Z/X']}$,
	      where the last equality holds because of the way we chose
	      the variable~$X'$.
       \end{enumerate}
  \item[($\forall$E, \textminus)]
       \fix{renew}
       The derivation for $t$ ends in
       \aseq{\tj x\phi,\D}{\tj t{\forall X\phi}}
       \useq{\tj x\phi,\D}{\tj t{\phi[\psi/X]}}
       \DisplayProof.
       Both statements follow immediately from the induction hypotheses
       because $\nsem{\phi[\psi/X]}$ is a subset of $\nsem{\forall X\phi}$.
 \end{description}
\end{proof}
Note that \ref{c:first} uses \ref{c:second} and vice versa.

\begin{proposition}
 Let $E$ be a set of executables.
 Then the following set is a pole: the set of executables that reduces to
 a multiset on $E$.
\end{proposition}
\begin{proof}
 \fix{fill}
\end{proof}

\begin{proposition}
 Let
 $\sequent{}{}{\tj c
 {\forall X\forall A\forall B
 [((A\limp B)\limp X)
  \limp((B\limp A)\limp X)
  \limp X]}}$
 be
 derivable.
 For all terms $\rho,\sigma, r$ and for all stacks $\pi, \pi_A$ and
 $\pi_B$,
 if $[(\rho)d  ,\pi]\reduce [d,a\cdot \pi_B]$ and
    $[(\sigma)f,\pi]\reduce [f,b\cdot \pi_A]$ hold for all $d$ and $f$,
 then
 $[(c)(\rho)\sigma,\pi],S_\epsilon$ reduces to a multiset on
 $E = \{(a,\pi_A),(b,\pi_B)\}$.
\end{proposition}
\begin{proof}
\end{proof}

\section{Adding Exponentials}


\section{Asynchronous Semantics}

A \textit{term}~$t$ is defined by BNF:
\[
 t::= x
 \mid (t)t
 \mid t\conc t
 \mid \lambda x.t
 \mid \ast_t
 \mid \comod c c
 \mid \reader  c
\]
where $x$ is a variable and $c\colon\phi$ is a channel.
A \textit{stack}~$\pi$ is defined by BNF:
\[
 \pi ::= \epsilon
 \mid t\cdot \pi
 \enspace.
\]
We write the set of terms as $\Lambda$ and stacks~$\Pi$.
A \textit{store} is a partial mapping from locations to
terms, equivalently, a mapping from locations to terms and $\bot$'s.
The \textit{empty store}~$S_\epsilon$ maps any channel to $\bot$.
For a store $S$, we define a store $S[c\colon\phi\mapsto x]$ to be
the same as $S$ except that $S[c\colon\phi\mapsto x](c\colon\phi\mapsto
x)$ is $x$.
A \textit{store} is a partial mapping from locations to
terms, equivalently, a mapping from locations to terms and $\bot$'s.
The \textit{empty store}~$S_\epsilon$ maps any channel to $\bot$.
For a store $S$, we define a store $S[c\colon\phi\mapsto x]$ to be
the same as $S$ except that $S[c\colon\phi\mapsto x](c\colon\phi\mapsto
x)$ is $x$.
An \textit{executable} is a finite multiset on $\Lambda \times \Pi$,
paired with a store.

% 2a' reduction
We have reduction relation~$\red$ on executables,
which is defined to be the smallest binary relation
that satisfies:
\begin{description}
 \item[(cong)] if
	    $[t,\pi],         S \red [ t',\pi'],        S'$
	    then
	    $[t,\pi \conc e], S \red [ t',\pi'\conc e], S'$\enspace;
 \item[(push)]
	    $[(t)u,\pi],S       \red [t,u\cdot\pi],S$;
 \item[(dist)]
	    $[t\conc u,\pi],S   \red [t,\pi\conc u,\pi],S$\enspace;
 \item[(store)]
	    $[\lambda x.t,u\cdot\pi],S
	     \red
	     [t[\ast_u/x],      \pi],S$\enspace;
 \item[(load)]
	    $[\ast_u,\pi],S\red[u,\pi],S$\enspace;
 \item[(write)]
	    $
	    [\comod c{\co c}, t\cdot\pi], S[\co
	    c\colon\psi\mapsto\bot]
	    \red
	    [\reader c, \pi],
	    S[\co c\colon\psi\mapsto t]
	    $\enspace;
 \item[(write')]
	    $
	    [\comod c{\co c}, t\cdot\pi], S[\co
	    c\colon\psi\mapsto u]
	    \red
	    [\reader c, \pi],
	    S[\co c\colon\psi\mapsto u]
	    $\enspace;
 \item[(read)]$
	    [\reader c ,\pi],
	    S[c\colon\phi\mapsto u]
	    \red
	    [u,\pi],
	    S[c\colon\phi\mapsto u]
	    $\enspace; and
 \item[(fail)]
	    $
	    [\reader c,\pi],
	    S[c\colon\phi\mapsto \bot]
	    \red
	    \emptyset,S[c\colon\phi\mapsto \bot]
	    $
	    \enspace.
\end{description}
We denote the reflexive transitive closure of $\red$ as $\reduce$.


We can implement something similar to Lafont's example,
showing that confluence does not hold.
Suppose $\G\tr\tj t\phi$ and $\G\tr\tj u\phi$ are both derivable.
Then, by induction on the derivation,
both $\G,\tj x{\phi\limp\phi}\tr\tj{(x)t}\phi$
and $\G,\tj y{\phi\limp\phi}\tr\tj{(y)u}\phi$ are derivable
for $x$ and $y$ not appearing in $\G$, $u$, or $t$.
By the communication rule,
$\G\tr\tj{(\comod c{\co c})t\conc(\comod {\co c} c)u}\phi$ is derivable.
But $[(\comod c{\co c})t\conc (\comod{\co c} c)u,
\pi],S_\epsilon$
can reduce both to $[t,\pi], S_\epsilon[c\mapsto u, \co c\mapsto t]$
and to $[u,\pi], S_\epsilon[c\mapsto u,\co c\mapsto t]$.
Since $t$ and $u$ were taken arbitrarily, we cannot distinguish terms
of the same type if we equate a term with what it reduces to.


Another important property of an executable is
non-abortfullness
which states that the executable has no reduction sequence leading to
$\emptyset,S$.
The name comes from hyper-lambda calculus formulation in \fix{cite},
where a term sometimes turns into $\mathsf{abort}$ and there the property
stated that a hyperterm has no reduction sequence leading to a hyperterm
full of $\mathsf{abort}'s$.
This property is important because it ensures existence of
constructive content of proofs.

Non-abortfullness cannot be verified using poles
because any pole must have $\emptyset,S$ as an element for
any store~$S$.
This condition for poles is used in the
communication case of the proof of adequacy.
In order to prove non-abortfullness as a logical property,
we define another variant of poles called scheduled poles.


\fix{find a property that can only be verified using scheduled poles}.
\fix{maybe exact specification}


\newcommand{\sche}{\sqsubseteq}

\begin{proposition}[Non-abortfullness]
Let $\tr\tj t\psi$ be derivable, then,
 $[t,\epsilon], S_\epsilon$ has no run leading to $\emptyset, S'$ for
 any store $S'$.
\end{proposition}
\begin{proof}
\end{proof}


%%%%%%%% some texts

\section{Comparison with the Hypersequent Formulation}


\section*{where to put them}

``Undecidability of second order linear logic'' has
\begin{align*}
 \phi\otimes \psi&\equiv \forall X((\phi\limp\psi\limp X)\limp X) \\
 {1}      &\equiv \forall X(X\limp X)\\
 \exists X\phi   &\equiv \forall Y((\forall X(\phi\limp Y))\limp Y)
\end{align*}

Not all disjunctive tautologies are valid in G\"odel--Dummett logic:
$(P\limp Q)\lor(Q\limp R)$.

Non-abortfullness is not a pole.
In order to change this, we have to do something more subtle.
