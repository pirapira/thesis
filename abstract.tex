\begin{eabstract}
 We give a computational interpretation of G\"odel--Dummett logic, based
 on the Curry--Howard isomorphism.  This leads to characterization
 of waitfree computation, which is a class of distributed
 computation.  Although both G\"odel--Dummett logic and waitfree computation are
 important players in their respective areas, their connection has not
 been known.  G\"odel--Dummett logic is an intermediate logic---a
 logic between classical and intuitionistic logics.
 Waitfree computation is a class of distributed computation where no process can
 wait for another.  Our contribution is defining a typed lambda calculus
 for G\"odel--Dummett logic, which can solve
 exactly the same problems solvable by waitfree computation.
 This is the first result characterizing shared memory consistency using
 a type system, marking an important step in
 an emerging area of research to be called ``implicit shared
 memory consistency.''

 In his Master's Thesis, the author sensed some semantic connection
 between G\"odel--Dummett logic and a degree of shared memory consistency
 called sequential consistency.
 However, we are heavily indebted to Arnon
 Avron's paper in 1991.  Our typed lambda calculus is based on Avron's
 hypersequents.
 Our problem comes from Avron's question
 ``it seems to us extremely important to determine the exact
 computational content of them [the logics with cut-elimination
 using hypersequents] and to develop corresponding $\lambda$-calculi.''
 To this twenty-year-old question, we give an answer for a typical
 example of G\"odel--Dummett logic.
 We show that the computational content of G\"odel--Dummett logic is
 waitfree computation.

 Technically, this research studies a typed lambda calculus.
 We show three theorems:
 no typed term can reduce infinitely often;
 no typed term can reduce to a term representing failure;
 and a typed term can solve a problem if and only if a waitfree protocol
 can solve the problem.

 \fix{MTL and Haskell}
\end{eabstract}

\begin{jabstract}
 $B%+%j!<!&%O%o!<%IBP1~$K4p$E$$$?!$%2!<%G%k!&%@%a%C%HO@M}$N7W;;E*2r<a$rM?$((B
 $B$k!%$9$k$H!$J,;67W;;$NLdBj%/%i%9$G$"$k(Bwaitfree$B7W;;$rFCD'$E$1$i$l$k!%(B
 $B%2!<%G%k!&%@%a%C%HO@M}$b(Bwaitfree$B7W;;$b!$$=$l$>$l$NJ,Ln$G=EMW$G$"$k$N$K!$(B
 $B$3$l$i$N4XO"$O!$$3$l$^$GCN$i$l$F$$$J$+$C$?!%%2!<%G%k!&%@%a%C%HO@M}$OCf(B
 $B4VO@M}(B($BD>4Q<g5AO@M}$H8EE5O@M}$N4V$NO@M}(B)$B$N0l$D$G$"$k!%(BWaitfree$B7W;;$O!$%W%m%;%9$,B>$N(B
 $B%W%m%;%9$rBT$F$J$$$H$$$&@)8B$G7h$^$k!$J,;67W;;$N%/%i%9$G$"$k!%(B
 $BK\8&5f$N9W8%$O!$%2!<%G%k!&%@%a%C%HO@M}$N7?IU$-%i%`%@7W;;$rDj5A$7$F!$(B
 $B$A$g$&$I(Bwaitfree$B7W;;$G2r$1$kLdBj$@$1$r2r$1$k$H<($7$?$3$H$G$"$k!%(B
 $BK\8&5f$O!$(B
 $B6&M-%a%b%j$N0l4S@-$r!$7?%7%9%F%`$GFCD'$E$1$?=i$a$F$N8&5f$G$"$C$F!$(B
 $B6=$j$D$D$"$k!VHsL@<(%a%b%j0l4S@-!W$H8F$V$Y$-8&5fJ,Ln$N=EMW$J0lJb$G$"$k!%(B

 $BCx<T$O=$;NO@J8$G!$%2!<%G%k!&%@%a%C%HO@M}$H6&M-%a%b%j0l4S@-$N0l<o$G$"$k(B
 $BC`<!0l4S@-$H$N0UL#O@E*$J4XO"$K5$$E$$$F$$$?!%(B
 $B$7$+$7!$K\8&5f$N<jK!$bLdBj$b(B
 Arnon Avron$B$N(B1991$BG/$NO@J8$K$h$k$H$3$m$,Bg$-$$!%(B
 $B$^$:!$$=$NO@J8$G(BAvron$B$,Ds0F$7$?%O%$%Q!<%7!<%1%s%H$,!$(B
 $BK\8&5f$N7?IU$-%i%`%@7W;;$N4pAC$G$"$k!%(B
 $B$5$i$K!$$=$NO@J8$G(BAvron$B$,Ds5/$7$?LdBj(B
 $B!V%O%$%Q!<%7!<%1%s%H$G%+%C%H=|5n$G$-$kO@M}$N7W;;E*0UL#$O$J$K$+!W(B
 $B$,!$K\8&5f$GC55a$9$kLdBj$G$"$k!%(B
 $B$3$NFs==G/Mh$NLdBj$NE57?Nc$G$"$k%2!<%G%k!&%@%a%C%HO@M}$N>l9g$K!$2rEz$r(B
 $BM?$($k!%%2!<%G%k!&%@%a%C%HO@M}$N7W;;E*0UL#$O!$(Bwaitfree$B7W;;$G$"$k$H<($9!%(B

 $B5;=QE*$K!$K\8&5f$O7?IU$-%i%`%@7W;;$N8&5f$G$"$k!%(B
 $B<($7$?DjM}$O!$(B
 $B7?$,IU$$$?9`$N4JLs$,L58B$KB3$+$J$$$3$H$H!$(B
 $B7?$,IU$$$?9`$N4JLs$,<:GT$rI=$99`$G=*$o$i$J$$$3$H$H!$(B
 $B7?$,IU$$$?9`$G2r$1$kLdBj$,(Bwaitfree$B$K2r$1$kLdBj$H(B
 $B0lCW$9$k$3$H$N;0$D$G$"$k!%(B

 \fix{$B:G=*E*$K1Q8l$K9g$o$;$l$P$$$$(B}
\end{jabstract}
