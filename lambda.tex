\chapter{A Hyper $\lambda$-Calculus for G\"odel--Dummett Logic}

\subsection{Summary}

We propose a typed lambda calculus based on Avron's hypersequent
calculus for G\"odel--Dummett logic.  This calculus turns out to model
waitfree computation.
Besides strong normalization and non-abortfullness, 
we give soundness and completeness of
the calculus against the typed version of waitfree protocols.
The calculus is not only proof theoretically interesting,
but also valuable as a basis for distributed programming languages.

The Curry--Howard isomorphism~\cite{curryhoward} is surprising because the same
method works for two different purposes: a logical purpose of
removing redundancy from proofs and a computational purpose of finding a
class of terminating programs.
We extend
this surprise to G\"odel--Dummett logic and
waitfreedom.
G\"odel--Dummett logic~\cite{dummett59}
is one of the intermediate logics
between classical and intuitionistic logics.
Waitfreedom~\cite{Herlihy88,Saks:1993vq} is a class of distributed
computation without synchronization among processes.

We connect G\"odel--Dummett logic and waitfreedom using
Avron's hypersequent calculus~\cite{avron91}.
We respond to his suggestion:
\begin{quote}
it seems to us extremely important to determine the exact
       computational content of them~[intermediate logics] ---
       and {to develop corresponding `$\lambda$-calculi'}
       ---Avron~\cite{avron91}. 
\end{quote}
Differently from intuitionistic logic, G\"odel--Dummett logic validates
all formulae of the form
 $(\varphi\supset\psi)\vee(\psi\supset\varphi)$.
Our aim is building a typed lambda calculus
with some terms witnessing those formulae.
Such a term
$\tj{M}{(\varphi\supset\psi)\vee(\psi\supset\varphi)}$ must choose 
$M\reduction \linl\cdots$ or $M\reduction \linr\cdots$.
We devise a nondeterministic $\lambda$-calculus in Sect.~\ref{lgd}.

Waitfreedom is a class of distributed computation where
processes cannot wait for other processes.  When two processes try to
exchange information, the faster process can pass information to the
slower one, but not always vice versa because the slower process might
start after the faster one finishes.
So, the computation is nondeterministic.
The contribution of this paper is capturing
this nondeterminism using the nondeterministic $\lambda$-calculus for
G\"odel--Dummett logic: in Sect.~\ref{comparison}, we show that the
$\lambda$-terms in the calculus can solve a typed input-output
problems if and only if it is waitfreely solvable.

\section{\lgd}
\label{lgd}

We first present a proof system for G\"odel--Dummet logic.
Then we turn the proof system into typing rules for $\lambda$-terms
of~\lgd, give a set of reductions and prove strong-normalization and
non-abortfullness.

\subsection{Logic}

\newcommand{\m}[1]{{#1}^+}

Let us assume a countably infinite set of \textit{propositional variables}.
We define \textit{local formulae} $\varphi, \psi$ by the following BNF,
where $I$ is a
propositional variable%
\footnote{We include $\bot$ because G\"odel--Dummett logic has it
although $\bot$ is not necessary for us to encode waitfree computation.}%
:
\[
 \varphi,\psi ::= \bot \mid I \mid (\varphi\supset\psi) \mid (\varphi\wedge\psi) \mid
 (\varphi\vee\psi)\enspace.
\]
Further, we define \textit{global formulae} $\m\varphi, \m\psi$ with the following
BNF:
\[
 \m\varphi,\m\psi :: = [i]\varphi \mid
 (\m\varphi\wedge\m\psi)\mid (\m\varphi\vee\m\psi)
\]
where $i$ is a natural number (representing a process).  The unary operators $[0],
[1],\ldots$ are called \textit{modalities}.
We omit parentheses following the usual convention.
Informally, the local formulae describe datatypes used by each process.
The global formulae describe inputs or outputs of all
processes together.

A \textit{context} (denoted by $\Gamma$ and $\Delta$ possibly
subscripted) is a potentially empty
finite sequence of global formulae.
A \textit{sequent}~$\Gamma\vdash\m\varphi$ is a pair of a context and a
global formula.
A \textit{hypersequent} is a finite sequence of sequents.

The underlying logic has the derivation rules in Fig.~\ref{fig:logic}.  If
we omit all the modalities, these rules characterize
G\"odel--Dummett logic.
However, the modalities have at least some sense: while
$([0]\varphi\supset[0]\psi)\vee([1]\psi\supset[1]\varphi)$ is provable,
$([0]\varphi\supset[1]\psi)\vee([0]\psi\supset[1]\varphi)$ is not.
A semantics will be given in Sect.~\ref{comparison}.

\begin{figure}
 \small
  \begin{center}
  \textbf{External Rules}
   \vskip 2mm
%%communication
   \BinaryRule
   {$\hyper\hmid\Gamma,\Delta\vdash[i]\varphi$}
   {$\hyper\hmid\Gamma,\Delta\vdash[j]\psi$}
   {com'}
   {$\hyper\hmid\Gamma\vdash[i]\psi\hmid\Delta\vdash[j]\varphi$}
  \hfill
%% external structural   
 \UnaryRule
 {$\hyper^+$}
 {EW}
 {$\hyper^+\hmid\Gamma\vdash \m\varphi$}
 \vskip 2mm
 \UnaryRule
 {$\hyper\hmid\Gamma\vdash \m\varphi\hmid\Gamma\vdash \m\varphi$}
 {EC}
 {$\hyper\hmid\Gamma\vdash \m\varphi$}
 \hfill
 \UnaryRule
 {$\hyper\hmid\Gamma\vdash \m\varphi\hmid\Delta\vdash \m\psi\hmid \hyper'$}
 {EE}
 {$\hyper\hmid\Delta\vdash \m\psi   \hmid\Gamma\vdash \m\varphi\hmid
   \hyper'$}
 \vskip 2mm
\textbf{Inner Global Rules}
\vskip 2mm
%% structural
   \UnaryRule{$\hyper\hmid\Gamma,\m\varphi,\m\psi,\Delta\vdash\m\theta$}
   {IE}
   {$\hyper\hmid\Gamma,\m\psi,\m\varphi,\Delta\vdash \m\theta$}
   \hfill
   \UnaryRule{$\hyper\hmid\Gamma\vdash\m\varphi$}
   {IW}
   {$\hyper\hmid\m\psi,\Gamma\vdash\m\varphi$}
   \hfill
   \UnaryRule{$\hyper\hmid\m\psi,\m\psi,\Gamma\vdash\m\varphi$}
   {IC}
   {$\hyper\hmid\m\psi,\Gamma\vdash\m\varphi$}
   \vskip 2mm
%% local conj intro
  \BinaryRule
   {$\hyper\hmid\Gamma\vdash \m\varphi$}
   {$\hyper\hmid\Gamma\vdash \m\psi$}
   {$\wedge\intro$}
   {$\hyper\hmid\Gamma\vdash \m\varphi\wedge\m\psi$}
   \vskip 2mm
%% local conj elim
  \UnaryRule
   {$\hyper\hmid\Gamma\vdash \m\varphi\wedge\m\psi$}
   {$\wedge\elim_0$}
   {$\hyper\hmid\Gamma\vdash \m\varphi$}
   \hfill
  \UnaryRule
   {$\hyper\hmid\Gamma\vdash\m\varphi\wedge\m\psi$}
   {$\wedge\elim_1$}
   {$\hyper\hmid\Gamma\vdash\m\psi$}
\vskip 2mm
%% local disj intro
  \UnaryRule
   {$\hyper\hmid\Gamma\vdash\m\varphi$}
   {$\vee\intro_0$}
   {$\hyper\hmid\Gamma\vdash\m\varphi\vee\m\psi$}
   \hfill
  \UnaryRule
   {$\hyper\hmid\Gamma\vdash\m\psi$}
   {$\vee\intro_1$}
   {$\hyper\hmid\Gamma\vdash\m\varphi\vee\m\psi$}
\vskip 2mm
%% local disj elim
   \TrinaryRule
   {$\hyper\hmid\Gamma\vdash           \m\varphi\vee\m\psi$}
   {$\hyper\hmid\m\varphi, \Gamma\vdash\m\theta$}
   {$\hyper\hmid\m\psi,    \Gamma\vdash\m\theta$}
   {$\vee\elim$}
   {$\hyper\hmid         \Gamma\vdash\m\theta$}
\vskip 2mm
\textbf{Inner Local Rules}
\vskip 2mm
%% axiom
  \UnaryRule{}{$[i]$Ax}
   {$[i]\varphi,\Gamma\vdash [i]\varphi$}
   \hfill
%% bot elim   
 \UnaryRule{$\hyper\hmid\Gamma\vdash[i]\bot$}
   {$[i]\bot\elim$}
   {$\hyper\hmid\Gamma\vdash[i]\varphi$}
   \vskip 2mm
%% local imp intro
  \UnaryRule{$\hyper\hmid[i]\varphi,\Gamma\vdash [i]\psi$}
  {$[i]\supset\intro$}
  {$\hyper\hmid\Gamma\vdash [i](\varphi\supset \psi)$}
  \hfill
%% local imp elim   
  \BinaryRule
  {$\hyper\hmid\Gamma\vdash [i](\varphi\supset\psi)$}
  {$\hyper\hmid\Gamma\vdash [i]\varphi$}
  {$[i]\supset\elim$}
  {$\hyper\hmid\Gamma\vdash [i]\psi$}
   \vskip 2mm
%% local conj intro
  \BinaryRule{$\hyper\hmid\Gamma\vdash [i]\varphi$}
   {$\hyper\hmid\Gamma\vdash [i]\psi$}
   {$[i]\wedge\intro$}
   {$\hyper\hmid\Gamma\vdash [i](\varphi\wedge\psi)$}
   \vskip 2mm
%% local conj elim
  \UnaryRule{$\hyper\hmid\Gamma\vdash [i](\varphi\wedge\psi)$}
   {$[i]\wedge\elim_0$}
   {$\hyper\hmid\Gamma\vdash[i]\varphi$}
   \hfill
  \UnaryRule{$\hyper\hmid\Gamma\vdash[i](\varphi\wedge\psi)$}
   {$[i]\wedge\elim_1$}
   {$\hyper\hmid\Gamma\vdash[i]\psi$}
\vskip 2mm
%% local disj intro
  \UnaryRule
   {$\hyper\hmid\Gamma\vdash[i]\varphi$}
   {$[i]\vee\intro_0$}
   {$\hyper\hmid\Gamma\vdash[i](\varphi\vee\psi)$}
   \hfill
  \UnaryRule{$\hyper\hmid\Gamma\vdash[i]\psi$}
   {$[i]\vee\intro_1$}
   {$\hyper\hmid\Gamma\vdash[i](\varphi\vee\psi)$}
\vskip 2mm
%% local disj elim
   \TrinaryRule
   {$\hyper\hmid\Gamma\vdash[i](\varphi\vee\psi)$}
   {$\hyper\hmid[i]\varphi, \Gamma\vdash[i]\theta$}
   {$\hyper\hmid[i]\psi,    \Gamma\vdash[i]\theta$}
   {$[i]\vee\elim$}
   {$\hyper\hmid         \Gamma\vdash[i]\theta$}
\vskip 2mm
\end{center}
\caption[The underlying logic \fix{of what}.]
 {The underlying logic.
 Metavariables~$i$ and $j$ stand for a process.
 $\hyper$ stands for a
 hypersequent.
 $\hyper^+$ stands for a nonempty hypersequent.
 $\Gamma$ and $\Delta$ stand for possibly empty contexts.
 The most important com' rule is based on a rule by
 Avron~\cite{avron96}.
 }
\label{fig:logic}
\end{figure}

\subsection{Term Assignment}
\label{term}

We assume distinct, countably infinite sets of \textit{variables},
\textit{locations}
and
\textit{processes}.
Locations are denoted by~$l$; process~$i,j, \ldots$ and variables~$x,
y, \ldots$.
Later, locations will be used to specify a pair of stores
holding terms.  
Like in the $\lambda$-calculus, some terms reduces to other
terms, but in this calculus, terms may interact with the store (like
a program written in Haskell or OCaml does with an i-var).
This behavior will be shown later in the definition of reductions.

We define \textit{terms}~$\term$ by the BNF
where $\Gamma$ is a sequence of variables:
\begin{align*}
\term ::=&\,
 x\mid \ltor i l \Gamma \term \mid \rtol i l \Gamma \term \mid
 \gpair {\term,\term}\mid
 \gpil \term \mid \gpir \term \mid
 \ginl\term \mid
 \ginr\term \mid \\ & \gmat \term x\term y \term \mid 
 \cotuple{\term,\term} \mid \abort \mid
  \lpil \term \mid \lpir \term \mid
 \lpair {\term,\term}\mid \\&
  \linl \term \mid  \linr \term \mid
 \lambda x.\term\mid (\term \term)
\mid \lmat \term x\term y \term \enspace.
\end{align*}
All variable occurrences (including those in $\Gamma$)
except the first clause are
binding.
The constructs with a~$\g$ represent the global rules and the
constructs without a~$\g$ represent the local rules.

We extend a \textit{sequent}
to $\Gamma\tr \tj M{\m \varphi}$, where $\Gamma$ is
a sequence like $\tj{x}{\m\psi}, \tj{y}{\m\theta}$ and $M$ is a term.
In a sequent $\Gamma\tr\tj{M}{\m\varphi}$, we require the variables in
$\Gamma$ to be distinct from each other.
A \textit{contexted type}
 $\Gamma\tr\m\varphi$ is a sequent without a term but with variables in
 $\Gamma$.
A \textit{hypersequent} is a finite sequence of sequents (called
\textit{components})
where the same
variable has the same type even if it appears in different components.
The typing rules for the terms are given in Fig.~\ref{termassign}.

\begin{figure}[p]
 \small
  \begin{center}
\textbf{External Rules}
\vskip 2mm
%% communication
\BinaryRule
   {$\hypert_0\hmid\Gamma,\Delta\tr\tj{M}{[i]\varphi}$}
   {$\hypert_1\hmid\Gamma,\Delta\tr\tj{N}{[j]\psi}$}
   {}
   {$\cotuple{\hypert_0,\hypert_1}\hmid\Gamma\tr\tj{\ltor i{l}\Delta{M}}{[i]\psi}\hmid
   \Delta\tr\tj{\rtol j{l}\Gamma {N}}{[j]\varphi}$}
\hfill
%% external structural   
 \UnaryRule
 {$\hypert^+$}
 {}
 {$\hypert^+\hmid\Gamma\tr \tj \abort {\m\varphi}$}
\vskip 2mm
 \UnaryRule
 {$\hypert\hmid\Gamma\tr\tj M {\m\varphi}\hmid\Gamma\tr \tj N {\m\varphi}$}
 {}
 {$\hypert\hmid\Gamma\tr \tj {\cotuple{M,N}}{\m\varphi}$}
\hfill
 \UnaryRule
 {$\hypert\hmid\Gamma\tr M\colon\m\varphi\hmid\Delta\tr N\colon \m\psi\hmid \hypert'$}
 {}
 {$\hypert\hmid\Delta\tr N\colon\m\psi   \hmid\Gamma\tr M\colon \m\varphi\hmid \hypert'$}
 \vskip 2mm
\textbf{Inner Global Rules}
   \vskip 2mm
   \UnaryRule
   {$\hypert\hmid \Gamma\tr\tj M {\m\varphi}$}
   {}
   {$\hypert\hmid \tj x {\m\psi}, \Gamma\tr \tj{M}{\m\varphi}$}
   \hfill
   \UnaryRule
   {$\hypert\hmid \tj{x}{\m\varphi}, \tj{y}{\m\varphi}, \Gamma\tr \tj
   M{\m\psi}$}
   {}
   {$\hypert \hmid \tj{x}{\m\varphi}, \Gamma\tr \tj{M[x/y]}{\m\psi}$}
\vskip 2mm
\UnaryRule
   {$\hypert\hmid\Gamma,\tj x{\m\varphi},\tj y{\m\psi},\Delta\tr
   \tj M{\m\theta}$}{}
   {$\hypert\hmid\Gamma,\tj y{\m\psi},\tj x{\m\varphi},\Delta\tr \tj M{\m\theta}$} %IE
\hfill
%% conj intro
\BinaryRule
   {$\hypert_0\hmid\Gamma\tr \tj M{\m\varphi}$}
   {$\hypert_1\hmid\Gamma\tr \tj N{\m\psi}$}
   {}
   {$\cotuple{\hypert_0,\hypert_1}\hmid\Gamma\tr
     \tj{\gpair{M,N}}{\m\varphi\wedge\m\psi}$}
   \vskip 2mm
%% conj elim
  \UnaryRule
   {$\hypert\hmid\Gamma\tr M\colon\m\varphi\wedge\m\psi$}
   {}
   {$\hypert\hmid\Gamma\tr \gpil{M}\colon\m\varphi$}
   \hfill
  \UnaryRule
   {$\hypert\hmid\Gamma\tr M\colon\m\varphi\wedge\m\psi$}
   {}
   {$\hypert\hmid\Gamma\tr\gpir M\colon\m\psi$}
\vskip 2mm
%% disj intro
  \UnaryRule
   {$\hypert\hmid\Gamma\tr M\colon\m\varphi$}
   {}
   {$\hypert\hmid\Gamma\tr\ginl{M}\colon\m\varphi\vee\m\psi$}
   \hfill
  \UnaryRule
   {$\hypert\hmid\Gamma\tr M\colon\m\psi$}
   {}
   {$\hypert\hmid\Gamma\tr \ginr{M}\colon\m\varphi\vee\m\psi$}
\vskip 2mm
%% disj elim
\TrinaryRule
   {$\hypert_0\hmid\Gamma\tr \tj M {\m\varphi\vee\m\psi}$}
   {$\hypert_1\hmid\tj x{\m\varphi},\Gamma\tr \tj {N_0}{\m\theta}$}
   {$\hypert_2\hmid\tj y{\m\psi},   \Gamma\tr \tj {N_1}{\m\theta}$}
   {}
   {$\cotuple{\cotuple{\hypert_0,\hypert_1},\hypert_2}\hmid\Gamma \tr \tj{\gmat M x {N_0} y {N_1}}{\m\theta}$}
\vskip 2mm
\textbf{Inner Local Rules}
\vskip 2mm
%% axiom   
  \UnaryRule{}{}{$\tj x{[i]\varphi}, \Gamma\tr \tj x{[i]\varphi}$}
                       \hfill
%% bot elim   
 \UnaryRule
   {$\hypert\hmid\Gamma\tr \tj M{[i]\bot}$} {}
   {$\hypert\hmid\Gamma\tr \tj{\abort}{[i]\varphi}$}
   \vskip 2mm
%% local imp intro
  \UnaryRule
   {$\hypert\hmid\tj x{[i]\varphi},\Gamma\tr \tj M{[i]\psi}$}
   {}
   {$\hypert\hmid\Gamma\tr \tj{\lambda x.M}{[i](\varphi\supset\psi)}$}
                       \hfill
%% local imp elim
  \BinaryRule{$\hypert_0\hmid\Gamma \tr \tj M{[i](\varphi\supset \psi)}$}
  {$\hypert_1\hmid\Gamma\tr \tj N{[i]\varphi}$}
  {}
  {$\cotuple{\hypert_0,\hypert_1}\hmid\Gamma\tr \tj{MN}{[i]\psi}$}
                       \vskip 2mm
%% local conj intro
\BinaryRule{$\hypert_0\hmid\Gamma\tr \tj M{[i]\varphi}$}
   {$\hypert_1\hmid\Gamma\tr \tj N{[i]\psi}$}
   {}
   {$\cotuple{\hypert_0,\hypert_1}\hmid\Gamma\tr
     \tj{\lpair {M,N}}{[i](\varphi\wedge\psi)}$}
   \vskip 2mm
%% local conj elim
  \UnaryRule
   {$\hypert\hmid\Gamma\tr M\colon[i](\varphi\wedge\psi)$}
   {}
   {$\hypert\hmid\Gamma\tr \lpil{M}\colon[i]\varphi$}
   \hfill
  \UnaryRule
   {$\hypert\hmid\Gamma\tr M\colon[i](\varphi\wedge\psi)$}
   {}
   {$\hypert\hmid\Gamma\tr\lpir M\colon[i]\psi$}
\vskip 2mm
%% local disj intro
  \UnaryRule
   {$\hypert\hmid\Gamma\tr M\colon[i]\varphi$}
   {}
   {$\hypert\hmid\Gamma\tr\linl {M}\colon[i](\varphi\vee\psi)$}
   \hfill
  \UnaryRule
   {$\hypert\hmid\Gamma\tr M\colon[i]\psi$}
   {}
   {$\hypert\hmid\Gamma\tr \linr{M}\colon[i](\varphi\vee\psi)$}
\vskip 2mm
%% local disj elim
\TrinaryRule
   {$\hypert_0\hmid\Gamma\tr \tj M {[i](\varphi\vee\psi)}$}
   {$\hypert_1\hmid\tj x{[i]\varphi},\Gamma\tr \tj {N_0}{[i]\theta}$}
   {$\hypert_2\hmid\tj y{[i]\psi},   \Gamma\tr \tj {N_1}{[i]\theta}$}
   {}
   {$\cotuple{\cotuple{\hypert_0,\hypert_1},\hypert_2}\hmid\Gamma \tr \tj{\lmat M x {N_0} y {N_1}}{[i]\theta}$}
\vskip 2mm
\end{center}
 \caption[Term assignment.\fix{of what}]
 {Term assignment.
 $\hypert$ stands for
 a possibly empty hypersequent (with possible subscripts).
 $\hypert^+$ stands for a non-empty hypersequent.
 Within each rule, $\hypert_0$, $\hypert_1$ and $\hypert_2$ have the
 same length and the same type so that $\cotuple{\hypert_0,\hypert_1}$
 can be defined as
 the elementwise application of $\cotuple{\term,\term}$.
 }
 \label{termassign}
\end{figure}

\subsection{Reduction}

A term~$M$ is \textit{of} type~$\m\varphi$ iff there is a derivation of
$\Gamma\tr
\tj{M}{\m\varphi}$.
A \textit{local} term is a term without $\overleftarrow l$,
$\overrightarrow l$ or any $\g$ constructs.
A \textit{hyperterm}~$\hypert$ is a nonempty sequence of terms.
A \textit{store} maps a location to a
local term or $\epsilon$.
For a store~$\sigma$, the updated store $\sigma[l\mapsto x]$ maps $l$ to
$x$ and $l'$ to $\sigma(l')$ if $l'$ is different from~$l$.
A \textit{configuration} is a triple $\conf{}{}\hypert$ of two
stores $\lstore, \rstore$ and a hyperterm~$\hypert$.

To complete the definition of \lgd,
 we define the \textit{reductions} $\reduce_\spadesuit$ of
 configurations for $\spadesuit\in\{\mathrm B, \mathrm W, \mathrm R, \mathrm A,
 \mathrm P\}$.
 We consider terms up to $\alpha$-equivalence and implicitly
 require all instances
 of $\rightsquigarrow_\spadesuit$ to avoid free variable captures.
 Below, $\square$ and $\blacksquare$ match $\g$ or nothing.

\begin{definition}[Basic Reduction]
 The basic reduction $\breduce$ is the smallest congruence containing
 the followings:
 \begin{itemize}
  \item  $\conf{}{}{(\lambda x.M)O}\breduce
 \conf{}{}{M[O/x]}$
  \item $\conf{}{}{\spil{\spair{M,N}}} \breduce
	 \conf{}{}{           M   }$
  \item $\conf{}{}{\spir{\spair{M,N}}} \breduce
	 \conf{}{}{             N }$
  \item $\conf{}{}{\smat{\sinl M}x N y O} \breduce
	 \conf{}{}{              N[M/x]}$
  \item $\conf{}{}{\smat{\sinr M}x N y O} \breduce
	 \conf{}{}{                  O[M/y]}$
 \end{itemize}
\end{definition}

There are two sorts of reductions that interact with the store.
In summary, $\rtol i l \Gamma N$ tries to write to the right store and read from
the left store of the configuration.
If a term tries to read from an empty location of a store,
the term changes into $\abort$.  If a term writes to a full location of
a store, it does not abort but the store is not updated.
The formal definition follows.
\begin{definition}[Write Reduction]
 The write reduction $\wreduce$ is the smallest congruence
 containing the followings where $M$ is a local term:
 \begin{itemize}
  \item $\conf{}{[l\mapsto\epsilon]}{\rtol i l \Gamma M}
	\wreduce
	\conf{}{[l\mapsto M]}{\rtol i l \Gamma M}
	$\enspace,
  \item $\conf{[l\mapsto\epsilon]}{}{\ltor j l \Delta M}
	\wreduce
	\conf{[l\mapsto M]}{}{\ltor j l \Delta M}
	$\enspace.
 \end{itemize}
\end{definition}

\begin{definition}[Read Reduction]
 \label{read}
 The read reduction $\rreduce$ is the smallest congruence
 containing the
 followings:
\begin{itemize}
 \item  $\conf{[l\mapsto {N}]}{[l\mapsto M']}{\rtol i l\Gamma M}\rreduce
	\conf{[l\mapsto {N}]}{[l\mapsto M']}{N}$
 \item  $\conf{[l\mapsto \epsilon]}
	{[l\mapsto M']}{\rtol i l\Gamma M}\rreduce
	\conf{[l\mapsto \epsilon]}{[l\mapsto M']}{\abort}$
 \item  $\conf{[l\mapsto M']}{[l\mapsto {N}]}{\ltor i l \Delta M}\rreduce
	\conf{[l\mapsto M']}{[l\mapsto {N}]}{N}$
 \item  $\conf{[l\mapsto M']}{[l\mapsto \epsilon]}{\ltor i l \Delta M}\rreduce
	 \conf{[l\mapsto M']}{[l\mapsto \epsilon]}{\abort}$\enspace .
\end{itemize}
\end{definition}

\begin{definition}[Abort Propagation Reduction]
 The abort propagation reduction $\areduce$ is the smallest
 congruence containing the
 followings:
\begin{itemize}
 \item  $\conf{}{}{\cotuple{\abort, M}}\areduce
 \conf{}{}{M}$, and
   $\conf{}{}{\cotuple{M,\abort}}\areduce
 \conf{}{}{M}$
 \item  $\conf{}{}{C[\abort]}\areduce
 \conf{}{}{\abort}$
\end{itemize}
 where $C[\bullet]$ is 
$\bullet N$, 
${M \bullet}$,
$\ltor i l \Delta \bullet$,
$\rtol i l \Gamma \bullet$,
$\sinl \bullet$,
$\sinr \bullet$,
$\spair {\bullet, N}$,
$\spair {M, \bullet}$,
$\pi^\square_i \bullet$.
$\smat M x N y \bullet$,
$\smat  \bullet x N y O$, or  \\
$\smat  M x \bullet y O$ 
\end{definition}
In order to obtain subformula property
 via proof normalization
we add yet another kind of reduction rules.
\begin{definition}[Permutative Reduction]
 The permutative reduction~$\preduce$ is the smallest congruence
 containing the followings:
\begin{itemize}
 \small
 \item $\conf{}{}{ \left(\smat  M x N y O\right) P }\preduce$ \\
       $\conf{}{}{ \smat M x {N P} y {O P} }$
 \item $\conf{}{}{ \pi^\square_d \left(\bsmat M x N y
       O\right)}\preduce$\\
       $\conf{}{}{ \bsmat M x
       {\pi^\square_d N} y {\pi^\square_d O} }$
 \item {
       $\conf{}{}{ \smat
                          {\left(\bsmat  M x N y O\right)}
                          u P v Q
                      }\preduce$ \\
       $(\lstore, \rstore, 
        \mathsf{match}^{\blacksquare}\,{M}\,\mathsf{of}\, \bsinl{x}. {
                          {\left(\smat N u P v Q\right)}
       } /$ \\ \phantom{mmmmmmmmmmm}$
       \bsinr{y}. {\left(\smat  O u P v Q\right)}
                      )
       $}
 \item $\conf{}{}{ \cotuple{M, N} P }\preduce
        \conf{}{}{ \cotuple{MP, NP} }$
 \item $\conf{}{}{ \pi^\square_d\cotuple{M,N} }\preduce
        \conf{}{}{ \cotuple{\pi^\square_d M, \pi^\square_d N} }$
 \item $\conf{}{}{ \smat {\cotuple{M,N}} x P y Q }\preduce$\\
       $\conf{}{}{ \cotuple{
                          \smat  M x P y Q,
                          \smat N x P y Q
                        } }$
\end{itemize}
\end{definition}

We define $\reduce$ to be the union of $\breduce$, $\wreduce$, $\rreduce$,
$\areduce$ and $\preduce$.
The reflexive transitive closure of $\reduce$ is
written as~$\reduction$.
A \textit{redex} is a subterm that can be rewritten by a reduction.
A configuration~$\mathcal{C}$ is \textit{normal} when there is no configuration
$\mathcal{D}$ with $\mathcal{C}\reduce \mathcal{D}$.
A term~$M$ is \textit{normal} when the configuration $\conf{}{}{M}$ is
normal (the choice of $\lstore$ and $\rstore$ is irrelevant).

\subsection{Properties}

An important property of
\lgd is \textit{strong normalization}:
every typed hyperterm has a maximal number of reductions it can
take.
Another is \textit{non-abortfullness}: although some reductions yield
$\abort$ terms, a typed hyperterm never reduces to a hyperterm that only
contains $\abort$'s.  We show this first for its simpler proof.

\begin{theorem}[Non-abortfullness]
 \label{nab}
 All normal forms of a typed configuration contain at least one term
 that is not $\abort$.
\end{theorem}
\begin{proof}
 When a reduction sequence is fixed, for any location~$l$, depending on
 whether the right or the left store is filled first, 
 either:
 (i) no ${\ltor i l \Delta M} \reduce {\abort}$ occurs for any~$i$, or
 (ii) no ${\rtol i l\Gamma M} \reduce {\abort}$ occurs for any~$i$.

If the former is the case, we can rewrite
a communication rule occurrence
\begin{center}
 \BinaryRule
 {$\hypert_0\hmid  \Gamma,\Delta\tr \tj M{[i]\psi}$}
 {$\hypert_1\hmid \Gamma,\Delta\tr \tj N{[j]\tau}$}
 {}
 {$\cotuple{\hypert_0,\hypert_1}\hmid \Gamma\tr \tj
   {\ltor i l \Delta M}{[i]\tau}\hmid
   \Delta\tr \tj{\rtol i l\Gamma N}{[j]\psi}$}
\end{center}
into successive external weakening occurrences
\begin{center}
 \AxiomC
 {$\hypert_0\hmid  \Gamma^j, \Delta^j\tr \tj M{[j]\psi}$}
\doubleLine
 \UnaryInfC
 {$\hypert_0\hmid \tr \tj \abort
 {[i]\tau}\hmid
   \Gamma^j,\Delta^j\tr \tj{M}{[j]\psi}$}
 \DisplayProof
\end{center}
 where $\Gamma^j$ and $\Delta^j$ can be obtained from the originals
 by changing every
 modality~$[k]$ to $[j]$.
In the lattar case, we can do the symmetric.

After these rewritings for all appearing locations,
we obtain a derivation not containing
$\overrightarrow{l}$ or $\overleftarrow{l}$.
Moreover, the end hypersequent of the resulting derivation has a component
not containing $\abort$.
The reductions of the original hyperterm can be simulated by the
resulting hyperterm.  And, even after reductions, the resulting
hyperterm has a component not containing $\abort$.
\end{proof}

\begin{theorem}[Strong Normalization]
 \lgd\, is strongly normalizing.
\end{theorem}
\begin{proof}
For proving this, we consider the \textit{local fragment} that does not contain
$\ltor i l \Delta M$, $\rtol j l\Gamma N$ or any construct with $\g$.
We first reduce the strong
normalization of the \lgd\, to that of the local fragment, and 
ultimately to that of de Groote's
natural deduction with permutation-conversion~\cite{Philippe2002js}%
\footnote{
To the 
same effect, we might be able to use other strong normalization
 results for lambda calculi with commutative conversions, like Balat,
 Di~Cosmo
 and Fiore~\cite{bdf}.
}%
.

We assume an infinite sequence of reductions
$
\conf{_0}{_0}{\hypert_0}
\reduce
\conf{_1}{_1}{\hypert_1}
\reduce
\conf{_2}{_2}{\hypert_2}
\reduce\cdots 
$.  From this, we are going to construct an infinite sequence of
reductions in the local fragment.

For that, we first
build an infinite reduction sequence with constant stores. 
Using the original infinite sequence, we define a pair of stores called the
\textit{store prophecy} $(\lstore_\infty, \rstore_\infty)$ where
$ \lstore_\infty(l)= \epsilon$ if $\lstore_k(l)=\epsilon$ for all
 $k\in\omega$ and
$ \lstore_\infty(l)=M $ if $\lstore_k(l)=M$ for some $k\in\omega$
; and $\rstore_\infty(l)$ is symmetrically defined.
Since store contents are never overwritten,
$\lstore_\infty$ and $\rstore_\infty$ are well-defined.
Moreover,
$\lstore_i(l)$ and $\lstore_\infty(l)$ coincide unless
$\lstore_i(l)=\epsilon$.

We build another reduction sequence
$
\conf{_\infty}{_\infty}{\hypert_0}
\reduce^\ast
\conf{_\infty}{_\infty}{\hypert'_1}
\reduce^\ast
\conf{_\infty}{_\infty}{\hypert'_2}
\reduce^\ast\cdots
$
with the following invariant:
$\mathcal M'_i$ can be obtained by replacing some $\abort$ occurrences
in $\mathcal M_i$ with some terms.
More specifically, we translate each reduction as follows, keeping the
invariant inductively on the number of steps
(the base case is satisfied by $\mathcal M'_0 = \mathcal M_0$ immediately):
\begin{itemize}
 \item a read reduction $\conf{_k}{_k}{\hypc{}{\rtol i l\Delta M}}
       \rreduce
       \conf{_{k+1}}{_{k+1}}{\hypc{}{O}}$ is translated into
       $\conf{_\infty}{_\infty}{\hypc'{\rtol i l\Delta M}} \rreduce
       \conf{_{k+1}}{_{k+1}}{\hypc'{O'}}$.
       If $\lstore_i(l)$ is a term,
       $\lstore_\infty(l)$ and $O'$ are also identical to the term.
       Otherwise, $O'$ must be $\abort$.
       Thus, the invariant
       holds for $k+1$.
 \item a write reduction disappears;
 \item an $\abort$ propagation
       $\hypc{}{C[\abort]} \areduce \hypc{}{\abort}$ can be translated
       either to a similar reduction or no reduction if the $\abort$ in
       the redex is replaced by another term in the $\mathcal{M'}_k$.
       Note that even in that case, the result $\mathcal{M'}_{k+1}$ can
       be obtained by replacing some $\abort$ occurrences in
       $\mathcal{M}_{k+1}$ with other terms;
 \item any other reduction $\conf{_k}{_k}{\hypc{}{M}} \bpreduce
       \conf{_{k+1}}{_{k+1}}{\hypc{}{N}}$
       is translated into one similar reduction
       $\conf{_\infty}{_\infty}{\hypc{'}{M'}}\bpreduce
        \conf{_\infty}{_\infty}{\hypc{'}{N'}}$.
\end{itemize}
Here, we have to show that the translated sequence is infinite.
For that, we can use the facts that
 there are only finite
number of usedlocations each of which allows only one write, and that
an $\abort$ propagation always
strictly shortens the term under operation. 
 
 After that, we can replace
 every ${\rtol i l\Gamma M}$ with
 $\rstore_\infty(l)$.
 Since $\rtol i l\Gamma M$ either reduces to $\rstore_\infty(l)$ or $\abort$,
 replacing it with $\rstore_\infty(l)$ will only ``shorten'' the reduction
 sequence for at most one read step.
 We can do the same for $\ltor j l \Gamma N$.
 Replacing every such occurrences
 makes an infinite reduction sequence where every occurring term is
 in the local fragment.
 Moreover, 
 the result of the translation is also well-typed.
 A typing derivation of the resulting hyperterm can be obtained by
 replacing com' rules with EW rules and changing the process number in
 types of some variables (c.f. the proof of Thm.~\ref{nab}).

We are aiming at reducing the problem to the strong normalization result
by de Groote~\cite{Philippe2002js}.
Since we have eliminated $\ltor i l \Delta M$ or $\rtol i l\Gamma N$ occurrences,
the remaining difference is small: some $\abort$ propagation reductions
 and some permutative reductions involving $\cotuple{\term,\term'}$.
We just have to make sure that there are no infinite reduction sequences
that consist of these two kinds of reductions only.
We can deal with the permutative reductions following de
 Groote~\cite{Philippe2002js}'s strategy for introducing~$\bot$.
For $\abort$ propagation, we can apply the previous argument.
\end{proof}

\section{Typed Waitfreedom}
\label{waitfreedom}

Waitfreedom~\cite{Herlihy88,Saks:1993vq} is a class of protocols
that can solve
some of the input-output problems~\cite{Moran:1987ep,Biran:1988hh}.
We define the typed version of waitfreedom.

\subsection{Typed Input-Output Problem}

Saks and Zaharoglou~\cite{Saks:1993vq} formulated waitfreedom as a class
of input-output
problems.
Given inputs for all processes and outputs of all
processes, an input-output problem decides whether the processes have
succeeded or not.
We change the standard definition and have typed terms as inputs and
outputs.

For that, we let $\lterm(\varphi)$ denote the set of closed, local terms of
type~$\varphi$,
and $\lval(\varphi)$ denote the set of normal terms in $\lterm(\varphi)$.
For a finite set of processes~$\processes$, 
a \textit{typed input-output problem} consists of each process's input type
$(\iota_i)_{i\in \processes}$, each process's output type $(o_i)_{i\in
\processes}$, and a
task $R\subseteq \prod_{i\in \processes}\left(\lterm(\iota_i)\right)\times
 \prod_{i\in \processes}\left(\lval(o_i)\right)$.

\subsection{Typed Protocol}

We assume a finite set~$\processes$
of processes and a countably infinite
set of program variables $\ProV =\{\p x, \p y, \p z, \ldots\}$.
We assume an injection from variables to program variables $x\mapsto
\p{x}_x$, whose image leaves infinitely many unused program variables.

A \textit{program} is defined by BNF:
\[
 p ::= \epsilon\mid
 \p x\leftarrow E; p \mid
 l \leftarrow E; p
\]
where an \textit{expression} is
\begin{align*}
 E
 ::=\,\,
 &x\mid \p x \mid l \mid (EE)\mid \lambda
 x.E\mid \tuple{E,E}\mid \linl{E}\mid \linr{E}\mid \\
 &\lpil{E}\mid\lpir{E}\mid  \lmat E x {E} y {E}\mid \epsilon\enspace.
\end{align*}

\newcommand{\Wg}{W_{\mathrm g}}
\newcommand{\Wd}{W_{\mathrm d}}
A program is \textit{well-formed} when
a program variable (resp. location) first appears in a $\p x\leftarrow E$
(resp. $l\leftarrow E$)
sentence, and
after that, only appears in expressions.
For a contexted type $t = (\Gamma\tr\m\varphi)$,
we write $\tj{M}{t}$ for
$\Gamma\tr\tj{M}{\m\varphi}$.
For input types $(\iota_i)_{i\in\processes}$
and output types $(o_i)_{i\in\processes}$, 
a \textit{typed protocol} has:
\begin{itemize}
 \item two program variables
      $\p i_i$ and $\p o_i$ for each process~$i$;
 \item a finite set of locations~$L$;
 \item two functions $g\colon L\rightarrow \processes$
       and $d\colon L\rightarrow
       \processes$ (specifying the left side writer and the right side writer of
       each location);
       $g(l)$ might read what $d(l)$ writes and vice versa;
 \item $\Wg, \Wd$: each maps a location in $L$ to a contexted type;
       $g(l)$ writes a term of $\Wg(l)$ and $d(l)$ writes a term of $\Wd(l)$;
 \item a function $t_i$ for each $i\in \processes$;
       that maps a program variable to a contexted type
       $(\tj{x_k}{\varphi_k})_k \tr[i]\varphi$ with a special condition
       $t_i(\p i_i)= \iota_i$;
 \item a typed program~$p_i$ for each $i\in \processes$,
       where
a \textit{typed program} is a well-formed program where all
sentences are typed according to $(t,g,d,i,\Wg, \Wd)$.
A sentence $\p x \leftarrow E$ is typed  iff $E\colon t(\p x)$ is derivable with
assumptions of the form $\tr\tj{\p y}{t(\p y)}$,\quad
$\tr\tj{l}{W_{\mathrm d}}(l)$ and
$\tr\tj{l}{W_{\mathrm g}}(l)$.
\end{itemize}

\subsection{Typed Waitfree Computation}

We define when a protocol solves a typed
input-output problem.
These definitions are transferred from \cite{Saks:1993vq}.

Let $\processes$ be $\{0,\ldots, n-1\}$ and fix
a typed protocol.
A \textit{program variable content} $m$ for $i\in\processes$ is a
partial
function
that maps a program variable to a term of $t_i(\p x)$.
A term~$M$ \textit{is of} a contexted type $\Gamma\tr\varphi$ when
$\Gamma\tr
\tj M\varphi$ is derivable.
A \textit{process snapshot} of $i\in\processes$ is a tuple
$\tuple{p,m}$ where $p$ is either a program or $\abort$ and $m$ is a
program variable content for $i$.
We let $S_i$ denote the set of process snapshots for~$i$.
A \textit{system snapshot}
is a pair $\tuple{\vec s, \vec v}$, where $\vec s = \tuple{s_0,
{s_1,{\ldots,s_{n-1}}}} \in
\prod_{i\in \processes}\left(S_i\right)
$
and
$\vec v =
\left(\tuple{\vg{v}{l}, \vd{v}{l}}\right)_{l\in L} \in \prod_{l\in
L}\left((\val(\Wg(l))\cup\{\epsilon\})\times (\val(\Wd(l))\cup\{\epsilon\})\right)
$.

For a nonempty subset~$J$ of $\processes$, we define an operator $\update J$ that
takes a system snapshot and produces a system snapshot.
This operator depicts a computation step where the processes in~$J$
are fired.

We define $
(\vec s, \vec v) \update J = (\vec u, \vec m)
$ by defining
$u_i$ and $m_i$
where $s_i = \tuple{p,x}$:
\begin{align*}
 u_i & =
 \begin{cases}
 \tuple{p',x}  & (\text{if } p = l
 \leftarrow E; p' \text{ and }\sem{E}_{x,\vec v} \neq \epsilon)
 \\
 \tuple{\epsilon,x}  & (\text{if } p = l
 \leftarrow E; p' \text{ and }\sem{E}_{x,\vec v} = \epsilon)
 \\
 \tuple{p', x[\p x \mapsto \sem{E}_{x, \vec v}]}&
                       (\text{if } p = \p x \leftarrow E; p', \quad
 x(\p x) = \epsilon  \text{ and } \sem{E}_{x, \vec v} \neq \epsilon) \\
 \tuple{\epsilon, x}&
                      (\text{if } p = \p x \leftarrow E; p',  \quad
 x(\p x) = \epsilon  \text{ and } \sem{E}_{x, \vec v} = \epsilon) \\
 \tuple{\epsilon, x} & (\text{if } p = \p x \leftarrow E; p' \text{
 and }
 x(\p x) \neq \epsilon) \\
 s_i & (\text{if } p = \epsilon)
 \end{cases}\\
 \vg m l & =
 \begin{cases}
 \sem{E}_{x, \vec v} & (\text{if } p = l \leftarrow E; p',\quad g(l)= i \text{
 and } \vg v l = \epsilon) \\
 \vg v l &(\text{otherwise})
 \end{cases}\\
 \vd m l &=
 \begin{cases}
\sem{E}_{x, \vec v} &
  (\text{if } p = l \leftarrow E; p',\quad d(l)= i \text{
 and } \vd v l = \epsilon) \\
 \vd v l &(\text{otherwise})
 \end{cases}
\end{align*}
with the following notations.
We let $i(l)$ to be $\vg v l$ if $d(l)=i$ and
$\vd v l$ if $g(l) = i$.
The term
$\sem{E}_{x, \vec v}$ is defined as the unique normal form
%{show, show, show}
of $E[x(\vec{\p y}) / \vec{\p y}][\vec{i(l)}/\vec l]$, where
every program variable~$\p y$ is replaced by $x(\p y)$ and the
uniqueness is guaranteed by the absence of $\overleftarrow l$ or
$\overrightarrow l$.
If any of the substitutes is $\epsilon$,
$\sem{E}_{x, \vec v}$ is $\epsilon$.

A \textit{schedule} is an infinite sequence of nonempty subsets of~$\processes$,
which looks like $\sigma = \sigma_0\sigma_1\sigma_2\cdots$.
We say $i$ is \textit{nonfaulty} in $\sigma$
if it appears infinitely often.
When every process is nonfaulty, the schedule is \textit{fair}.

A \textit{run} is a triple $\tuple{\Pi, \vec x, \sigma}$,
where $\Pi$ is a typed protocol,
$\vec x \in \prod_{i\in \processes} \lterm(\iota_i)$ is the input,
and $\sigma$ is a schedule.
The \textit{execution} associated to the run
is defined as the infinite sequence of system snapshots
$C_0C_1C_2\cdots$, where $C_0 = \tuple{\vec{s^0}, \vec{v^0}}$ is
defined by $\vec{s^0_i} = \tuple{p_i, [\p i_i\mapsto x_i]}$ and
$\vg v l  = \vd v l = \epsilon$,
and $C_{k+1} = C_{k}\update
\sigma_{i+1}$.

Process~$i$'s \textit{output}~$\hat{o_k}$ at step~$k$ is
$M$ if the $i$-th process snapshot of $C_k$ is
$(p, x)$ and the $x[\p{o}_i] = M$, which can be $\epsilon$.
The decision value of $i$ on the run $\tuple{\Pi,\vec x,\sigma}$,
denoted~$d_i \in \lval(o_i)\cup\{\epsilon\}$
 is the first non-$\epsilon$ element in the sequence
 $\left(\hat{o_k}\right)_{k\in\omega}$,
 or
$\epsilon$ if such element does not exist.
The $n$-tuple~$\vec d$ is defined by $d_i$'s.

A vector $\vec b\in \prod_{i\in \processes}(\lval(o_i))$
is \textit{compatible with} $\vec d \in \prod_{i\in
\processes}\left(\lval(o_i)\cup\{\epsilon\}\right)$ iff
$d_i = b_i$ or $d_i = \epsilon$ holds for any process~$i$.
An input~$\vec x\in \prod_{i\in \processes}\lterm(\iota_i)$
is \linebreak[2] $R$-\textit{permissible} iff there is at least one
vector $\vec d\in \prod_{i\in \processes}(\lval(o_i))$ with $(\vec x, \vec b)\in R$.
A typed protocol~$\Pi$ \textit{solves} the typed input-output problem
  $\tuple{(\iota_i)_{i\in \processes}, (o_i)_{i\in \processes}, R}$ on
schedule~$\sigma$ iff for all $R$-permissible inputs~$\vec x$ and a
schedule~$\sigma$,
 the decision value of every nonfaulty process~$i$ is a term
       $M$ not $\epsilon$, and
 there is a vector $\vec b\in \prod_{i\in \processes}(\lval (o_i))$
with $\tuple{\vec x, \vec b} \in R$ which is compatible with the
 decision vector~$\vec d$.
A typed protocol is \textit{waitfree} iff it solves
the problem on every schedule $\sigma$.
In that case, the typed input-output problem is 
\textit{waitfreely solvable}.

\section{Characterization of Waitfreedom and \lgd}
\label{comparison}

We compare the ability of the waitfree protocols and \lgd\,.
\begin{definition}
 A typed input-output problem
 $\tuple{(\iota_i)_{i\in \processes}, (o_i)_{i\in \processes}, R}$ is
 solvable by a term
 $M$ of contexted type
 $\left(\tj{x_i}{[i]\iota_i}\right)_{i\in\processes}
 \tr\left(\wwedge_{i\in
 \processes}[i]o_i\right)$ 
 iff
 for any closed $(N_i)_{i\in \processes}$ of $\iota_i$,
 all normal forms of $M[\vec{N_i}/\vec{x_i}]$
 are in the form
 $\tuple{{V_0}, \tuple{{V_1},
 \cdots\tuple{{V_{n-2}},\tuple{{V_{n-1}},\bullet}}\cdots}}$
 where $\tuple{(N_i)_{i\in \processes}, (V_i)_{i\in \processes}}\in R$.
 % looks like we assume subformula property
\end{definition}

\begin{theorem}[Soundness]
If a typed input-output problem is solvable by a term,
there exists a typed protocol that solves the problem.
\end{theorem}

We are going to translate a typed hyperterm into a protocol inductively
on the type derivation.
To make induction work, we use the following auxiliary notions.
An \textit{investigator} $\tuple{i, \p x}$ is a pair of a process and a program
variable.
For a local formula~$\varphi$, a system snapshot $\tuple{\vec s,\vec v}$
satisfies
$\tuple{i,\p x}(\varphi)$ iff
$s_i(\p x)$ is an expression of $\varphi$.
For a set of investigators~$I$,
a system snapshot satisfies
$I([i]\varphi)$ iff it satisfies
$\tuple{i, \p x}(\varphi)$ for some $\p x$ with
$\tuple{i,\p x}\in I$.
This can be extended to all global formulae, as
$I(\m\varphi\wedge\m\psi)$ iff $I(\m\varphi)$ and $I(\m\psi)$;
$I(\m\varphi\vee\m\psi)$ iff $I(\m\varphi)$ or $I(\m\psi)$.
A \textit{system snapshot satisfies a global formula}~$\m\varphi$
iff there exists a
finite set of
investigators~$I$ such that the system snapshot satisfies $I(\m\varphi)$.
A \textit{system snapshot satisfies a context} iff it
satisfies every global formula in the context.
A \textit{protocol} $p$ \textit{realizes a hypersequent}~$
\left(\Gamma_0\vdash
\m\varphi_0 \hmid \cdots\hmid\Gamma_{k}\vdash \m\varphi_{k}\right)$
iff
for any initial system snapshot satisfying
every~$\Gamma_{k'}$,
there exists a family of investigator sets
$(I_{k'})_{k'\in\{0,\ldots,k\}}$ and,
when $p$ is executed with any fair schedule,
the resulting system snapshots eventually always satisfies at least one of
$\varphi_{k'}^+$.

For a typed 
hyperterm~$\hypert$,
we will give $\sem{\hypert}$, which is a tuple of programs indexed
by~$\processes$.
Also, we define $\semi{\hypert}$ at the same time as
$\sem{\hypert}$, where
$\semi{\hypert}$ is a sequence of finite sets of investigators whose
length is the same as that of $\hypert$.
We refer to the last element of $\semi{\hypert\hmid M}$ as
$\semi{\hypert\hmid \hat{M}}$, the second to last element of
$\semi{\hypert\hmid M\hmid N}$ as 
$\semi{\hypert\hmid \hat M\hmid N}$ and so on.
We denote the right projection of $\semi{\hypert\hmid \hat{M}}$ as
$\semi{\hypert\hmid \hat{M}}'$.
If two sequents of investigator sets $\semi{\hypert}$ and $\semi{\hypert'}$
have the same length, we define $\semi{\hypert}\cup \semi{\hypert'}$ to
be the elementwise union.

We let $\epsilon$ denote $(p_i)_{i\in\processes}$ where $p_i=\epsilon$
for all $i\in\processes$.
Also, $(p_i)_{i\in\processes}; (q_i)_{i\in\processes}$ denotes
$(p_i; q_i)_{i\in\processes}$ where the
same program variable does not have multiple substitutions
(we rename variables in the original typing derivation to satisfy this).
And $(p)_j$ denotes $(q_i)_{i\in\processes}$ where $q_j = p$ and $q_i =
\epsilon$ for all $i\neq j$.
Below, we always choose fresh program variables.
The definition is inductive over the type derivation.
{\footnotesize
\begin{align*}
 \sem{\cotuple{\hypert_0,\hypert_1}\hmid \ltor j l \Delta M\hmid \rtol i l\Gamma N}
 =& \sem{\hypert_0\hmid M};
 \sem{\hypert_1\hmid N};\\
 &(l\leftarrow \semi{\hypert_0\hmid\hat M}'; \p
 y\leftarrow l;)_j; \\
 &(l\leftarrow \semi{\hypert_1\hmid\hat N}'; \p
 x\leftarrow l;)_i\enspace,\\
 \semi{\cotuple{\hypert_0,\hypert_1}\hmid\ltor j l \Delta M\hmid \rtol i l\Gamma N} =&
 \semi{\hat \hypert_0\hmid M} \cup \semi{\hat \hypert_1\hmid N} \hmid\\& \tuple{\{j,\p
 y\}}\hmid \{\tuple{i,\p x}\}\enspace,\\
 \sem{\hypert\hmid \abort}=\sem{\hypert}\enspace,\quad
 \semi{\hypert\hmid\abort}=&\semi{\hypert}\hmid \emptyset\enspace,\quad
 \sem{x}= \epsilon \enspace,\\
 \semi{x}=& \{\tuple{i,\p x_x}\} \text{ where $[i]\varphi$ is the type
 of $x$}\enspace,\\
 \sem{\hypert\hmid \gpair{M,N}}=&
 \sem{\hypert\hmid M}; \sem{\hypert\hmid N}\enspace,\\
 \semi{\hypert\hmid \gpair{M,N}} =&
 \semi{\hypert\hmid M}\cup \semi{\hypert\hmid N}\enspace,\\
 \sem{\hypert\hmid \pi_a^\g(M)}=&\sem{\hypert\hmid M}\enspace,\\
 \semi{\hypert\hmid\pi_a^\g(M)}=&\semi{\hypert\hmid M}\enspace,\\
  \sem{\cotuple{\cotuple{\hypert_0,\hypert_1},\hypert_2}\hmid
   \gmat  M x {N_0} y {N_1}}
 =&\sem{\hypert_0\hmid M}; \\
  \p{x}_x \leftarrow \lmat {\semi{\hypert_0\hmid \hat M}'} {x} {x} {y} {\epsilon}; &\sem{\hypert_1 \hmid N_0}; \\
  \p{x}_y \leftarrow \lmat {\semi{\hypert_0\hmid \hat M}'} {x} {\epsilon} {y} {y}; &\sem{\hypert_2 \hmid N_1}\enspace,\\
  \semi{\cotuple{\cotuple{\hypert_0,\hypert_1},\hypert_2}\hmid \gmat  M x {N_0} y {N_1}}
 =&\semi{\hypert_1 \hmid N_0}\cup \semi{\hypert_2 \hmid N_1}\enspace,\\
 \sem{\hypert\hmid \lambda x.M}=& \sem{\hypert\hmid M}; \p z\leftarrow
 \lambda x. \semi{\hypert\hmid\hat M}' \enspace,\\
 \sem{\cotuple{\hypert_0,\hypert_1}\hmid MN}=& \sem{\hypert_0\hmid M}; \sem{\hypert_1\hmid N}; \\&
 \p z\leftarrow \semi{\hypert_0\hmid \hat M}'\semi{\hypert_1\hmid \hat N}'\enspace,\\
 \semi{\cotuple{\hypert_0,\hypert_1}\hmid MN} =& (\semi{\hat \hypert_0 \hmid M}\cup
 \semi{\hat\hypert_1\hmid N})\hmid \{\tuple{i,\p z}\}\enspace, \\
 \sem{\hypert\hmid \lpil M} =& \sem{\hypert\hmid M}; \p z \leftarrow
 \lpil {\semi{\hypert\hmid\hat M}'}\enspace, \\
 \semi{\hypert\hmid \lpil M}=& \semi{\hat \hypert\hmid M}\hmid
 \{\tuple{i, \p z}\}\enspace, \\
 \sem{\hypert\hmid \lpir M} =& \sem{\hypert\hmid M}; \p z \leftarrow
 \lpir{ \semi{\hypert\hmid\hat M}' }\enspace, \\
 \semi{\hypert\hmid \lpir M}=& \semi{\hat \hypert\hmid M}\hmid
 \{\tuple{i, \p z}\}\enspace, \\
 \sem{\hypert\hmid \linl{M}} =& \sem{\hypert\hmid M}; \p z \leftarrow
 \linl{ \semi{\hypert\hmid\hat M}'}\enspace, \\
 \semi{\hypert\hmid \linl M}=& \semi{\hat \hypert\hmid M}\hmid
 \{\tuple{i, \p z}\}\enspace, \\
 \sem{\hypert\hmid \linr M} =& \sem{\hypert\hmid M}; \p z \leftarrow
 \linr{\semi{\hypert\hmid\hat M}'}\enspace, \\
 \semi{\hypert\hmid \linr{M}}=& \semi{\hat \hypert\hmid M}\hmid
 \{\tuple{i, \p z}\}\enspace.
\end{align*} }
When $(\tj{x_i}{[i]\iota_i})_{i\in\processes}
\tr\tj{M}(\wwedge_{i\in \processes}[i]o_i)$ is
derivable,
we can define a protocol using the above translation.
We set $\mathtt i_i$ to be ${\p x}_{x_i}$, ${\p o}_i$ to be arbitrarily
chosen fresh program variables, $L$ to be the set of locations
occurring in the derivation, we set the family of programs to be
$\sem{M}; ({\p o}_i\leftarrow \pi_i(\semi{\hat M}'))_{i\in\processes}$,
where $\pi_i$ is obtained by composing $i$~times $\pi_{\mathrm r}$ to
$\pi_{\mathrm l}$.
We set $g,d,t_i$ accordingly so that the program is typed.
We can simulate a reduction sequence of
the hyperterm using a fair execution of the protocol.


And, since the protocol solves a problem for any fair schedule, it 
solves the problem waitfreely.
If we deny the claim, there must be an execution where a nonfaulty
process
either (a)~gives a
wrong answer or
 (b)~never gives an answer.
 Either case, there is a step~$k$ when
 such a failure is inevitable.
We can modify the schedule after step $k$ to a fair one,
keeeping the failing behavior of
the process.

\begin{theorem}[Completeness]
If there exists a typed protocol that solves a typed input-output
 problem, 
the problem is solvable by a term.
\end{theorem}

Saks and Zaharoglou~\cite{Saks:1993vq} showed that a finite repetition of the participating set
problem universally solves any waitfreely solvable problem.
Also, $n$-party participating problem can be solved by a tournament of
the two-party participating set problem.
It suffices to show a \lgd\, term solving the two-party problem.


In the \textit{participating set problem}~\cite{borowsky},
each process~$i$ receives an id $c_i$ and
returns a set of id's $S_i$.
The outputs must satisfy (i)~$i\in S_i$; (ii)~either $S_i\subseteq S_j$
or $S_j\subseteq S_i$; and (iii)~$S_i\subseteq S_j$  if $i\in S_j$ for any
$i,j\in\processes$.
For two processes,
$\tuple{S_0, S_1}$ can be $\tuple{\{c_0\}, \{c_0, c_1\}}$, $\tuple{\{c_0, c_1\}, \{c_1\}}$
or
$\tuple{\{c_0, c_1\}, \{c_0, c_1\}}$.

We are going to encode the participating set problem in \lgd.
For this, we introduce a base type called $\Id$ for process id's.
Let there be an injection that maps a natural number~$i$ to a constant
$C_i\colon\Id$.
The additional typing rules involving $\Id$ are as follows, where $2 = (\bot\supset\bot)\vee(\bot\supset\bot)$:
\begin{center}
 \UnaryRule{}{}
 {$\tr \tj{c_n}{[i]\Id}$}
 \hfill
 \BinaryRule
 {$\Gamma\tr \tj{M_0}[i]\Id$}
 {$\Gamma\tr \tj{M_1}[i]\Id$}
 {}
 {$\Gamma \tr \tj{\compare{M_0}{M_1}}{[i]2}$}\enspace.
\end{center}
The additional reduction is
\[
 c_m == c_n \reduce 
\begin{cases}
 \linl{\lambda x.x}& (\text{if } m = n)\\
 \linr{\lambda x.x}& (\text{otherwise})\enspace.
\end{cases}
\]
Also, 
${\ifte M {N_0} {N_1}}$
is an abbreviation for
${\lmat M x {N_0} y {N_1}}$.

We represent a finite set of id's as a
typed lambda term, whose type is $[i](\Id\supset 2)$.  Intuitively, a
set takes an id and decides whether it is \textit{in} or \textit{out}.
The emptyset is represented by a term $\lambda x. \linr {\bullet}$.  
When a finite set~$S$ is represented by a term~$M$,
the set $S \cup \{c\}$ is represented by a term
$\lambda x.\left(\ifte{x==c}{\linl {\bullet}}{Mx}\right)$.
With the above construction, we define abbreviations
like $\{c_0, c_1, c_2\}$.

Now, we are ready to construct a hyperterm solving the two-party
participating set problem.
We can  obtain a derivation of:
\begin{center}
 \small
$
 \tj{x}{[0]\Id},
 \tj{y}{[1]\Id}
 \tr
 \tj{\cotuple{\gpair{\{\ltor 0 l \epsilon x, x\}, \{y\}},
 \gpair{\{x\}, \{\rtol 1 l\epsilon y, y\}}
 }}
 {[0](\Id\supset 2)\wedge ([1](\Id\supset 2))}\enspace.$
\end{center}

One possible reduction sequence is as follows:
\begin{align*}
 & \conf{[]}{[]}
 {\cotuple{\gpair{\{\ltor 0 l \epsilon {c_0}, {c_0}\}, \{{c_1}\}},
 \gpair {\{{c_0}\}, \{\rtol 1 l\epsilon {c_1}, {c_1}\}}}} \\
 \reduction &
 \conf{[l\mapsto {c_0}]}{[]}
 {
 \cotuple{
 \gpair{\{\abort, {c_0}\}, \{{c_1}\}},
 \gpair{\{{c_0}\}, \{\rtol 1 l\epsilon {c_1}, {c_1}\}}
 }
 }
 \\
 \reduction &
 \concreteconf{[l\mapsto {c_0}]}{[l\mapsto {c_1}]}{\gpair{\{{c_0}\}, \{c_0, c_1\}}}\enspace.
\end{align*}
Moreover, the same initial configuration can reduce to
\begin{Equation*}
 \concreteconf{[l\mapsto c_0]}{[l\mapsto c_1]}{\gpair{\{c_1,c_0\},
 \{c_1\}}}\text{ or }
 \concreteconf{[l\mapsto c_0]}{[l\mapsto c_1]}{\gpair{\{c_1,c_0\},
 \{c_0, c_1\}}}\enspace. 
\end{Equation*}
There are no other normal forms.
These three normal forms correspond to the three answers for the
two-party participating set problem.

\section{Related Work}
\label{related}
Avron~\cite{avron91} formulates a
hypersequent calculus for G\"odel--Dummett logic and proves
cut-elimination theorem using a method
similar to Gentzen~\cite{gentzen}.
Also, he explains the intuition behind the communication rule as
``the inputs through the ports in $\Gamma_2'$ are transmitted to the
component with output of type $A_1.$  The inputs through $\Gamma_1'$ are
treated similarly.''  He did not mention the possibility of
any transmission failures, which we exploited
in order to characterize waitfreedom.
Ciabattoni, Galatos and Terui~\cite{alg} gives a class of logics
that have
hypersequent calculi with
cut-elimination.
Their cut-elimination proof is general but it does not
obviously reveal the computational content.

Baaz, Ciabattoni and Ferm\"uller~\cite{natural} propose a 
hypersequent-style natural deduction for G\"odel--Dummett logic, but
did not define reduction.
Ferm\"uller~\cite{parallel} gives a game semantics for G\"odel--Dummett
logic, which is based on Lorenzen game~\cite{curryhoward} and essentially
proof searching bottom-to-up.

Among numerous typed programming languages with parallelism,
to our knowledge, none exhibits
the connection of G\"odel--Dummett logic and waitfreedom.
Abramsky~\cite{abramsky1993computational}'s calculus $\mathsf{PE}_2$
for classical linear logic is
deterministic
\cite[Theorem~7.9]{abramsky1993computational} so that it is
impossible to model
waitfreedom using $\mathsf{PE}_2$.
The $\pi$-calculus~\cite{milner1999communicating}, 
Join calculus~\cite{join},
and even asynchronous
$\pi$-calculus \cite{hondatokoro}
have too strong synchronization abilities to model waitfreedom because
a process can wait for an input.

Hirai~\cite{hirailpar} compares the temporal order of waitfree
computation and the Kripke models of a modal logic similar to
G\"odel--Dummett logic.  The current
work witnesses the constructive content of
his model theoretic comparison.

\section{Future Work}
\label{future}

As a programming language, \lgd\, allows efficient execution because it
requires no synchronization among processes.
We implemented a calculus similar to \lgd\, in a programming language
Haskell%
\footnote{Given Haskell Platform, a command \texttt{cabal
waitfree} installs the implementation.}.
A possible extension is adding synchronization primitives.
It would be interesting to compare different synchronization primitives
and different intermediate logics, generalizing waitfreedom and
G\"odel--Dummett logic.

We are also planning to develop a waitfree protocol verification mechanism in Coq
because it is valuable to
remove unnecessary synchronization while keeping the program correct
in high performance computing.

An anonymous refree pointed out that the introduction of
modalities is interesting on its own.
We have not investigated the semantics of these modalities.

In \lgd, the source of nondeterminism can be explicitly expressed as the
store prophecy.
If we can find a semantic counterpart $\mathsf{Sch}$ of the store
prophecy, possibly, we
can obtain a denotation $\mathcal{D}^\mathsf{Sch}$ of terms
using a denotation $\mathcal{D}$ for normal forms.
If that succeeds for classical logic, it will be interesting%
\footnote{Kazushige Terui suggested the potential impact for classical logic.}%
.

\section{Conclusion}
\label{conc}
We proposed \lgd, a lambda calculus
based on hypersequent calculus of 
G\"odel--Dummett logic.
We proved normalization and non-abortfullness.
The calculus characterizes
the typed version of waitfree computation.
Our result 
hints broader correspondence between
proof theory and distributed computation.

\section{About the modality}

consider the Kripke model.

translation back and forth.
